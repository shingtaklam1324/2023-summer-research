\section{Flows and the Lie Derivative}

\subsection{Flows}

Let \(v\) be a vector field on \(X\).

\begin{definition}
    [integral curve] An integral curve of \(v\) is a smooth curve \(\gamma : (-\epsilon, \epsilon) \to X\), such that

    \[\dot\gamma(t) = v(\gamma(t))\]
\end{definition}

\begin{definition}
    [local flow]

    A local flow of \(v\) is a smooth map \(\Phi : U \to X\), where

    \begin{enumerate}
        \item \(U \subseteq X \times \R\) is an open neighbourhood of \(X \times 0\), and \(U \cap \{p\} \times \R\) is connected for all \(p \in X\).
        \item \(\Phi(\cdot, 0) = \id\),
        \item \(\dv{t}\Phi(p, t) = v(\Phi(p, t))\) for all \((p, t) \in U\).
    \end{enumerate}

    We will write \(\Phi^t = \Phi(\cdot, t)\).
\end{definition}

\begin{lemma}
    Local flows always exist.
\end{lemma}

\begin{lemma}
    Any local flow \(\Phi : U \to X\) of \(v\) satisfies \(\Phi^s \circ \Phi^t = \Phi^{s + t}\), whenever this makes sense.
\end{lemma}

\begin{definition}
    [complete vector field]

    A vector field \(v\) is complete if it admits a global flow, i.e. a flow defined on \(X \times \R\).
\end{definition}

\begin{lemma}
    Compactly supported vector fields are complete.
\end{lemma}

\begin{definition}
    [exponential map]

    Define

    \[\exp(tX) = \Phi^t\]

    for the one-parameter group of diffeomorphisms given by the flow of \(X\).
\end{definition}

\subsection{Lie derivative}

Let \(v\) be a vector field, with flow \(\Phi\).

\begin{definition}
    [Lie derivative]

    The Lie derivative of a tensor \(T\) along \(v\) is

    \[\mcL_vT = \dv{t}\bigg\vert_{t=0}(\Phi^t)^*T\]
\end{definition}

\begin{remark}
    The brackets in the above expression is

    \[\dv{t}\bigg\vert_{t=0}\left((\Phi^t)^*T\right)\]
\end{remark}

\begin{lemma}
    For general \(t\), we have

    \[\dv{t}(\Phi^t)^*T = (\Phi^t)^*\mcL_vT\]
\end{lemma}

\begin{lemma}
    If \(f\) is a function on \(X\), then \(\mcL_v(f) = \dd f(v)\). If \(\alpha = \alpha_i \dd x^i\) is a \(1\)-form, then

    \[\mcL_v\alpha = \left(v^j\pdv{\alpha_i}{x_j} + \alpha_j\pdv{v_j}{x^i}\right)\dd x^i\]
\end{lemma}

\begin{lemma}
    For a vector field \(w\), and a \(1\)-form \(\alpha\), we have

    \[\mcL_v(w^i\alpha_i) = (\mcL_v w)^i \alpha_i + w^i(\mcL_v\alpha)_i\]

    and if \(S, T\) are tensors, then

    \[\mcL_v(S \otimes T) = (\mcL_v S)\otimes T + S \otimes (\mcL_v T)\]
\end{lemma}

\begin{corollary}
    If \(v, w\) are vector fields, then

    \[\mcL_v(w) = \left(v^j \pdv{w^i}{x^j} - w^j\pdv{v^i}{x^j}\right)\dd x^i\]
\end{corollary}

\begin{definition}
    [Lie bracket]

    The Lie bracket of vector fields \(v, w\) is

    \[[v, w] = \mcL_vw = -\mcL_w v\]

    This makes the space of vector fields on \(X\) into a Lie algebra.
\end{definition}

\begin{lemma}
    Let \(F : X \to Y\) be a diffeomorphism, \(v\) a vector field on \(Y\), \(T\) a tensor on \(Y\), then

    \[F^*(\mcL_vT) = \mcL_{F^*v}(F^*T)\]
\end{lemma}

\begin{definition}
    Given a vector field \(v\), and an \(r\)-form \(\alpha\), \(\iota_v\alpha\) or \(v\lrcorner\alpha\) is the \((r-1)\)-form defined by

    \[(\iota_v\alpha)_{i_1\cdots i_{r-1}} = v^j\alpha_{ji_1\cdots i_{r-1}}\]
\end{definition}

\begin{proposition}
    [Cartan's magic formula]

    \[\mcL_v\alpha = \dd(\iota_v\alpha) + \iota_v(\dd\alpha)\]
\end{proposition}
