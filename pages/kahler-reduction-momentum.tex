\documentclass{article}

\usepackage{../Style}

\DeclareMathOperator{\SU}{SU}
\newcommand{\su}{\mathfrak{su}}

\renewcommand{\sl}{\mathfrak{sl}}

\DeclareMathOperator{\gr}{grad}

\newcommand{\iinner}[1]{\left\langle\!\left\langle #1 \right\rangle\!\right\rangle}

\title{K\"ahler reduction with momentum}

\author{Shing Tak Lam}

\begin{document}

\maketitle

In this note, we will generalise the K\"ahler quotient construction to more general \(\mu^{-1}(\xi)\), for \(\xi \in \su(n)\). We will assume the K\"ahler reduction construction, and that (co)adjoint orbits of \(\su(n)\) are K\"ahler manifolds.

Let \(X\) be a K\"ahler manifold, where \(\SU(n)\) acts on \(X\) preserving the K\"ahler structure, and with a moment map \(\mu_X : X \to \su(n)\). Note that we use the minus the Killing form

\[\inner{\xi, \eta} = -\kappa(\xi, \eta) = -\tr(\xi\eta)\]

to define an inner product on \(\su(n)\), and hence an isomorphism \(\su(n) \cong \su(n)^*\).

Let \(H\) be a subgroup of \(\SU(n)\) preserving \(\mu_X^{-1}(\xi)\). Since \(\mu_X\) is equivariant with respect to the \(\SU(n)\) action and the adjoint action, preserving \(\mu_X^{-1}(\xi)\) is equivalent to

\[\Ad_h(\xi) = \xi \text{ for all }h \in H\]

Assuming \(H\), acting via the adjoint action, fixes \(\xi\), we would like to make \(\mu_X^{-1}(\xi)/H\) into a K\"ahler manifold.

\begin{lemma}
    Suppose \(X, Y\) are K\"ahler manifolds, \(\SU(n)\) acts on \(X\) and \(Y\) preserving the K\"ahler structure, and we have moment maps \(\mu_X : X \to \su(n)\) and \(\mu_Y : Y \to \su(n)\). Then the map

    \[\mu(x, y) = \mu_X(x) + \mu_Y(y)\]

    defines a moment map for the action

    \[g\vdot(x, y) = (g\vdot x, g\vdot y)\]

    of \(\SU(n)\) on \(X \times Y\).
\end{lemma}

Note that the product of complex manifolds is a complex manifold, with the natural choice of coordinate charts. If we consider the decomposition

\[\T_{(x, y)}(X \times Y) \cong \T_xX \oplus \T_yY\]

induced by projection maps, then the Riemannian metric is defined by

\[g((x_1, y_1), (x_2, y_2)) = g_X(x_1, x_2) + g_Y(y_1, y_2)\]

where \(g_X, g_Y\) are the Riemannian metrics on \(X, Y\) respectively. Similarly, the symplectic form is defined by

\[\omega((x_1, y_1), (x_2, y_2)) = \omega_X(x_1, x_2) + \omega_Y(y_1, y_2)\]

\begin{proof}
    \textbf{Equivariance} follows immediately from the equivariance of \(\mu_X\) and \(\mu_Y\).

    \[\mu(g\vdot (x, y)) = \mu(g \vdot x, g \vdot y) = \mu_X(g \vdot x) + \mu_Y(g \vdot y) = \Ad_g\mu_X(x) + \Ad_g\mu_Y(y) = \Ad_g(\mu_X(x) + \mu_Y(y)) = \Ad_g(\mu(x,y))\]

    \textbf{Hamiltonian function}

    Next, fix \(\eta \in \su(n)\). Let \(U^\eta\) be the vector field on \(X\) generated by \(\eta\), and \(Y^\eta\) the vector field on \(V\) generated by \(\eta\). Define \(\mu_X^\eta(p) = \inner{\mu_X(p), \eta}\). Then we have that for any \(W \in \TT_pX\),

    \[(\dd\mu_X^\eta)_p(W) = \omega_X(U^\eta_p, W)\]

    and a similar statement holds for \(\mu_Y\). If we now define

    \[\mu^\eta(x, y) = \inner{\mu_X(x) + \mu_Y(y), \eta} = \mu_X^\eta(x) + \mu_Y^\eta(y)\]

    Then

    \begin{align*}
        (\dd\mu^\eta)_{(x, y)}(u, v) &= (\dd\mu_X^\eta)_x(u) + (\dd\mu_Y^\eta)_y(v)\\
        &= \omega_X(U_x^\eta, u) + \omega_Y(V_y^\eta, v) \\
        &= \omega((U^\eta, V^\eta)_{(x, y)}, (u, v))
    \end{align*}

    But the vector field on \(X \times Y\) generated by \(\eta\) is precisely \((U^\eta, V^\eta)\).
\end{proof}

\begin{lemma}
    Let \(\xi \in \su(n)\), and \(M\) be its adjoint orbit, with the symplectic structure given by the Kirillov-Kostant-Souriau form

    \[\omega_\xi([\xi, \eta], [\xi, \zeta]) = -\inner{\xi, [\eta, \zeta]}\]

    Then \(\SU(n)\) acts on \(M\) via the adjoint action preserving the K\"ahler structure, and with moment map \(\mu(\xi) = -\xi\).
\end{lemma}

\begin{proof}
    We will omit the proof that the adjoint action preserves the K\"ahler structure. It is clear that \(\mu(\xi) = -\xi\) is equivariant, since both the actions are the adjoint action. 
    
    Fix \(\eta \in \su(n)\). The vector field on \(M\) generated by \(\eta\) is precisely \(U^\eta_\xi = [\xi, \eta]\). Set \(\mu^\eta(\xi) = \inner{\mu(\xi), \eta} = -\inner{\xi, \eta}\). This extends to a linear map on \(\su(n)\), and so we have that

    \[(\dd\mu^\eta)_\xi([\xi, \zeta]) = -\inner{[\xi, \zeta], \eta} = -\inner{\xi, [\zeta, \eta]} = \omega_{\xi}([\xi, \zeta], [\xi, \eta])\]

    as required.
\end{proof}

\begin{theorem}
    [K\"ahler reduction with momentum] Suppose \(X\) is a K\"ahler manifold, where \(\SU(n)\) acts on \(X\) preserving the K\"ahler structure. Let \(\xi \in \su(n)\), and

    \[H = \left\{h \in G \mid \Ad_h(\xi) = \xi\right\}\]

    is the stabiliser for the adjoint action of \(\SU(n)\). Let \(\mu_X : X \to \su(n)\) be a moment map for the \(\SU(n)\) action on \(X\), and suppose \(\SU(n)\) acts freely on \(\mu^{-1}(\xi)\). Then \(\mu_X^{-1}(\xi)/H\) is a K\"ahler manifold.
\end{theorem}

\begin{proof}
    Let \(M\) be the adjoint orbit of \(\xi\). Combining the previous lemmas, we have that \(\SU(n)\) acts on \(X \times M\) preserving the K\"ahler structure, with moment map

    \[\mu(p,\eta) = \mu_X(p) - \eta\]

    \textbf{Step 1: \(\SU(n)\) acts freely on \(\mu^{-1}(0)\).} Let \((p, \eta) \in \mu^{-1}(0)\), and \(g \in \SU(n)\) fixing \((p, \eta)\). That is,
    
    \begin{align*}
        g \vdot p &= p \\
        \Ad_g(\eta) &= \eta
    \end{align*}

    Say \(\eta = \mu_X(p) = \Ad_h(\xi)\). Then

    \[h^{-1}gh \vdot (h^{-1} \vdot p) = h^{-1}\vdot p\]

    But

    \[\mu_X(h^{-1}\vdot p) = \Ad_{h^{-1}}\mu_X(p) = \Ad_{h^{-1}}\Ad_h(\xi) = \xi\]

    and as \(\SU(n)\) acts freely on \(\mu^{-1}(\xi)\), we must have that \(h^{-1}gh = 1\). So \(g = 1\). Using this, we can take the symplectic quotient \(\mu^{-1}(0)/\SU(n)\).

    \textbf{Step 2: Bijection \(\mu^{-1}(0)/\SU(n) \cong \mu_X^{-1}(\xi)/H\)}. Define
    
    \begin{align*}
        F : \mu_X^{-1}(\xi) &\to \mu^{-1}(0) \\
        F(p) &= (p, \xi)
    \end{align*}

    Then for \(h \in H\),

    \[F(h \vdot p) = (h \vdot p, \xi) = (h \vdot p, \Ad_h(\xi)) = h \vdot F(p)\]

    Therefore, we have a smooth map \(\Phi\) making the diagram

    % https://q.uiver.app/#q=WzAsNCxbMCwwLCJcXG11X1heey0xfShcXHhpKSJdLFswLDIsIlxcbXVfWF57LTF9KFxceGkpL0giXSxbMiwwLCJcXG11XnstMX0oMCkiXSxbMiwyLCJcXG11XnstMX0oMCkvXFxTVShuKSJdLFswLDIsIkYiXSxbMCwxLCJcXHBpX1giLDIseyJzdHlsZSI6eyJoZWFkIjp7Im5hbWUiOiJlcGkifX19XSxbMiwzLCJcXHBpIl0sWzEsMywiXFxQaGkiLDJdXQ==
\[\begin{tikzcd}
	{\mu_X^{-1}(\xi)} && {\mu^{-1}(0)} \\
	\\
	{\mu_X^{-1}(\xi)/H} && {\mu^{-1}(0)/\SU(n)}
	\arrow["F", from=1-1, to=1-3]
	\arrow["{\pi_X}"', two heads, from=1-1, to=3-1]
	\arrow["\pi", from=1-3, to=3-3]
	\arrow["\Phi"', from=3-1, to=3-3]
\end{tikzcd}\]

    commute, where \(\pi_X, \pi\) are the quotient maps. We would like to show that \(\Phi\) is a bijection.

    \textbf{Injectivity.} If \(\Phi([p]) = \Phi([q])\), then \(\pi(p, \xi) = \pi(q, \xi)\). Therefore, there exists \(g \in \SU(n)\) such that

    \[(p, \xi) = (g \vdot q, \Ad_g\xi)\]

    Since \(\Ad_g(\xi) = \xi\), \(g \in H\). Therefore, \(p\) and \(q\) are in the same \(H\)-orbit.

    \textbf{Surjectivity.} Let \([(q, \eta)] \in \mu^{-1}(0)/\SU(n)\). Then \(\eta\) is in the adjoint orbit \(M\), so there exists \(g \in \SU(n)\) such that \(\eta = \Ad_g(\xi)\). In this case,

    \[\mu_X(g^{-1} \vdot q) = \Ad_{g^{-1}}\mu(q) = \Ad_{g^{-1}}\Ad_g(\xi) = \xi\]

    and so \(g^{-1} \vdot q \in \mu_X^{-1}(\xi)\). In this case,

    \[\Phi([g^{-1} \vdot q]) = [(g^{-1} \vdot q, \xi)] = [(q, \eta)]\]

    and so \(\Phi\) is a bijection.

    \textbf{Step 3: \(\Phi\) is a diffeomorphism.} Since \(\Phi\) is a smooth bijection, suffices to show that it is a submersion. As \(\pi_X\) is a surjective submersion, suffices to show that \(\pi \circ F\) is a submersion. The map
    
    \begin{align*}
        \widehat{\Ad} : \SU(n) &\to M \\
        g &\mapsto \Ad_g(\xi)
    \end{align*}

    is a submersion, therefore there exists a local right inverse \(\sigma\), with \(\sigma(\xi) = 1\). So \(\Ad_{\sigma(\eta)}(\xi) = \eta\). Then for \(\eta\) sufficiently close to \(\xi\), with \(\mu_X(q) = \eta\),

    \[\mu_X(\sigma(\eta)^{-1} \vdot q) = \Ad_{\sigma(\eta)^{-1}}\mu_X(q) = \xi\]

    Define the map

    \begin{align*}
        \psi(\eta) = \sigma(\eta)^{-1} \vdot q
    \end{align*}

    Then we have that

    \[\pi(F(\psi(\eta))) = [(\sigma(\eta)^{-1} \vdot q, \xi)] = [q, \Ad_{\sigma(\eta)}\xi] = [(q, \eta)]\]

    Therefore, if \(\alpha\) is a local right inverse for \(\pi\), with \(\alpha([(q, \eta)]) = \eta\), then

    \[(\pi \circ F) \circ (\psi \circ \alpha)([(q, \eta)]) = \pi(F(\psi(\eta))) = [(q, \eta)]\]

    Hence \(\psi \circ \alpha\) is a local right inverse, and so \(\pi \circ F\) is a submersion.
\end{proof}

\end{document}