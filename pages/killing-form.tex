\documentclass{article}

\usepackage{../Style}

\title{Killing form}
\author{Shing Tak Lam}

\DeclareMathOperator{\rad}{rad}

\begin{document}

\maketitle

\section{Solvability}

In this section, let \(\mfg\) be a real Lie algebra.

\begin{definition}
    [ideal] A subspace \(I \subseteq \mfg\) is an \emph{ideal} if \([I, \mfg] \subseteq I\).
\end{definition}

\begin{remark}
    Every ideal is a Lie subalgebra of \(\mfg\).
\end{remark}

\begin{definition}
    [simple] \(\mfg\) is simple if \(\mfg \ne 0\), and the only ideals of \(\mfg\) are \(0\) and \(\mfg\).
\end{definition}

\begin{definition}
    [derived series, solvable] The \emph{derived series} of \(\mfg\) is

    \[\mfg^{(0)} = \mfg \qqtext{and} \mfg^{(i+1)} = [\mfg^{(i)}, \mfg^{(i)}]\]

    Each \(\mfg^{(i)}\) is an ideal of \(\mfg\). We say that \(\mfg\) is \emph{solvable} if \(\mfg^{(n)} = 0\) for some \(n\).
\end{definition}

\begin{definition}
    [radical, semisimple]

    \(\mfg\) has a unique maximal solvable ideal, called the \emph{radical} of \(\mfg\), and denoted \(\rad(\mfg)\). We say that \(\mfg\) is \emph{semisimple} if \(\rad(\mfg) = 0\).
\end{definition}

\begin{lemma}
    \label{lem:solv}
    Suppose \(\mfg\) is a complex Lie algebra, \(\tr(\ad_x\ad_y) = 0\) for all \(x \in \mfg, y \in [\mfg, \mfg]\). Then \(\mfg\) is solvable.
\end{lemma}

\section{Killing form}

In this section, \(\mfg\) is a finite dimensional complex Lie algebra.

\begin{definition}
    [Killing form] The \emph{Killing form} of \(\mfg\) is

    \[\kappa(x, y) = \tr(\ad_x \ad_y)\]

    where \(\ad_x(y) = [x, y]\), \(\ad : \mfg \to \gl(\mfg)\) is the adjoint representation of \(\mfg\).
\end{definition}

\begin{lemma}
    \(\kappa\) defines a symmetric bilinear form on \(\mfg\). Moreover,

    \[\kappa([x, y], z) = \kappa(x, [y, z])\]
\end{lemma}

\begin{definition}
    [radical] The radical of \(\kappa\) is the ideal

    \[\rad(\kappa) = \{x \in \mfg \mid \kappa(x, y) = 0 \text{ for all }y \in \mfg\}\]
\end{definition}

\begin{theorem}
    The following are equivalent:

    \begin{enumerate}[(i)]
        \item \(\mfg\) is semisimple,
        \item \(\kappa\) is non-degenerate, that is, \(\rad(\kappa) = 0\),
        \item if \(x_1, \dots, x_n\) is a basis of \(\mfg\), then \(\det(\kappa(x_i, x_j)) \ne 0\).
    \end{enumerate}
\end{theorem}

\begin{theorem}
    Suppose \(\mfg\) is semisimple. Then there exists ideals \(I_1, \dots, I_t\) of \(\mfg\) which are simple (as Lie algebras), such that

    \[\mfg = I_1 \oplus \cdots \oplus I_t\]

    Moreover, each simple ideal of \(\mfg\) is one of the \(I_j\), and the Killing form of \(I_j\) is \(\kappa\vert_{I_j}\).
\end{theorem}

\subsection{Killing form over \(\R\)}

Now suppose instead that \(\mfg\) is a real Lie algebra.

\begin{definition}
    [abelian] \(\mfg\) is abelian if \([x, y] = 0\) for all \(x, y \in \mfg\).
\end{definition}

\begin{lemma}
    \(\mfg\) is semisimple if and only if there are no non-zero abelian ideals of \(\mfg\).
\end{lemma}

\begin{lemma}
    Any abelian ideal of \(\mfg\) is contained in \(\rad(\kappa)\).
\end{lemma}

\begin{proof}
    Let \(I \trianglelefteq \mfg\) be an abelian ideal, \(x \in I\), \(y \in \mfg\). We want to show that \(\kappa(x, y) = 0\). First, note that we have

    % https://q.uiver.app/#q=WzAsNSxbMCwwLCJcXG1mZyJdLFsyLDAsIlxcbWZnIl0sWzQsMCwiSSJdLFs2LDAsIkkiXSxbOCwwLCIwIl0sWzAsMSwiXFxhZF95Il0sWzEsMiwiXFxhZF94Il0sWzIsMywiXFxhZF95Il0sWzMsNCwiXFxhZF94Il1d
\[\begin{tikzcd}
	\mfg && \mfg && I && I && 0
	\arrow["{\ad_y}", from=1-1, to=1-3]
	\arrow["{\ad_x}", from=1-3, to=1-5]
	\arrow["{\ad_y}", from=1-5, to=1-7]
	\arrow["{\ad_x}", from=1-7, to=1-9]
\end{tikzcd}\]

    as \(I\) is an ideal, and \(I\) is abelian. Therefore, we have that \((\ad_x\ad_y)^2 = 0\). As any nilpotent endomorphism is tracefree, we must have that \(\kappa(x, y) = 0\).
\end{proof}

\begin{theorem}
    \(\mfg\) is semisimple if and only if \(\kappa\) is non-degenerate.
\end{theorem}

\begin{proof}
    Suppose \(\kappa\) is non-degenerate. Then we've shown any abelian ideal is contained in \(\rad(\kappa) = 0\), therefore we must have that \(\rad(\mfg) = 0\), i.e. \(\mfg\) is semisimple.

    On the other hand, suppose \(\rad(\kappa) \ne 0\). Let \(\mfh\) be any real Lie algebra. \(\mfh_\C = \mfh \otimes_\R \C\) be the complexification of \(\mfh\). We can make \(\mfh_\C\) into a complex Lie algebra via

    \[[v \otimes \lambda, w \otimes \mu] = [v, w] \otimes (\lambda\mu)\]

    With this, we can see that \(\mfh\) is abelian if and only if \(\mfh_\C\) is abelian, and as \([\mfh_\C, \mfh_\C] = [\mfh, \mfh]_\C\), \(\mfh\) is solvable if and only if \(\mfh_\C\) is solvable. Moreover, by the above definition of the Lie bracket, we can see that the Killing form of \(\mfg_\C\) is the complexification of the Killing form of \(\mfg\), i.e.

    \[\kappa_\C(v \otimes \lambda, w \otimes \mu) = \lambda\mu \cdot \kappa(v, w)\]

    Therefore, we get that \(\rad(\kappa_\C) = \rad(\kappa)_\C\). In particular, by \cref{lem:solv}, we see that \(\rad(\kappa)_\C\) is solvable, hence \(\rad(\kappa)\) is solvable. Therefore, \(\mfg\) is not semisimple.
\end{proof}

Moreover, we have a similar result to the complex case, in

\begin{theorem}
    Suppose \(\mfg\) is semisimple. Then there exists ideals \(I_1, \dots, I_t\) of \(\mfg\) which are simple (as Lie algebras), such that

    \[\mfg = I_1 \oplus \cdots \oplus I_t\]

    Moreover, each simple ideal of \(\mfg\) is one of the \(I_j\), and the Killing form of \(I_j\) is \(\kappa\vert_{I_j}\).
\end{theorem}

\subsection{Diagonalisation}

Recall Sylvester's law of inertia:

\begin{theorem}
    [Sylvester's law of inertia]
    Let \(A\) be a symmetric bilinear form on a finite dimensional real vector space \(V\). Then there exists a basis of \(V\) such that

    \[[A] = \begin{pmatrix}
        I_p \\ & -I_q \\ && 0
    \end{pmatrix}\]
\end{theorem}

In the complex case, we can get

\begin{corollary}
    Let \(A\) be a symmetric bilinear form on a finite dimensional complex vector space \(V\). Then there exists a basis of \(V\) such that

    \[[A] = \begin{pmatrix}
        I_{p+q} \\ & 0
    \end{pmatrix}\]
\end{corollary}

Note however a general symmetric bilinear form on a complex vector space will \emph{not} be positive definite, since \(A(iv, iv) = -A(v,v)\).

\end{document}
