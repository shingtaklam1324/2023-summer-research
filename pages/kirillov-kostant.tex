\documentclass{article}

\usepackage{../Style}

\title{Kirillov-Kostant-Souriau symplectic form}
\author{Shing Tak Lam}

\begin{document}

\maketitle

Throughout, let \(G\) be a Lie group, with Lie algebra \(\TT_eG = \mfg\).

Let \(G\) act on itself by conjugation, that is, \(C_g(h) = ghg^{-1}\). This is a smooth map, with \(C_g(e) = e\). Taking the derivative, we have

\[\Ad_g = \dd (C_g) : \mfg \to \mfg\]

The map \(\Ad : G \to \GL(\mfg)\), \(\Ad(g) = \Ad_g\) is called the \emph{adjoint representation} of \(G\). Dualising, we get the \emph{coadjoint representation}, that is, \(\Ad^* : G \to \GL(\mfg^*)\), given by

\[\Ad^*_g = (\Ad_{g^{-1}})^*\]

On the other hand, if we differentiate \(\Ad\), we get \(\ad : \mfg \to \End(\mfg)\).

\section{Infinitesimal action}

Let \(M\) be a manifold, \(\Phi : G \times M \to M\) a smooth action.

\begin{definition}
    [proper action] The action \(\Phi\) is proper if the map

    \[(g, x) \mapsto (\Phi(g, x), x)\]

    is proper, i.e. the preimage of a compact set is compact.
\end{definition}

\begin{proposition}
    Suppose \(\Phi\) is a free and proper action. Then the quotient \(M/G\) is a smooth manifold, and \(\pi : M \to M/G\) is a smooth submersion.
\end{proposition}

\begin{definition}
    [infinitesimal action] Suppose \(\Phi : G \times M \to M\) is an action. For \(\xi \in \mfg\), define \(\Phi^\xi : \R \times M \to M\)

    \[\Phi^\xi(t, m) = \Phi(\exp(t\xi), x)\]

    Then \(\Phi^\xi\) defines an \(\R\)-action, i.e. \(\Phi^\xi(t, \cdot)\) is a flow on \(M\). The corresponding vector field

    \[\xi_M(x) = \dv{t}\bigg\vert_{t = 0} \Phi^\xi(t, x)\]

    is called the infinitesimal action of \(\xi\).
\end{definition}

In particular, for the (co)adjoint representation, we find that

\[\xi_{\mfg}(\eta) = \dv{t}\bigg\vert_{t=0}\Ad_{\exp(t\xi)}(\eta) = \ad_\xi(\eta) = [\xi, \eta]\]

and if \(\inner{,} : \mfg^* \times \mfg \to \R\) is the usual pairing, then

\begin{align*}
    \inner{\xi_{\mfg^*}(\alpha), \eta} &= \inner{\dv{t}\bigg\vert_{t=0}\Ad_{\exp(t\xi)}^*(\alpha), \eta} \\
    &= \dv{t}\bigg\vert_{t=0}\inner{\Ad_{\exp(t\xi)}^*(\alpha), \eta} \\
    &= \dv{t}\bigg\vert_{t=0}\inner{\alpha, \Ad_{\exp(-t\xi)}(\eta)} \\
    &= \inner{\alpha, \dv{t}\bigg\vert_{t=0}\Ad_{\exp(-t\xi)}(\eta)} \\
    &= \inner{\alpha, -[\xi, \eta]} \\
    &= \inner{\alpha, -\ad_\xi(\eta)} \\
    &= -\inner{(\ad_\xi)^*(\alpha), \eta}
\end{align*}

So \(\xi_{\mfg^*} = -(\ad_\xi)^*\), and \(\xi_{\mfg^*}(\alpha)(\eta) = -\inner{\alpha, [\xi, \eta]}\).

\section{Coadjoint orbits}

For \(\mu \in \mfg^*\), we'll write \(\Orb(\mu) = \{\Ad_g^*(\mu) \mid g \in G\}\) for the \emph{coadjoint orbit} of \(\mu\). We will assume without proof that this is a submanifold of \(\mfg^*\), which is diffeomorphic to \(G/G_\mu\), where \(G_\mu = \{g \in G \mid \Ad_g^*(\mu) = \mu\}\) is the \emph{isotropy} or \emph{stabiliser} of \(\mu\).

\subsection{Tangent space}

Let \(\mcO\) be a coadjoint orbit, \(\mu \in \mcO\). For \(\xi \in \mfg\), let \(g(t) = \exp(t\xi)\). Then \(g'(0) = \xi\). Now define

\[\mu(t) = \Ad^*_{g(t)}\mu\]

which is a curve in \(\mcO\) (which means it is a curve in the vector space \(\mfg^*\)), with \(\mu(0) = \mu\). By definition, for any \(\eta \in \mfg\),

\[\inner{\mu(t), \eta} = \inner{\mu, \Ad_{g(t)^{-1}}\eta}\]

We can differentiate this at \(t = 0\), to get

\[\inner{\mu'(0), \eta} = -\inner{\mu, \ad_\xi(\eta)} = -\inner{(\ad_\xi)^*\mu, \eta}\]

where as usual we use the isomorphism \(\mu'(0) \in \Hom(\TT_0\R, \TT_\mu\mfg^*) \simeq \mfg^*\). This then gives us that

\[\TT_\mu\mcO = \left\{(\ad_\xi)^*(\mu) \mid \xi \in \mfg\right\}\]

Moreover, this also gives us that the infinitesimal generator is

\[\xi_{\mfg^*}(\mu) = -(\ad_\xi)^*(\mu)\]

\section{Kirillov-Kostant-Souriau symplectic form}

\begin{theorem}
    Let \(G\) be a Lie group, \(\mcO \subseteq \mfg^*\) be a coadjoint orbit. Define the \(2\)-form \(\omega\) on \(\mcO\) by

    \[\omega(\mu)(\xi_{\mfg^*}(\mu), \eta_{\mfg^*}(\mu)) = -\inner{\mu, [\xi, \eta]}\]

    Then \(\omega\) and \(-\omega\) are symplectic forms on \(\mcO\).
\end{theorem}

We will only prove the result for \(\omega\). The proof for \(-\omega\) is similar.

\subsection{\(\omega\) is well defined}

First of all, we show that \(\omega\) is well defined. That is, it is independent of the choice of \(\xi, \eta \in \mfg\).

Suppose \(\zeta \in \mfg\) is such that \(\zeta_{\mfg^*}(\mu) = \xi_{\mfg^*}(\mu)\). Then as \(\xi_{\mfg^*} = (\ad_\xi)^*\), we must have that

\[\inner{\mu, [\xi, \eta]} = \inner{\mu, [\zeta, \eta]}\]

for all \(\eta \in \mfg\).

\subsection{\(\omega\) is non-degenerate}

Since the pairing \(\inner{,}\) is non-degenerate, \(\omega(\mu)(\xi_{\mfg^*}(\mu), \eta_{\mfg^*}(\mu))\) for all \(\eta_{\mfg^*}(\mu)\) implies that \(\inner{\mu, [\xi, \eta]} = 0\), for all \(\eta\). But this then means that \(\xi_{\mfg^*}(\mu) = 0\), so \(\omega\) is non-degenerate.

\subsection{\(\omega\) is closed}

First of all, we will need some preliminary results.

\begin{lemma}
    \[(\Ad_\xi)_{\mfg^*} = \Ad_g^* \circ \xi_{\mfg^*} \circ \Ad_{g^{-1}}^*\]
\end{lemma}

\begin{proof}
    Let \(h(t) = \exp(t\xi)\). Then

    \begin{align*}
        (\Ad_g\xi)_{\mfg^*}(\mu) &= \dv{t}\bigg\vert_{t=0}\Ad^*_{gh(t)g^{-1}}(\mu) \\
        &= \dv{t}\bigg\vert_{t=0}\Ad^*_g\Ad^*_{h(t)}\Ad^*_{g^{-1}}(\mu) \\
        &= \Ad_h^* \circ \left(\dv{t}\bigg\vert_{t=0}\Ad_{h(t)}^*\right) \circ \Ad_{g^{-1}}^*(\mu) \\
        &= \Ad_h^* \circ \xi_{\mfg^*} \circ \Ad_{g^{-1}}^*(\mu)
    \end{align*}

    where we used the fact that \(\Ad_g^*\) is a linear map, so we can exchange it with the derivative operator.
\end{proof}

\begin{lemma}
    \[\Ad_g([\xi, \eta]) = [\Ad_g(\xi), \Ad_g(\eta)]\]
\end{lemma}

\begin{proof}
    First, notice that

    \[C_g(C_h(k)) = ghkh^{-1}g^{-1}= C_g(h)C_g(k)C_g(h^{-1})\]

    Differentiating this at \(h = e\) and \(k = e\) gives the result.
\end{proof}

\begin{lemma}
    \(\Ad_g^* : \mcO \to \mcO\) preserves \(\omega\), that is,

    \[(\Ad_g^*)^*\omega = \omega\]
\end{lemma}

\begin{proof}
    Evaluating \((\Ad_\xi)_{\mfg^*} = \Ad_g^* \circ \xi_{\mfg^*} \circ \Ad_{g^{-1}}^*\) at \(\nu = \Ad_g^*(\mu)\), we get

    \[(\Ad_g\xi)_{\mfg^*}(\nu) = \Ad_g^* \circ \xi_{\mfg^*}(\mu) = \dd_\mu\Ad_g^* \circ \xi_{\mfg^*}(\mu)\]

    Therefore,

    \begin{align*}
        ((\Ad_g^*)^*\omega)(\mu)(\xi_{\mfg^*}(\mu), \eta_{\mfg^*}(\mu)) &= \omega(\nu)(\dd_\mu \Ad_g^* \cdot \xi_{\mfg^*}(\mu), \dd_\mu \Ad_g^* \cdot \eta_{\mfg^*}(\mu)) \\
        &= \omega(\nu)((\Ad_g\xi)_{\mfg^*}(\nu), (\Ad_g\eta)_{\mfg^*}(\nu)) \\
        &= -\inner{\nu, [\Ad_g\xi, \Ad_g\eta]} \\
        &= -\inner{\nu, \Ad_g([\xi, \eta])} \\
        &= -\inner{\Ad_{g^{-1}}^*(\nu), [\xi, \eta]} \\
        &= -\inner{\mu, [\xi, \eta]} \\
        &= \omega(\mu)(\xi_{\mfg^*}(\mu), \eta_{\mfg^*}(\mu))
    \end{align*}
\end{proof}

For \(\nu \in \mfg^*\), define the left-invariant one-form

\[\nu_\ell(g) = (\dd_g\ell_{g^{-1}})^*(\nu)\]

for \(g \in G\). Similarly, for \(\xi \in \mfg\), let \(\xi_\ell\) be the corresponding left invariant vector field on \(G\). Then \(\nu_\ell(\xi_\ell) = \inner{\nu, \xi}\) at all \(g \in G\).

Fix \(\nu \in \mcO\), and consider the map \(\varphi_\nu : G \to \mcO\), defined by

\[\varphi_\nu(g) = \Ad_g^*(\nu)\]

We can use this to pullback \(\sigma = (\varphi_\nu)^*\omega\) to a two form on \(G\).

\begin{lemma}
    \(\sigma\) is left invariant. That is, \(\ell_g^*\sigma = \sigma\) for all \(g \in G\).
\end{lemma}

\begin{proof}
    First, notice that \(\varphi_\nu \circ \ell_g = \Ad_g^* \circ \varphi_\nu\), since

    \[\varphi_\nu(\ell_g(h)) = \Ad_{gh}^*(\nu) = \Ad_g^*\circ \Ad_h^*(\nu) = \Ad_g^*(\varphi_\nu(h))\]

    With this,

    \[\ell_g^*\sigma = \ell_g^*\varphi^*\omega = (\varphi\circ\ell_g)^*\omega = (\Ad_g^* \circ \varphi_\nu)^*\omega = (\varphi_\nu)^*(\Ad_g^*)^*\omega = (\varphi_\nu)^*\omega = \sigma\]
\end{proof}

\begin{lemma}
    \(\sigma(\xi_\ell, \eta_\ell) = -\inner{\nu_\ell, [\xi_\ell, \eta_\ell]}\).
\end{lemma}

\begin{proof}
    By left invariance of both sides, suffices to show that the result holds at \(e\). First notice that

    \[\dd_e\varphi_\nu(\eta) = \eta_{\mfg_*}(\nu)\]

    Therefore, \(\varphi_\nu\) is a submersion at \(e\). By definition of the pullback,

    \begin{align*}
        \sigma(e)(\xi, \eta) &= (\varphi_\nu)^*\omega(e)(\xi, \eta)\\ 
        &= \omega(\varphi_\nu(e))(\dd_e\varphi_\nu \cdot \xi, \dd_e\varphi_\nu\cdot\eta) \\
        &= \omega(\nu)(\xi_{\mfg^*}(\nu), \eta_{\mfg^*}(\nu)) \\
        &= -\inner{\nu, [\xi, \eta]}
    \end{align*}

    Hence

    \[\sigma(\xi_\ell, \eta_\ell)(e) = \sigma(e)(\xi, \eta) = -\inner{\nu, [\xi, \eta]} = -\inner{\nu_\ell, [\xi_\ell, \eta_\ell]}(e)\]
\end{proof}

Now for a one form \(\alpha\), we have that

\[\dd\alpha(X, Y) = X[\alpha(Y)] - Y[\alpha(X)] - \alpha([X, Y])\]

where for a smooth function \(f : M \to \R\), and a vector field \(X\) on \(M\), \(X[f] := \dd f(X)\) is a smooth function \(M \to \R\).

Since \(\nu_\ell(\xi_\ell)\) is constant, \(\eta_\ell[\nu_\ell(\xi_\ell)] = 0\). Similarly, \(\xi_\ell[\nu_\ell(\eta_\ell)] = 0\). Therefore, we have that

\[\dd \nu_\ell(\xi_\ell, \eta_\ell) = -\nu_\ell([\xi_\ell, \eta_\ell]) = \sigma(\xi_\ell, \eta_\ell)\]

Now suppose \(X, Y\) are vector fields on \(G\). We want to show that \(\sigma(X, Y) = \dd\nu_\ell(X, Y)\). As \(\sigma\) is left invariant,

\begin{align*}
    \sigma(X, Y)(g) &= (\ell_{g^{-1}}^* \sigma)(g)(X(g), Y(g)) \\
    &= \sigma(e)(\underbrace{\dd \ell_{g^{-1}} \cdot X(g)}_{=\xi}, \underbrace{\dd \ell_{g^{-1}} \cdot Y(g)}_{=\eta}) \\
    &= \sigma(e)(\xi, \eta) \\
    &= \dd\nu_\ell(\xi_\ell, \eta_\ell)(e) \\
    &= (\ell_g^*\dd\nu_\ell)(\xi_\ell, \eta_\ell)(e) \\
    &= (\dd\nu_\ell)(g)(\dd\ell_g\cdot \xi_\ell(e), \dd\ell_g\cdot\eta_\ell(e)) \\
    &= (\dd\nu_\ell)(g)(\dd \ell_g\cdot \xi, \dd\ell_g\cdot \eta) \\
    &= (\dd\nu_\ell)(g)(X(g), Y(g)) \\
    &= \dd\nu_\ell(X, Y)(g)
\end{align*}

With this, \(\dd\sigma = \dd^2\nu_\ell = 0\). Hence \((\varphi_\nu)^*\dd\omega = \dd((\varphi_\nu)^*\omega) = \dd\sigma = 0\). Since \(\varphi_\nu \circ \ell_g = \Ad_g^* \circ \ell_g\), and \(\varphi_\nu\) is a submersion at \(e\), it is infact a submersion everywhere. Moreover, \(\varphi_\nu\) is surjective, by definition.

For \(\mu \in \mcO\), and \(X, Y \in T_\mu \mcO\), we have that

\[\dd\omega(\mu)(X, Y) = \dd\omega_{\varphi_\nu(g)}(\dd\varphi_\nu(\xi), \dd\varphi_\nu(\eta)) = ((\varphi_\nu)^*\dd\omega)(g)(\xi, \eta) = 0\]

where \(g \in G\) is such that \(\varphi_\nu(g) = \mu\), which exists by surjectivity, and \(\xi, \eta \in T_gG\) such that \(\dd\varphi_\nu(\xi) = X\) and \(\dd\varphi_\nu(\eta) = Y\), which exists as \(\varphi_\nu\) is a submersion. Thus, as \(\mu \in \mcO\) is arbitrary, \(\omega\) is closed.

\end{document}
