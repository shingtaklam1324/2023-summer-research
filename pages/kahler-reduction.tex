\documentclass{article}

\usepackage{../Style}
\usepackage{stmaryrd}

\DeclareMathOperator{\SU}{SU}
\newcommand{\su}{\mfs\mfu}

\title{K\"ahler reduction}
\author{Shing Tak Lam}

\renewcommand{\tilde}{\widetilde}
\newcommand{\surj}{\twoheadrightarrow}
\DeclareMathOperator{\gr}{grad}

\begin{document}

\maketitle

Throughout,

\begin{enumerate}
    \item \((M, \omega, g, I)\) is a K\"ahler manifold,
    \item \(G\) is a compact Lie group acting on \(M\),
    \item \((M, \omega, G, \mu)\) is a Hamiltonian \(G\)-space,
    \item \(G\) acts by biholomorphisms on \(M\),
    \item \(G\) acts freely on \(\mu^{-1}(0)\).
\end{enumerate}

In particular, as \(\omega(u, v) = g(I(u), v)\), \(I\) is an isometry on \(M\), \(G\) acts by isometries on \(M\). Let \(Z = \mu^{-1}(0)/G\). Let

% https://q.uiver.app/#q=WzAsMyxbMCwwLCJcXG11XnstMX0oMCkiXSxbMCwyLCJaPVxcbXVeey0xfSgwKS9HIl0sWzIsMCwiTSJdLFswLDIsImkiLDAseyJzdHlsZSI6eyJ0YWlsIjp7Im5hbWUiOiJob29rIiwic2lkZSI6InRvcCJ9fX1dLFswLDEsIlxccGkiLDIseyJzdHlsZSI6eyJoZWFkIjp7Im5hbWUiOiJlcGkifX19XV0=
\[\begin{tikzcd}
	{\mu^{-1}(0)} && M \\
	\\
	{Z=\mu^{-1}(0)/G}
	\arrow["i", hook, from=1-1, to=1-3]
	\arrow["\pi"', two heads, from=1-1, to=3-1]
\end{tikzcd}\]

be the natural inclusion and quotient maps. In particular, note that \(\pi\) is a surjective submersion. Therefore, any tensor \(\alpha\) of type \((0, r)\) on \(Z\) is determined by its pullback \(\pi^*\alpha\).

The Marsden-Weinstein reduction theorem from symplectic geometry states that there exists a symplectic form \(\tilde\omega\) on \(Z\), such that

\[\pi^*\tilde\omega = i^*\omega\]

We will now construct the almost complex structure and Riemannian metric on \(Z\).

Since \(\pi\) is a submersion, for \(p \in \mu^{-1}(0)\), \(z = \pi(p)\), we have

\[\dd\pi_p : \TT_p\mu^{-1}(0) \surj \TT_zZ\]

Let \(V_p = \ker(\dd\pi_p)\) be the vertical bundle, and \(H_p = V_p^\perp \le \TT_p\mu^{-1}(0)\) be the horizontal bundle. Therefore, we have an isomorphism

\[\dd\pi_p\vert_{H_p} : H_p \cong \TT_zZ\]

For brevity, we write this isomorphism as

\begin{align*}
    \dd\pi_p\vert_{H_p} : H_p &\to \TT_zZ \\
    v &\mapsto v_* \\
    w^* &\mapsfrom w
\end{align*}

With this, we can see that

\[\tilde\omega(u, v) = \omega(u^*, v^*)\]

and that

\[\tilde g(u, v) = g(u^*, v^*)\]

defines a Riemannian metric on \(Z\). Therefore, the almost complex structure we want must be given by

\[\tilde I(u) = I(u^*)_*\]

Assuming this is well defined, then we have that

\begin{align*}
    \tilde\omega(u, v) &= \omega(u^*, v^*) \\
    &= g(I(u^*), v^*) \\
    &= g(\tilde I(u)^*, v^*) \\
    &= \tilde g(\tilde I(u), v)
\end{align*}

so \((\omega, g, I)\) is a compatible triple.

\begin{lemma*}
    \(I\) restricts to a map \(H_p \to H_p\).
\end{lemma*}

\begin{proof}
    Let \(N_p = \left(\TT_p\mu^{-1}(0)\right)^\perp \le \TT_pM\) be the normal bundle of \(\mu^{-1}(0) \subseteq M\). This gives us an orthogonal direct sum

    \[\TT_p M = N_p \oplus V_p \oplus H_p\]

    Fix \(X \in \mfg\). Then for \(v \in \TT_pM\),

    \[g(\gr(\mu^X), v) = \dd\mu^X(v) = \omega(X^\#, v) = g(I(X^\#), v)\]

    where \(\gr(f)\) is the \(g\)-dual of \(\dd f\). In particular, this means that \(\gr(\mu^X) = I(X^\#)\). Let \(X_1, \dots, X_k\) be a basis  of \(\mfg\), with corresponding dual basis \(\xi^1, \dots, \xi^k\). Then the moment map can be written as

    \[\mu(p) = \mu^{X_1}(p)\xi^1 + \dots + \mu^{X_k}(p)\xi^k\]

    But this means that

    \[\left\{\gr(\mu^{X_1}), \dots, \gr(\mu^{X_k})\right\} = \left\{I(X_1^\#), \dots, I(X_k^\#)\right\}\]

    is a basis of \(N_p\). As \(X_1^\#, \dots, X_k^\#\) is a basis for \(V_p\), we have that \(I\) restricts to a map \(N_p \oplus V_p \to N_p \oplus V_p\). By orthogonality, this means that \(I\) restricts to a map \(H_p \to H_p\).
\end{proof}

Therefore, the map \(\tilde I\) as above is well defined. Finally, we need to show that we have a K\"ahler structure. That is, \(I\) is integrable.

\begin{lemma*}
    Let \(M\) be a manifold, \((\omega, g, I)\) a compatible triple on \(M\). Then \((M, \omega, g, I)\) is a K\"ahler manifold if and only if \(\nabla I = 0\), where \(\nabla\) is the Levi-Civita connection induced by \(g\).

    Moreover, we have the expression

    \[\nabla I(u) = \nabla(I(u)) - I(\nabla u)\]

    and so \(\nabla I = 0\) if and only if \(\nabla(I(u)) = I(\nabla u)\) for all vector fields \(u\).
\end{lemma*}

\begin{proof}
    See Huybrechts, \S 4.A. for the first part. For the second part, see Nicolaescu page 96.
\end{proof}

\begin{lemma*}
    The Levi-Civita connection induced by \(\tilde g\) is

    \[\tilde\nabla_XY = \pr_H\left(\nabla_{X^*}Y^*\right)_*\]

    for vector fields \(X, Y\) on \(Z\), and we extend \(X^*, Y^*\) arbitrarily to a neighbourhood of \(\mu^{-1}(0) \subseteq M\). In addition, \(\pr_H : \TT_p M \to H_p\) is the orthogonal projection.
\end{lemma*}

\begin{proof}
    Omitted.
\end{proof}

Finally, we note that since \(I\) respects the orthogonal decomposition

\[\TT_pM = \left(N_p \oplus V_p\right) \oplus H_p\]

\(\pr_H\) and \(I\) commute. With this, we can now compute \(\tilde\nabla\tilde I\).


\begin{align*}
    \left(\tilde\nabla_X\tilde I(Y)\right)^* &= \pr_H\left(\nabla_{X^*}\tilde I(Y)^*)\right) \\
    &= \pr_H\left(\nabla_{X^*}I(Y^*)\right)\\
    &= \pr_H\left(I(\nabla_{X^*}Y^*)\right) \\
    &= I\left(\pr_H(\nabla_{X^*}Y^*)\right) \\
    &= \tilde I(\tilde\nabla_XY)^*
\end{align*}

Hence we have that

\[\tilde\nabla_X\tilde I(Y) = \tilde I(\tilde\nabla_XY)\]

for any vector fields \(X, Y\) on \(Z\), and so \(\tilde\nabla\tilde I = 0\), and \((Z, \tilde\omega, \tilde g, \tilde I)\) is a K\"ahler manifold.

\end{document}
