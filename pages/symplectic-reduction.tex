\documentclass{article}

\usepackage{../Style}

\title{K\"ahler reduction}
\author{Shing Tak Lam}


\newcommand{\sslash}{/\!/}
% \DeclareMathOperator{\Sympl}{Sympl}
\newcommand{\red}{\mathrm{red}}
\DeclareMathOperator{\Lie}{Lie}
\newcommand{\NN}{\mathrm{N}}
\DeclareMathOperator{\gr}{grad}

\begin{document}

\maketitle

\section{Preliminaries}

Throughout, let \((M, \omega)\) be a symplectic manifold, \(G\) a compact Lie group, \(\psi : G \to \Sympl(M, \omega)\) a symplectic (left) action of \(G\) on \(M\).

Let \(\psi(g)(p) = \psi_g(p) = \psi^p(g) = g \cdot p\). We call \(\psi^p : G \to M\) the orbit map, the orbit of \(p\) is \(G \cdot p\), and the stabiliser, or isotropy subgroup of \(G\) is denoted by \(G_p\).

For \(X \in \mfg\), define the vector field \(X^\#\) on \(M\) by

\[X^\#_p = \dv{t}\bigg\vert_{t=0}\psi(\exp(tX), p) = (\dd\psi^p)_e(X_e)\]

and write \(\hat\psi : \mfg \to \mathfrak{X} (M)\) for the map \(X \mapsto X^\#\).

\subsection{Orbit space}

In this subsection, we list some theorems, and basic facts about orbit spaces, which we shall not prove.

\begin{theorem}
    Suppose \(G\) is a compact Lie group acting smoothly, freely and properly on a smooth manifold \(M\). Then the quotient space \(M/G\) is a manifold, and has a unique smooth structure such that the natural projection \(\pi : M \to M/G\) is a smooth submersion. Moreover, \(\pi : M \to M/G\) has the structure of a principal \(G\)-bundle.
\end{theorem}

\begin{proposition}
    The stabiliser \(G_p\) of a point \(p\) is a closed Lie subgroup of \(G\). Moreover, the right action of \(G_p\) on \(G\) by right multiplication gives a compact smooth manifold \(G/G_p\).
\end{proposition}

\begin{proposition}
    The orbit \(G \cdot p\) is a properly embedded submanifold of \(M\), diffeomorphic to \(G/G_p\). Moreover, the restriction of the orbit map to \(G \cdot p\),

    \[\psi^p : G \to G \cdot p\]

    is a surjective smooth submersion, and its derivative at \(e\) induces a surjective linear map

    \[\mfg \to \TT_p(G \cdot p)\]

    given by \(X \mapsto X_p^\#\), with kernel \(\mfg_p = \Lie(G_p)\). Therefore, we have that

    \[\TT_p(G\cdot p) \cong \mfg/\mfg_p\]
\end{proposition}

\begin{proposition}
    Let \(\pi : M \to M/G\) be the projection map. Then

    \[\TT_p(G\cdot p) = \ker(\dd\pi_p)\]

    and

    \[\TT_{\pi(p)}(M/G) \simeq \frac{\TT_pM}{\TT_p(G\cdot p)}\]
\end{proposition}

\subsection{Hamiltonian vector fields}

Let \(H : M \to \R\) be a smooth function. Then

\[\dd H = \pdv{H}{x^j}\dd x^j \in \Omega^1(M)\]

is a \(1\)-form. On the other hand, if \(V\) is any vector field, then we can contract \(V\) and \(\omega\) to get a \(1\)-form

\[\iota_V \omega = V^i \omega_{ij}\dd x^j\]

Since \(\omega\) is non-degenerate, there exists a unique vector field \(X_H\) such that

\[\iota_{X_H}\omega = \dd H\]

\begin{definition}
    [Hamiltonian vector field] \(X_H\) is called a \emph{Hamiltonian vector field} with \emph{Hamiltonian function} \(H\).
\end{definition}

\subsection{Moment maps}

We say that the action \(\psi\) is \emph{Hamiltonian} if there exists a map

\[\mu : M \to \mfg^*\]

such that

\begin{enumerate}
    \item For each \(X \in \mfg\), let
    \begin{itemize}
        \item \(\mu^X : M \to \R\), \(\mu^X(p) = \inner{\mu(p), X}\) be the component of \(\mu\) along \(X\),
        \item \(X^\#\) the vector field on \(M\) generated by the one-parameter subgroup \(\{\exp(tX) : t \in \R\} \le G\), that is,
        \[X^\#_p = \dv{t}\bigg\vert_{t=0} \exp(tX)\cdot p\]
    \end{itemize}
    Then 

    \[\iota_{X^\#}\omega = \dd \mu^X\]

    That is, \(X^\#\) is a Hamiltonian vector field, with Hamiltonian function \(\mu^X\).
    \item \(\mu\) is equivariant with respect to the \(G\) action on \(M\) and the \(\Ad^*\) action on \(\mfg^*\). That is,
    % https://q.uiver.app/#q=WzAsNCxbMCwwLCJNIl0sWzIsMCwiTSJdLFswLDIsIlxcbWZnXioiXSxbMiwyLCJcXG1mZ14qIl0sWzAsMiwiXFxtdSIsMl0sWzEsMywiXFxtdSJdLFswLDEsIlxccHNpX2ciXSxbMiwzLCJcXEFkX2deKiIsMl1d
\[\begin{tikzcd}
	M && M \\
	\\
	{\mfg^*} && {\mfg^*}
	\arrow["\mu"', from=1-1, to=3-1]
	\arrow["\mu", from=1-3, to=3-3]
	\arrow["{\psi_g}", from=1-1, to=1-3]
	\arrow["{\Ad_g^*}"', from=3-1, to=3-3]
\end{tikzcd}\]
commutes for all \(g \in G\).
\end{enumerate}

\begin{definition}
    [Hamiltonian \(G\)-space, moment map] \((M, \omega, G, \mu)\) is called a \emph{Hamiltonian \(G\)-space}, \(\mu\) is called a \emph{moment map}.
\end{definition}

\subsection{Prinical bundles}

\begin{definition}
    [vertical bundle, (Ehresmann) connection] Let \(G\) be a Lie group, \(\pi : M \to M/G\) a principal \(G\)-bundle. The \emph{vertical bundle} \(V \subseteq \TT M\) is

    \[V_p = \ker(\dd\pi_p)\]

    \(V\) is a \(G\)-invariant subbundle of \(\TT M\), and an \emph{Ehresmann connection} on \(M\) is a \(G\)-invariant subbundle \(H \subseteq \TT M\) such that

    \[\TT_p M = V_p \oplus H_p\]

    for all \(p \in M\). We call \(H\) the \emph{horizontal bundle}.
\end{definition}

\begin{lemma}
    \(\dd\pi : \TT M \to \TT (M/G)\) restricts to isomorphisms

    \[\dd\pi_p : H_p \cong \TT_{\pi(p)}(M/G)\]
\end{lemma}

\begin{proposition}
    [horizontal lift] Let \(\pi : M \to M/G\) be a principal \(G\)-bundle, \(\TT M = V \oplus H\) a connection. Every vector field \(X \in \mathfrak X(M/G)\) has a unique \emph{horizontal lift}. That is, a unique smooth \(G\)-invariant vector field \(X^* \in \mathfrak X(M)\) such that

    \begin{enumerate}
        \item \(X_p^* \in H_p\)
        \item \(\dd\pi_p(X_p^*) = X_{\pi(p)}\)
    \end{enumerate}

    for all \(p \in M\). Conversely, given a \(G\)-invariant section \(Y\) of \(H\), there exists a unique vector field \(X \in \mathfrak X(M/G)\) such that \(Y = X^*\). We write \(X = \pi_*(Y)\). Moreover,

    \[(fX)^* = (f \circ \pi)X^* \quad (X + Y)^* = X^* + Y^* \quad \pi_*((f \circ \pi)Y) = f\pi_*(Y)\quad \pi_*(X + Y) = \pi_*(X) + \pi_*(Y)\]
\end{proposition}

\begin{proposition}
    Let \(\omega\) be a \(G\)-invariant \(k\)-tensor on \(M\). Then there exists a unique covariant \(k\)-tensor \(\eta\) on \(M/G\) such that

    \[\eta(X_1, \dots, X_k) \circ \pi = \omega(X_1^*, \dots, X_k^*)\]

    for all smooth vector fields \(X_1, \dots, X_k\) on \(M/G\).
\end{proposition}

\section{Symplectic reduction}

% \begin{theorem}
%     [Marsden-Weinstein reduction] Let \((M, \omega, G, \mu)\) be a Hamiltonian \(G\)-space, with \(G\) compact. Suppose \(G\) acts freely on \(\mu^{-1}(0)\). Then

%     \begin{enumerate}
%         \item the orbit space \(M_\red = \mu^{-1}(0)/G\) is a manifold,
%         \item \(\pi : \mu^{-1}(0) \to M_\red\) is a principal \(G\)-bundle,
%         \item if \(i : \mu^{-1}(0) \to M\) is the inclusion map, then there exists a symplectic form \(\omega_\red\) on \(M_\red\) such that
%         \[i^*\omega = \pi^*\omega_\red\]
%     \end{enumerate}
% \end{theorem}

% \begin{definition}
%     [reduction] \((M_\red, \omega_\red)\) is called the \emph{reduction} of \((M, \omega)\) with respect to \(G, \mu\).
% \end{definition}

\begin{proposition}
    Let \(\mfg_p\) be the Lie algebra of \(G_p\) for some \(p \in M\). Then \(\dd \mu_p : \TT _p M \to \mfg^*\) has

    \begin{align*}
        \ker(\dd\mu_p) &= \left(\TT_p(G \cdot p)\right)^{\omega_p} = \{v \in \TT_p M \mid \omega_p(u, v) = 0 \text{ for all }u \in \TT_p(G\cdot p) \le \TT_p M\} \\
        \im(\dd\mu_p) &= \Ann(\mfg_p) = \{\xi \in \mfg^* \mid \inner{\xi, X} = 0 \text{ for all }X \in \mfg_p\}
    \end{align*}
\end{proposition}

\begin{proof}
    First note that \(G \cdot p\) is a properly embedded submanifold of \(M\). For \(X \in \mfg\), we get a linear map \(X : \mfg^* \to \R\), so its derivative is itself. Therefore, we have that

    \[\omega_p(X^\#_p, v) = \dd \mu^X_p(v) = \dd(X \circ \mu)_p(v) = X(\dd\mu_p(v)) = \inner{\dd\mu_p(v), X}\]

    Moreover, we have that

    \[\TT_p(G \cdot p) = \left\{X_p^\# \mid X \in \mfg\right\}\]

    which means that

    \begin{align*}
        \ker(\dd\mu_p) &= \{v \in \TT_pM \mid \inner{\dd\mu_p(v), X} = 0 \text{ for all }X \in \mfg\} \\
        &= \{v \in \TT_pM \mid \omega_p(X_p^\#, v) = 0 \text{ for all }X \in \mfg\} \\
        &= \{v \in \TT_p M \mid \omega_p(w, v) = 0 \text{ for all }w \in \TT_p(G\cdot p)\} \\
        &= \left(\TT_p(G \cdot p)\right)^{\omega_p}
    \end{align*}

    Now let \(v \in \TT_pM\) be arbitrary. We know that the kernel of the map \(X \mapsto X_p^\#\) is \(\mfg_p\), so for all \(X \in \mfg_p\),

    \[\inner{\dd \nu_p(v), X} = \omega_p(X_p^\#, v) = \omega_p(0, v) = 0\]

    which means that \(\im(\dd\mu_p) \subseteq \Ann(\mfg_p)\). Moreover,

    \begin{align*}
        \dim(\im(\dd \mu_p)) &= \dim(\TT_p M) - \dim(\ker(\dd\mu_p)) \\
        &= \dim(\TT_pM) - \dim(\left(\TT_p(G \cdot p)\right)^{\omega_p}) \\
        &= \dim(\TT_p(G \cdot p)) \\
        &= \dim(\mfg/\mfg_p) \\
        &= \dim(\Ann(\mfg_p))
    \end{align*}

    So equality holds.
\end{proof}

\begin{lemma}
    For \(\xi \in \mfg^*\), the stabiliser \(G_\xi\) of \(\xi\), with the coadjoint action, acts freely on \(\mu^{-1}(\xi)\) if and only if \(G_p = \{e\}\) for all \(p \in \mu^{-1}(\xi)\).
\end{lemma}

\begin{proof}
    \(G_\xi\) acts freely on \(\mu^{-1}(\xi)\) if and only if \((G_\xi)_p = \{e\}\) for all \(p \in \mu^{-1}(\xi)\). Therefore, suffices to show that \((G_\xi)_p = G_\xi \cap G_p = G_p\). For one inclusion,

    \[(G_\xi)_p = \{g \in G \mid \Ad_g^*(\xi) = \xi \text{ and } g \cdot p = p\} = G_\xi \cap G_p \subseteq G_p\]

    Conversely, for the other inclusion, let \(g \in G_p\), by equivariance,

    \[\Ad_g^*(\xi) = \Ad_g^*\mu(p) = \mu(g \cdot p) = \mu(p) = \xi\]

    Hence \(g \in G_p \cap G_\xi\).
\end{proof}

\begin{lemma}
    Suppose \(G_\xi\) acts freely on \(\mu^{-1}(\xi)\). Then \(\xi\) is a regular value of \(\mu\), and \(\mu^{-1}(\xi)/G_\xi\) is a smooth manifold.
\end{lemma}

\begin{proof}
    In this case, we have that \(G_p = \{e\}\), hence \(\mfg_p = 0\). This means that \(\im(\dd\mu_p) = \Ann(\mfg_p) = \Ann(0) = \mfg^*\), hence \(\xi\) is a regular value for \(\mu\). This means that \(\mu^{-1}(\xi)\) is a submanifold of \(M\), and so the action of \(G_\xi\) on \(M\) gives a smooth action on \(\mu^{-1}(\xi)\). Since the action is free and \(G_\xi\) is compact, the quotient \(\mu^{-1}(\xi)/G_\xi\) is a smooth manifold.
\end{proof}

\begin{proposition}
    Suppose \(\xi \in \mfg^*\) is such that \(G_\xi\) acts freely on \(\mu^{-1}(\xi)\). Then

    \begin{enumerate}[(i)]
        \item \(\mu^{-1}(\xi)\) is a properly embedded submanifold of \(M\), with
        \[\TT_p\mu^{-1}(\xi) = \left(\TT_p(G \cdot p)\right)^{\omega_p}\]
        \item for each \(p \in \mu^{-1}(\xi)\), the orbit \(G_\xi \cdot p\) is a properly embedded submanifold of \(\mu^{-1}(\xi)\), and
        \[\TT_p(G_\xi \cdot p) = \left(\TT_p\mu^{-1}(\xi)\right)^{\omega_p} \cap \TT_p\mu^{-1}(\xi)\]
    \end{enumerate}
\end{proposition}

\begin{proof}
    (i) follows from the fact that \(\xi\) is a regular value of \(\mu\), and we have that

    \[\TT_p(G_\xi \cdot p) = \ker(\dd\mu_p) = \left(\TT_p(G \cdot p)\right)^{\omega_p}\]

    For (ii), let \(p \in \mu^{-1}(\xi)\). Then we have that \(G_\xi \cdot p \subseteq \mu^{-1}(\xi)\). Hence \(G_\xi \cdot p\) is a properly embedded submanifold of \(\mu^{-1}(\xi)\). In particular, \(\TT_p(G_\xi \cdot p) \subseteq \TT_p\mu^{-1}(\xi)\). Combining this with (i) gives one inclusion.

    Conversely, suppose

    \[v \in \left(\TT_p\mu^{-1}(\xi)\right)^{\omega_p} \cap \TT_p\mu^{-1}(\xi) = \TT_p(G \cdot p) \cap \ker (\dd\mu_p)\]

    Since \(v \in \TT_p(G \cdot p)\), we have that \(v = X^\#_p = (\dd\psi^p)_e(X)\) for some \(X \in \mfg\). Let \(\widehat{\Ad^*}(X) \in \mathfrak X(\mfg^*)\) be the vector field associated with the coadjoint action. Then we have that

    \[(\Ad^*)^\xi(g) = \Ad_g^*\xi = \Ad_g^*\mu(p) = \mu(\psi_g(p)) = (\mu \circ \phi^p)(g)\]

    for all \(g \in G\). Hence

    \[\widehat{\Ad^*}(X)_\xi = \dd ((\Ad^*)^\xi)_e(X) = \dd(\mu \circ \psi^p)_e(X) = \dd\mu_p(X^\#_p) = \dd\mu_p(v) = 0\]

    This means that \(X \in \mfg_\xi = \Lie(G_\xi)\). But then \(X_p^\# \in \TT_p(G_\xi \cdot p)\), and so \(v = X_p^\# \in \TT_p(G\xi \cdot p)\).
\end{proof}

\begin{theorem}
    [Symplectic reduction] Let \((M, \omega, G, \mu)\) be a Hamiltonian \(G\)-space, with \(G\) compact. Suppose \(\xi \in \mfg^*\) is such that \(G_p = \{e\}\) for all \(p \in \mu^{-1}(\xi)\). Then the orbit space \(\mu^{-1}(\xi)/G_\xi\) is a smooth manifold, and there exists a unique symplectic form \(\omega_\xi\) on \(\mu^{-1}(\xi)/G_\xi\), such that \(\pi^*\omega_\xi = i^*\omega\), where \(\pi : \mu^{-1}(\xi) \to \mu^{-1}(\xi)/G_\xi\) is the quotient map, and \(i : \mu^{-1}(\xi) \hookrightarrow M\) is the inclusion map.

    We write \(M \sslash_\xi G_\xi\) for this symplectic space.
\end{theorem}

\begin{proof}
    We know that \(\mu^{-1}(\xi)\) is a smooth manifold, on which the compact Lie group \(G_\xi\) acts smoothly and freely, which gives us a smooth manifold \(\mu^{-1}(\xi)/G_\xi\), where \(\pi : \mu^{-1}(\xi) \to \mu^{-1}(\xi)/G_\xi\) is a principal \(G_\xi\) bundle.

    Let \(\tilde \psi : G_\xi \times \mu^{-1}(\xi) \to \mu^{-1}(\xi)\) denote the action map, then \(i \circ \tilde\psi_g = \psi_g \circ i\) for \(g \in G_\xi\). Consider the \(2\)-form \(i^*\omega\) on \(\mu^{-1}(\xi)\). We have that \(i^*\omega\) is \(G_\xi\) invariant, since

    \[\tilde\psi_g^*(i^*\omega) = (i \circ \tilde \psi_g)^*\omega = (\psi_g \circ i)^*\omega = i^*\psi^*_g \omega = i^*\omega\]

    where \(\psi^*_g \omega = \omega\) since the action of \(G\) on \(M\) is symplectic. Therefore, there exists a unique \(2\)-form \(\omega_\xi\) on \(\mu^{-1}(\xi)/G_\xi\), such that \(\pi^*\omega_\xi = i^*\omega\). Moreover, since \(\pi\) is a surjective submersion, \(\pi^*\) is injective. In particular, as

    \[\pi^*(\dd\omega_\xi) = \dd(\pi^*\omega_\xi) = \dd(i^*\omega) = i^*(\dd\omega) = 0\]

    we have that \(\omega_\xi\) is closed. To show that it is non-degenerate, let \(p \in \mu^{-1}(\xi)\), and \(x = \pi(p) \in \pi^{-1}(\xi)/G_\xi\). Suppose \(v \in \TT_x\left(\mu^{-1}(\xi)/G_\xi\right)\) is such that \((\omega_\xi)_x(v, w) = 0\) for all \(w\in \TT_x\left(\mu^{-1}(\xi)/G_\xi\right)\). As \(\pi\) is a submersion, \(v = \dd\pi_p(\hat v)\) for some \(\hat v \in \TT_p\mu^{-1}(\xi)\). For any \(\hat w \in \TT_p\mu^{-1}(\xi)\),

    \[(i^*\omega)_p(\hat v, \hat w) = (\pi^*\omega_\xi)_p(\hat v, \hat w) = \omega_\xi(\dd\pi_p(\hat v), \dd\pi_p(\hat w)) = 0\]

    Therefore,

    \[\hat v \in \left(\TT_p\mu^{-1}(\xi)\right)^{\omega_p} \cap \TT_p\mu^{-1}(\xi) = \TT_p(G_\xi \cdot p) = \ker(\dd\pi_p)\]

    This means that \(v = \dd\pi_p(\hat v) = 0\), and so \(\omega_\xi\) is non-degenerate.
\end{proof}

\section{Riemannian quotient}

Throughout, let \(g\) be a Riemannian metric on \(M\), and with associated Levi-Civita connection \(\nabla\) on \(\TT M\) and tensor/duals of this. Suppose \(E\) is a subbundle of \((\TT M)^{\otimes p} \otimes (\TT ^*M)^{\otimes q}\), with the Levi-Civita connection \(\nabla\) on \(E\). Then for a section \(s \in \Gamma(E)\), and a vector field \(X\) on \(M\), we write

\[\nabla_Xs = (\nabla s)(X)\]

where \(\nabla s \in \TT^* M \otimes E\), and so we contract the \(\TT M \times \TT^*M\) factors.

\begin{definition}
    [orthogonal connection] Suppose \(M\) has a \(G\)-invariant Riemannian metric. Then we can define the \emph{orthogonal connection} as

    \[H_p = V_p^\perp\]
\end{definition}

Throughout, \(H\) will be the orthogonal connection, and \(\pr_H : V \oplus H \to H\) is the orthogonal projection.

\begin{theorem}
    [metric on quotient manifold] Suppose \(\pi : M \to M/G\) is a principal \(G\)-bundle, \(g\) is a \(G\)-invariant Riemannian metric on \(M\). Then there exists a unique Riemannian metric \(\overline g\) on \(M/G\), such that \(\overline g(X, Y) \circ \pi = g(X^*, Y^*)\) for all smooth vector fields \(X, Y\) on \(M/G\). Moreover, the Levi-Civita connection \(\overline\nabla\) of \(\overline g\) is given by

    \[\overline \nabla_X Y = \pi_*\left(\pr_H(\nabla_{X^*}Y^*)\right)\]
\end{theorem}

\begin{theorem}
    [metric on submanifold] Let \((M, g)\) be a Riemannian manifold, \(\tilde M \subseteq M\) an embedded submanifold with the induced metric \(\tilde g = i^*g\). Then the Levi-Civita connection \(\tilde\nabla\) of \(\tilde g\) is given by

    \[\tilde\nabla_X Y = \pr(\nabla_XY)\]

    where the vector fields on the right are extensions of \(X, Y \in \mathfrak X(\tilde M)\) to a neighbourhood of \(\tilde M\), and \(\pr : \TT M \to \TT \tilde M\) is the orthogonal projection.
\end{theorem}

\section{K\"ahler reduction}

For simplicity, we will assume throughout that \(\xi = 0\). We want to show

\begin{theorem}
    [K\"ahler reduction]
    Let \((M, \omega, g, I)\) be a K\"ahler manifold, and \(G\) a compact Lie group acting on \(M\) isometrically and in a Hamiltonian way. Let \(\mu\) be the moment map for the action. Suppose \(G\) acts freely on \(\mu^{-1}(0)\).

    Then the symplectic reduction \((M\sslash_0 G, \omega_0)\) and the quotient metric \(g_0\) induced by \(i^*g\), where \(i : \mu^{-1}(0) \hookrightarrow M\) is the inclusion map, make \(M\sslash G := M\sslash_0 G = \mu^{-1}(0)/G\) into a K\"ahler manifold.
\end{theorem}

\subsection{Almost complex structure}

Since we already have the symplectic form and the Riemannian metric, we need to define the almost complex structure.

\begin{lemma}
    Let \((M, g)\) be a Riemannian manifold, \(f = (f_1, \dots, f_k) : M \to \R^k\) be a smooth function with regular value \(c\). Let \(\tilde M = f^{-1}(c)\), then \(\{\gr(f_1), \dots, \gr(f_k)\}\) is a smooth global frame for the normal bundle \(\NN \tilde M\) over \(\tilde M\).
\end{lemma}

Let \(\pi : \mu^{-1}(0) \to \mu^{-1}(0)/G\) be the quotient map, \(V\) be the vertical bundle of this principal \(G\)-bundle, and \(H\) its orthogonal complement with respect to the metric \(i^*g\) on \(\mu^{-1}(0) \subseteq M\). Then for \(p \in \mu^{-1}(0)\), we have an orthogonal decomposition

\[\TT_p M = N_p \oplus V_p \oplus H_p\]

where \(N_p = (\NN \mu^{-1}(0))_p\) is the normal bundle of \(\mu^{-1}(0)\).

\begin{lemma}
    The horizonal bundle \(H\) is invariant under \(I\).
\end{lemma}

\begin{proof}
    First note that for \(X \in \mfg\) and \(Y \in \TT_pM\), we have that

    \[g(\gr(\mu^X), Y) = \dd\mu^X(Y) = \omega(X^\#, Y) = g(IX^\#, Y)\]

    where we use the fact that \(g(\gr(f), \cdot) = \dd f\) by the tangent-cotangent isomorphism given by \(g\), and that \(\mu^X\) is a Hamiltonian function for \(X^\#\). Therefore, this means that

    \[\grad(\mu^X) = IX^\#\]

    since \(g\) is non-degenerate. Now let \(X_1, \dots, X_k \in \mfg\) be a basis, with dual basis \(\xi^1, \dots, \xi^k\) for \(\mfg^*\). Then we have that

    \[\mu(p) = \mu^{X_1}(p)\xi^1 + \dots + \mu^{X_k}(p)\xi^k\]

    Hence by the previous lemma, a global frame for \(N\) is

    \[\{\gr(\mu^{X_1}), \dots, \gr(\mu^{X_k})\} = \{IX_1^\#, \dots, IX_k^\#\}\]

    Moreover, since \(G\) acts freely on \(\mu^{-1}(0)\), the map

    \[\mfg \to \TT_p(G \cdot p) = \ker(\dd\pi_p) = V_p\]

    sending \(X \to X^\#_p\) is an isomorphism. Therefore, \(\{X^\#_1, \dots, X^\#_k\}\) is a basis for \(V_p\). Hence

    \[\{X_1^\#, \dots, X_k^\#, IX_1^\#, \dots, IX_k^\#\}\]

    is a basis for \(N_p \oplus V_p\), and so \(N_p \oplus V_p\) is invariant under \(I\). Now for \(v \in H_p\), \(w \in N_p \oplus V_p\), we have that

    \[g(I(v), w) = -g(v, I(w)) = 0\]

    Hence \(I(v) \in (N_p \oplus V_p)^\perp = H_p\).
\end{proof}

\begin{lemma}
    Let \(X\) be a smooth vector field on \(M\sslash G\), then the map \(\mu^{-1}(0) \to H\), given by

    \[p \mapsto I_p(X_p^*)\]

    is a smooth \(G\)-invariant section of \(H\), which we denote by \(IX^*\).
\end{lemma}

\begin{proof}
    Since \(I\) preserves \(H\), it defines a bundle isomorphism \(\overline I : H \to H\), and the map above is the composition \(\overline I \circ X^*\). This is \(G\)-invariant since \(G\) perserves \(I\) and \(X^*\) is \(G\)-invariant.
\end{proof}

With this, we can define an almost complex structure on \(M\sslash G\) by

\[I_0X = \pi_*(IX^*)\]

for all smooth vector fields \(X\) on \(M\sslash G\).

\begin{lemma}
    \(I\) defines a \((1, 1)\)-tensor field, and \(I_0^2 = -\id\).
\end{lemma}

\begin{proof}
    We need to show that \(I\) is \(C^\infty\) linear. But

    \[I_0(fX) = \pi_*(I(fX)^*) = \pi_*(I((f \circ \pi)X)^*) = \pi_*((f \circ \pi)I_X^*) = f\pi_*(IX^*) = fI_0(X)\]

    Moreover, by definition, we have that \((I_0X)^* = IX^*\), so

    \[(I_0^2X)^* = (I_0(I_0(X)))^* = I(I_0(X))^* = I(I(X^*)) = -X^* = (-X)^*\]

    and so \(I_0^2X = -X\).
\end{proof}

\subsection{Levi-Civita connection}

\begin{lemma}
    The Levi-Civita connection of \(g_0\) is given by

    \[\overline\nabla_XY = \pi_*\left(\proj_H(\nabla_{X^*}Y^*)\right)\]

    where \(X^*, Y^*\) are extensions of \(X, Y\) to a neighbourhood of \(\mu^{-1}(0)\), \(\nabla\) is the Levi-Civita connection of \(g\), and \(\proj_H : \TT M \to H\) is the orthogonal projection.
\end{lemma}

\begin{proof}
    Let \(\widetilde\nabla\) be the Levi-Civita connection on \(\mu^{-1}(0)\). Then we have that

    \[\widetilde\nabla_XY = \proj_{\TT \mu^{-1}(0)}\left(\nabla_XY\right)\]

    where on the right hand side, we extend \(X, Y\) to a neighbourhood of \(\mu^{-1}(0)\). With this, we get that the Levi-Civita connection for \(g_0\) is given by

    \[\overline\nabla_XY = \pi_*\left(\proj_H(\widetilde\nabla_{X^*}Y^*)\right) = \pi_*\left(\proj_H\left(\proj_{\TT\mu^{-1}(0)}(\nabla_{X^*} Y^*)\right)\right) = \pi_*\left(\proj_H(\nabla_{X^*}Y^*)\right)\]
\end{proof}

\begin{lemma}
    \[\overline\nabla I_0 = 0\]
\end{lemma}

\begin{proof}
    Let \(X, Y\) be smooth vector fields on \(M \sslash G\). Since \(I\) preserves \(H\), it commutes with the projection \(\proj_H : \TT M \to H\). Moreover, we have that \((I_0Y)^* = IY^*\), hence

    \[(\overline\nabla_X I_0Y)^* = \proj_H(\nabla_{X^*}IY^*) = \proj_H(I\nabla_{X^*}Y^*) = I\proj_H(\nabla_{X^*}Y^*) = I(\overline\nabla_XY)^*\]

    Therefore, \(\overline\nabla_XI_0Y = I_0(\overline\nabla_XY)\), and so \(\overline \nabla I_0 = 0\).
\end{proof}

\subsection{Compatibility}

\begin{proof}
    [Proof of K\"ahler reduction] With the above, all thet remains is to show that \((I_0, g_0, \omega_0)\) are compatible. Let \(X, Y\) be smooth vector fields on \(\mu^{-1}(0)/G\), \(p \in \mu^{-1}(0)\), we have that

    \begin{align*}
        \omega_0(X_{\pi(p)}, Y_{\pi(p)}) &= \omega_0(\dd\pi_p(X_p^*), \dd\pi_p(Y_p^*)) \\
        &= \pi^*\omega_0(X_p^*, Y_p^*) \\
        &= i^*\omega(X_p^*, Y_p^*) \\
        &= i^*g(I(X_p^*), Y_p^*) \\
        &= i^*g((I_0X)^*_p, Y_p^*) \\
        &= g_0((I_0X)_{\pi(p)}, Y_{\pi(p)})
    \end{align*}

    Hence \(\omega_0(X, Y) = g_0(I_0(X), Y)\).
\end{proof}

\end{document}
