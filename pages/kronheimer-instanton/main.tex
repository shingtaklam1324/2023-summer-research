\documentclass{article}

\usepackage{../../Style}
\usepackage{biblatex}
\addbibresource{../../bibliography.bib}

\DeclareMathOperator{\SU}{SU}
\newcommand{\su}{\mathfrak{su}}

\renewcommand{\sl}{\mathfrak{sl}}

\DeclareMathOperator{\gr}{grad}

\title{\citetitle{kronheimer_instantons_1990} by \Citeauthor{kronheimer_instantons_1990}}
\author{Shing Tak Lam}

\begin{document}

\maketitle

In this document, we will discuss the paper \cite{kronheimer_instantons_1990}. For concreteness, instead of general Lie groups and Lie algebras, we will focus on the case

\[G = \SU(n) \qquad \mfg = \su(n)\]

which has complexification

\[G^c = \SL(n, \C) \qquad \mfg^c = \sl(n, \C)\]

\section{Introduction}

The inner product on \(\su(n)\) is given by \(-\kappa\), where \(\kappa\) is the Killing form. That is,

\[\inner{A, B} = -\tr(AB)\]

Define

\begin{align*}
    \varphi : \su(n) \times \su(n) \times \su(n) &\to \R \\
    \varphi(A_1, A_2, A_3) &= \sum_{j=1}^3 \inner{A_j, A_j} + \inner{A_1, [A_2, A_3]}
\end{align*}

We are interested in studying the gradient flow of \(\varphi\). That is, \(A_1, A_2, A_3 : I \to \su(n)\) such that

\begin{equation}
    \label{eq:gradient-flow}
    (\dot A_1, \dot A_2, \dot A_3) = -\grad \varphi(A_1, A_2, A_3)
\end{equation}

First of all, notice that

\[\varphi(A_1 + H_1, A_2, A_3) = \varphi(A_1, A_2, A_3) + 2\inner{H_1, A_1} + \inner{H_1, [A_2, A_3]}\]

and that \(\inner{A_1, [A_2, A_3]} = \inner{A_2, [A_3, A_1]} = \inner{A_3, [A_1, A_2]}\). Therefore, \cref{eq:gradient-flow} becomes

\begin{equation}
    \label{eq:gradient-flow-system}
    \begin{split}
        \dot A_1 &= -2A_1 - [A_2, A_3] \\
        \dot A_2 &= -2A_2 - [A_3, A_1] \\
        \dot A_3 &= -2A_3 - [A_1, A_2]
    \end{split}
\end{equation}

The critical points of \cref{eq:gradient-flow-system} are triples \((A_1, A_2, A_3)\) satisfying

\[[A_1, A_2] = -2A_3 \quad [A_2, A_3] = -A_1 \quad [A_3, A_1] = -2A_2\]

Recall that the Lie algebra \(\su(2)\) has basis

\[e_1 = \begin{pmatrix}
    -i & 0 \\
    0 & i
\end{pmatrix} \quad e_2 = \begin{pmatrix}
    0 & 1 \\
    -1 & 0
\end{pmatrix} \quad e_3 = \begin{pmatrix}
    0 & -i \\
    -i & 0
\end{pmatrix}\]

satifying the above relations. Therefore, critical points of \cref{eq:gradient-flow-system} correspond to Lie algebra homomorphisms \(\rho : \su(2) \to \su(n)\). From this, we see that at all critical points of \cref{eq:gradient-flow-system}, \(\varphi\) is nonnegative, and it is zero only at \((0, 0, 0)\).

Next, we will identify \(\su(n) \times \su(n) \times \su(n) \cong \rm L(\su(2), \su(n))\), the space of linear maps \(\su(2) \to \su(n)\), sending \((A_1, A_2, A_3)\) to the linear map \(A\) given by \(e_i \mapsto A_i\).

The adjoint action of \(\SU(n)\) on \(\su(n)\) is given by

\[\Ad_g(A) = gAg^{-1}\]

and this induces an action on \(\rm L(\su(2), \su(n))\) by

\[g \cdot A : e_i \mapsto gA_ig^{-1}\]

For any Lie algebra homomorphism \(\rho : \su(2) \to \su(n)\), define

\[C(\rho) = \left\{g \cdot \rho \mid g \in \SU(n)\right\}\]

for the critical manifold of all homomorphisms which are conjugate to \(\rho\) via the adjoint action. For Lie algebra homomorphisms \(\rho_-, \rho_+ : \su(2) \to \su(n)\), define \(M(\rho_-, \rho_+)\) for the space of solutions \(A(t)\) to \cref{eq:gradient-flow-system}, with boundary conditions

\begin{equation}
    \label{eq:boundary-conditions}
    \begin{split}
        \lim_{t\to-\infty}A(t) &\in C(\rho_-) \\
        \lim_{t\to\infty}A(t) &= \rho_+
    \end{split}
\end{equation}

Note that we are considering parametrised trajectories, therefore there is a natural \(\R\)-action sending \(A(t)\) to \(A(t+c)\).

For a Lie algebra homomorphism \(\rho : \su(2) \to \su(n)\), we can extend it to a Lie algebra homomorphism \(\rho : \sl(2, \C) \to \sl(n, \C)\), and define

\[H = \rho\begin{pmatrix}
    1 & 0 \\
    0 & -1
\end{pmatrix}\quad X = \rho\begin{pmatrix}
    0 & 1 \\
    0 & 0
\end{pmatrix} \quad Y = \rho\begin{pmatrix}
    0 & 0 \\
    1 & 0
\end{pmatrix}\]

We will then define \(\mcN(\rho)\) for the nilpotent orbit of \(Y\) in \(\sl(n, \C)\), and the affine subspace

\[S(\rho) = Y + Z(X)\]

where \(Z(X) = \left\{A \in \sl(n, \C) \mid [A, X] = 0\right\}\). Using this, we have

\begin{theorem}
    For any pair of homomorphisms \(\rho_-, \rho_+\), there is a diffeomorphism

    \[M(\rho_-, \rho_+) \cong \mcN(\rho_-) \cap S(\rho_+)\]
\end{theorem}

If \(\rho_+ = 0\), then \(S(\rho_+) = \sl(n, \C)\), and in this case, we have a diffeomorphism

\[M(\rho_-, 0) \cong \mcN(\rho_-)\]

Moreover, every nilpotent orbit is \(\mcN(\rho)\) for some homomorphism \(\rho : \su(2) \to \su(n)\), which means that we have a description of all nilpotent orbits in \(\sl(n, \C)\).

\section{Complex trajectories}

\subsection{Nahm's equations}

Consider the change of variables

\[T_i = e^{2t}A_i \qquad s = -\frac12e^{-2t}\]

Using this, \cref{eq:gradient-flow-system} becomes

\begin{align*}
    \dv{T_1}{s} &= -[T_2, T_3] \\
    \dv{T_2}{s} &= -[T_3, T_1] \\
    \dv{T_3}{s} &= -[T_1, T_2]
\end{align*}

which are Nahm's equations.

\subsection{Gauge group}

First of all, we will extend \cref{eq:gradient-flow-system} by considering \(A_0, \dots, A_3 : \R \to \su(n)\), satisfying the equations

\begin{equation}
    \label{eq:gradient-flow-system-extended}
    \begin{split}
        \dot A_1 &= -2A_1 - [A_0, A_1] - [A_2, A_3] \\
        \dot A_2 &= -2A_2 - [A_0, A_2] + [A_1, A_3] \\
        \dot A_3 &= -2A_3 - [A_0, A_3] - [A_1, A_2] \\
    \end{split}
\end{equation}

Define the group

\[\mcG = \left\{g : \R \to \SU(n)\right\}\]

with pointwise operations. Then \(\mcG\) acts \(A = (A_0, \dots, A_3)\) by

\begin{equation}
    \label{eq:action}
    (g \vdot A)(t) = \left(g(t)A_0(t)g(t)^{-1} - \dv{g}{t}(t) \cdot g(t)^{-1}, g(t)A_1(t)g(t)^{-1}, g(t)A_2(t)g(t)^{-1}, g(t)A_3(t)g(t)^{-1}\right)
\end{equation}

For brevity, when clear, we will write this as

\[g \vdot A = (gA_0g^{-1} - \dot g g^{-1}, gA_1g^{-1}, gA_2g^{-2}, gA_3g^{-1})\]

Note that \(\dot g(t) \in \TT_{g(t)}\SU(n) = g(t)\su(n)\), and so \(\dot g(t)g(t)^{-1} \in g(t)\su(n)g(t)^{-1} = \su(n)\). First, we will show that \cref{eq:gradient-flow-system-extended} is invariant under the action \cref{eq:action}. To see this, the transformed right hand side (for the first equation) is

\begin{align*}
    -2gA_1g^{-1} - [gA_0g^{-1} - \dot g g^{-1}, gA_1g^{-1}] - [gA_2g^{-1}, gA_3g^{-1}] &= g(-2A_1 - [A_0, A_1] - [A_2, A_3])g^{-1} + [\dot g g^{-1}, gA_1g^{-1}] \\
    &= g\dot A_1g^{-1} + \dot g A_1 g^{-1} - gA_1g^{-1}\dot g g^{-1} \\
\end{align*}

which is precisely \(\dv{t}(gA_1g^{-1})\). Moreover, in \cref{eq:action}, we can always choose \(g\) to make \(A_0 = 0\), by considering the linear ODE

\[\dot g = gA_0\]

Therefore, we don't change the problem much by considering \cref{eq:gradient-flow-system-extended}. 

\subsection{Complex equations}

Next, we will break the symmetry in the equations, by choosing \(A_1\) to be `special'. More precisely, we will consider \(\alpha, \beta : \R \to \sl(n, \C)\), defined by

\[\alpha = \frac{1}{2}(A_0 + iA_1) \qquad \beta = \frac{1}{2}(A_2 + iA_3)\]

In this case, we have the followiing expressions:

\begin{align*}
    \alpha^* &= \frac{1}{2}(-A_0 + iA_1) \\
    \alpha + \alpha^* &= iA_1 \\
    [\alpha, \alpha^*] &= \frac12i[A_0, A_1] \\
    [\beta, \beta^*] &= \frac12i[A_2, A_3]
\end{align*}

and so the first equation in \cref{eq:gradient-flow-system-extended} can be written as the \emph{real equation}

\begin{equation}
    \label{eq:real-equation}
    \dv{t}(\alpha + \alpha^*) + 2(\alpha + \alpha^*) + 2([\alpha, \alpha^*] + [\beta, \beta^*]) = 0
\end{equation}

and using

\[[\alpha, \beta] = \frac14\left([A_0, A_2] + [A_3, A_1]\right) + \frac14i\left([A_0, A_3] + [A_1, A_2]\right)\]

the second equation in \cref{eq:gradient-flow-system-extended} becomes the \emph{complex equation}

\begin{equation}
    \label{eq:complex-equation}
    \dv{\beta}{t} + 2\beta + 2[\alpha, \beta] = 0
\end{equation}

As above, the real equation is invariant under the action of \(\mcG\). But in this case, the complex equation is invariant under the action of the complex gauge group

\[\mcG^c = \left\{\R \to \SL(n, \C)\right\}\]

via \cref{eq:action}. In particular, the action is given by

\[g \vdot (\alpha, \beta) = \left(g\alpha g^{-1} - \frac{1}{2}\dot g g^{-1}, g\beta g^{-1}\right)\]

and so substituting into \cref{eq:complex-equation}, we get

\begin{align*}
    \dot g \beta g^{-1} + g\dot\beta g^{-1} - g\beta g^{-1}\dot g g^{-1} + 2 g\beta g^{-1} + 2 g[\alpha, \beta]g^{-1} - [\dot g g^{-1}, g\beta g^{-1}] = g\left(\dot\beta + 2\beta + 2[\alpha,\beta]\right)g^{-1}
\end{align*}

\subsection{Complex trajectories}

Let \(\rho_+, \rho_- : \su(2) \to \su(n)\) be Lie algebra homomorphisms. Extend them to Lie algebra homomorphisms \(\sl(2, \C) \to \sl(n, \C)\), and define

\[H_{\pm} = \rho_\pm\begin{pmatrix}
    1 & 0 \\
    0 & -1
\end{pmatrix} \qquad X_\pm = \rho_\pm \begin{pmatrix}
    0 & 1 \\
    0 & 0
\end{pmatrix} \qquad Y_\pm = \rho_\pm \begin{pmatrix}
    0 & 0 \\
    1 & 0
\end{pmatrix}\]

\begin{definition}
    [complex trajectory] A \emph{complex trajectory} associated to \(\rho_+, \rho_-\) is a pair of smooth functions \(\alpha, \beta : \R \to \sl(n, \C)\), which satisfy the complex equation \cref{eq:complex-equation}, and the boundary conditions

    \begin{equation}
        \label{eq:complex-trajectory-boundary-conditions}
        \begin{split}
            \lim_{t \to \infty}2\alpha(t) &= H_+ \\
            \lim_{t \to -\infty}2\alpha(t) &= gH_-g^{-1} \\
            \lim_{t \to \infty}\beta(t) &= Y_+ \\
            \lim_{t \to -\infty}\beta(t) &= gY_-g^{-1}
        \end{split}
    \end{equation}

    for some \(g \in \SU(n)\). Moreover, we require that the convergence in \cref{eq:complex-trajectory-boundary-conditions} is exponential, that is,

    \[\norm{2\alpha(t) - H_+} < Ke^{-\eta t}\]

    for some \(\eta, K > 0\) and so on.
\end{definition}

Now define the subgroup \(\mcG^c_0\) of \(\mcG^c\) by

\[\mcG^c_0 = \left\{g \in \mcG^c \mid g \text{ bounded}, \lim_{t \to \infty}g(t) = 1\right\}\]

Using the operator norm, which satisfies \(\norm{gh} \le \norm{g}\norm{h}\), it is clear that \(\mcG_0^c\) is closed under multiplication. Therefore, all we need to show is that it is closed under inverses. One proof is as follows:

By Cayley-Hamilton, we have coefficients \(c_1(t), \dots, c_{n-1}(t)\) such that

\[g(t)^n + c_{n-1}g(t)^{n-1} + \dots + c_1(t)g(t) + 1 = 0\]

Multiplying by \(g(t)^{-1}\), we get

\[g(t)^{-1} = -\left(g(t)^{n-1} + c_{n-1}g(t)^{n-2} + \dots + c_1(t)\right)\]

The \(c_i(t)\) are the elementary symmetric functions in the eigenvalues of \(g(t)\), and the eigenvalues of \(g(t)\) are bounded, since any eigenvalue \(\lambda\) of \(g(t)\) necessarily satisfies \(\abs{\lambda} \le \norm{g(t)}\). Therefore, the coefficients on the right hand side are bounded. Hence by the triangle inequality, we have a bound on \(\norm{g(t)^{-1}}\).

\begin{definition}
    [equivalent] We say that two complex trajectories \((\alpha, \beta)\) and \((\alpha', \beta')\) are \emph{equivalent} if there exists \(g \in \mcG^c_0\) such that

    \[(\alpha', \beta') = g \vdot (\alpha, \beta)\]

    i.e. they are in the same \(\mcG_0^c\) orbit.
\end{definition}

\subsection{Classification of complex trajectories}

First of all, note that under the \(\mcG^c\) action, we can always make \(\alpha = 0\). In particular, we need

\[\dot g = 2g\alpha\]

Assuming this, the complex equation \cref{eq:complex-equation} becomes

\[\dv{\beta}{t} + 2\beta = 0\]

which has solution

\[\beta(t) = e^{-2t}\beta_0\]

for some \(\beta_0\). Therefore, the only local invariant under the \(\mcG^c\) (and \(\mcG^c_0\)) action is the conjugacy class of \(\beta_0\). Reversing the \(\mcG^c\) action, we find that a generic local solution is

\begin{align*}
    \alpha = \frac{1}{2}g^{-1}\dot g \\
    \beta = e^{-2t}g^{-1}\beta_0g
\end{align*}

As a consequence of this, we have

\begin{lemma}
    If \((\alpha, \beta)\) and \((\alpha', \beta')\) are complex trajectories which are equal outside of some compact set \(K \subseteq \R\), then \((\alpha, \beta)\) and \((\alpha', \beta')\) are equivalent.
\end{lemma}

\begin{proof}
    Without loss of generality, we may assume \(K = [-M, M]\) for some \(M > 0\). Using the \(\mcG^c\) action, we may assume that

    \[\alpha(t) = 0 \qquad \beta(t) = e^{-2t}\beta_0\]

    Now let \(g \in \mcG_0^c\) be such that

    \[g \vdot (\alpha', \beta') = (0, e^{-2t}\beta_0')\]

    In particular, as

    \[\dot g = 2g\alpha'\]

    \(\dot g = 0\) for \(t \notin [-M, M]\), and so \(g\) is constant outside of \([-M, M]\). Say \(g = g_-\) for \(t < -M\) and \(g = g_+\) for \(t > M\). By the boundary condition \(g(t) \to 1\) as \(t \to \infty\), \(g_+ = 1\). This means that for \(t > M\), \(\beta'(t) = e^{-2t}\beta_0'\). But in this case \(\beta = \beta'\), so \(\beta_0 = \beta_0'\). Hence \(g \cdot (\alpha', \beta') = (\alpha, \beta)\), and so they are equivalent.
\end{proof}

\printbibliography

\end{document}
