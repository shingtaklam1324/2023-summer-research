\documentclass{article}

\usepackage{../../Style}
\usepackage{biblatex}
\addbibresource{../../bibliography.bib}

\DeclareMathOperator{\SU}{SU}
\newcommand{\su}{\mathfrak{su}}

\renewcommand{\sl}{\mathfrak{sl}}

\DeclareMathOperator{\gr}{grad}

\title{\citetitle{kronheimer_instantons_1990} by \Citeauthor{kronheimer_instantons_1990}}
\author{Shing Tak Lam}

\begin{document}

\maketitle

In this document, we will discuss the paper \cite{kronheimer_instantons_1990}. For concreteness, instead of general Lie groups and Lie algebras, we will focus on the case

\[G = \SU(n) \qquad \mfg = \su(n)\]

which has complexification

\[G^c = \SL(n, \C) \qquad \mfg^c = \sl(n, \C)\]

\section{Introduction}

The inner product on \(\su(n)\) is given by \(-\kappa\), where \(\kappa\) is the Killing form. That is,

\[\inner{A, B} = -\tr(AB)\]

Define

\begin{align*}
    \varphi : \su(n) \times \su(n) \times \su(n) &\to \R \\
    \varphi(A_1, A_2, A_3) &= \sum_{j=1}^3 \inner{A_j, A_j} + \inner{A_1, [A_2, A_3]}
\end{align*}

We are interested in studying the gradient flow of \(\varphi\). That is, \(A_1, A_2, A_3 : I \to \su(n)\) such that

\begin{equation}
    \label{eq:gradient-flow}
    (\dot A_1, \dot A_2, \dot A_3) = -\grad \varphi(A_1, A_2, A_3)
\end{equation}

First of all, notice that

\[\varphi(A_1 + H_1, A_2, A_3) = \varphi(A_1, A_2, A_3) + 2\inner{H_1, A_1} + \inner{H_1, [A_2, A_3]}\]

and that \(\inner{A_1, [A_2, A_3]} = \inner{A_2, [A_3, A_1]} = \inner{A_3, [A_1, A_2]}\). Therefore, \cref{eq:gradient-flow} becomes

\begin{equation}
    \label{eq:gradient-flow-system}
    \begin{split}
        \dot A_1 &= -2A_1 - [A_2, A_3] \\
        \dot A_2 &= -2A_2 - [A_3, A_1] \\
        \dot A_3 &= -2A_3 - [A_1, A_2]
    \end{split}
\end{equation}

The critical points of \cref{eq:gradient-flow-system} are triples \((A_1, A_2, A_3)\) satisfying

\[[A_1, A_2] = -2A_3 \quad [A_2, A_3] = -A_1 \quad [A_3, A_1] = -2A_2\]

Recall that the Lie algebra \(\su(2)\) has basis

\[e_1 = \begin{pmatrix}
    -i & 0 \\
    0 & i
\end{pmatrix} \quad e_2 = \begin{pmatrix}
    0 & 1 \\
    -1 & 0
\end{pmatrix} \quad e_3 = \begin{pmatrix}
    0 & -i \\
    -i & 0
\end{pmatrix}\]

satifying the above relations. Therefore, critical points of \cref{eq:gradient-flow-system} correspond to Lie algebra homomorphisms \(\rho : \su(2) \to \su(n)\). From this, we see that at all critical points of \cref{eq:gradient-flow-system}, \(\varphi\) is nonnegative, and it is zero only at \((0, 0, 0)\).

Next, we will identify \(\su(n) \times \su(n) \times \su(n) \cong \rm L(\su(2), \su(n))\), the space of linear maps \(\su(2) \to \su(n)\), sending \((A_1, A_2, A_3)\) to the linear map \(A\) given by \(e_i \mapsto A_i\).

The adjoint action of \(\SU(n)\) on \(\su(n)\) is given by

\[\Ad_g(A) = gAg^{-1}\]

and this induces an action on \(\rm L(\su(2), \su(n))\) by

\[g \cdot A : e_i \mapsto gA_ig^{-1}\]

For any Lie algebra homomorphism \(\rho : \su(2) \to \su(n)\), define

\[C(\rho) = \left\{g \cdot \rho \mid g \in \SU(n)\right\}\]

for the critical manifold of all homomorphisms which are conjugate to \(\rho\) via the adjoint action. For Lie algebra homomorphisms \(\rho_-, \rho_+ : \su(2) \to \su(n)\), define \(M(\rho_-, \rho_+)\) for the space of solutions \(A(t)\) to \cref{eq:gradient-flow-system}, with boundary conditions

\begin{equation}
    \label{eq:boundary-conditions}
    \begin{split}
        \lim_{t\to-\infty}A(t) &\in C(\rho_-) \\
        \lim_{t\to\infty}A(t) &= \rho_+
    \end{split}
\end{equation}

Note that we are considering parametrised trajectories, therefore there is a natural \(\R\)-action sending \(A(t)\) to \(A(t+c)\).

For a Lie algebra homomorphism \(\rho : \su(2) \to \su(n)\), we can extend it to a Lie algebra homomorphism \(\rho : \sl(2, \C) \to \sl(n, \C)\), and define

\[H = \rho\begin{pmatrix}
    1 & 0 \\
    0 & -1
\end{pmatrix}\quad X = \rho\begin{pmatrix}
    0 & 1 \\
    0 & 0
\end{pmatrix} \quad Y = \rho\begin{pmatrix}
    0 & 0 \\
    1 & 0
\end{pmatrix}\]

We will then define \(\mcN(\rho)\) for the nilpotent orbit of \(Y\) in \(\sl(n, \C)\), and the affine subspace

\[S(\rho) = Y + Z(X)\]

where \(Z(X) = \left\{A \in \sl(n, \C) \mid [A, X] = 0\right\}\). Using this, we have

\begin{theorem}
    For any pair of homomorphisms \(\rho_-, \rho_+\), there is a diffeomorphism

    \[M(\rho_-, \rho_+) \cong \mcN(\rho_-) \cap S(\rho_+)\]
\end{theorem}

If \(\rho_+ = 0\), then \(S(\rho_+) = \sl(n, \C)\), and in this case, we have a diffeomorphism

\[M(\rho_-, 0) \cong \mcN(\rho_-)\]

Moreover, every nilpotent orbit is \(\mcN(\rho)\) for some homomorphism \(\rho : \su(2) \to \su(n)\), which means that we have a description of all nilpotent orbits in \(\sl(n, \C)\).

\section{Complex trajectories}

\subsection{Gauge group}

First of all, we will extend \cref{eq:gradient-flow-system} by considering \(A_0, \dots, A_3 : \R \to \su(n)\), satisfying the equations

\begin{equation}
    \label{eq:gradient-flow-system-extended}
    \begin{split}
        \dot A_1 &= -2A_1 - [A_0, A_1] - [A_2, A_3] \\
        \dot A_2 &= -2A_2 - [A_0, A_2] + [A_1, A_3] \\
        \dot A_3 &= -2A_3 - [A_0, A_3] - [A_1, A_2] \\
    \end{split}
\end{equation}

Define the group

\[\mcG = \left\{g : \R \to \SU(n)\right\}\]

with pointwise operations. Then \(\mcG\) acts \(A = (A_0, \dots, A_3)\) by

\begin{equation}
    \label{eq:action}
    (g \vdot A)(t) = \left(g(t)A_0(t)g(t)^{-1} - \dv{g}{t}(t) \cdot g(t)^{-1}, g(t)A_1(t)g(t)^{-1}, g(t)A_2(t)g(t)^{-1}, g(t)A_3(t)g(t)^{-1}\right)
\end{equation}

For brevity, when clear, we will write this as

\[g \vdot A = (gA_0g^{-1} - \dot g g^{-1}, gA_1g^{-1}, gA_2g^{-2}, gA_3g^{-1})\]

Note that \(\dot g(t) \in \TT_{g(t)}\SU(n) = g(t)\su(n)\), and so \(\dot g(t)g(t)^{-1} \in g(t)\su(n)g(t)^{-1} = \su(n)\). First, we will show that \cref{eq:gradient-flow-system-extended} is invariant under the action \cref{eq:action}. To see this, the transformed right hand side (for the first equation) is

\begin{align*}
    -2gA_1g^{-1} - [gA_0g^{-1} - \dot g g^{-1}, gA_1g^{-1}] - [gA_2g^{-1}, gA_3g^{-1}] &= g(-2A_1 - [A_0, A_1] - [A_2, A_3])g^{-1} + [\dot g g^{-1}, gA_1g^{-1}] \\
    &= g\dot A_1g^{-1} + \dot g A_1 g^{-1} - gA_1g^{-1}\dot g g^{-1} \\
\end{align*}

which is precisely \(\dv{t}(gA_1g^{-1})\). Moreover, in \cref{eq:action}, we can always choose \(g\) to make \(A_0 = 0\), by considering the linear ODE

\[\dot g = gA_0\]

Therefore, we don't change the problem much by considering \cref{eq:gradient-flow-system-extended}. 

\subsection{Complex equations}

Next, we will break the symmetry in the equations, by choosing \(A_1\) to be `special'. More precisely, we will consider \(\alpha, \beta : \R \to \sl(n, \C)\), defined by

\[\alpha = \frac{1}{2}(A_0 + iA_1) \qquad \beta = \frac{1}{2}(A_2 + iA_3)\]

In this case, we have the followiing expressions:

\begin{align*}
    \alpha^* &= \frac{1}{2}(-A_0 + iA_1) \\
    \alpha + \alpha^* &= iA_1 \\
    [\alpha, \alpha^*] &= \frac12i[A_0, A_1] \\
    [\beta, \beta^*] &= \frac12i[A_2, A_3]
\end{align*}

and so the first equation in \cref{eq:gradient-flow-system-extended} can be written as the \emph{real equation}

\begin{equation}
    \label{eq:real-equation}
    \dv{t}(\alpha + \alpha^*) + 2(\alpha + \alpha^*) + 2([\alpha, \alpha^*] + [\beta, \beta^*]) = 0
\end{equation}

and using

\[[\alpha, \beta] = \frac14\left([A_0, A_2] + [A_3, A_1]\right) + \frac14i\left([A_0, A_3] + [A_1, A_2]\right)\]

the second equation in \cref{eq:gradient-flow-system-extended} becomes the \emph{complex equation}

\begin{equation}
    \label{eq:complex-equation}
    \dv{\beta}{t} + 2\beta + 2[\alpha, \beta] = 0
\end{equation}

As above, the real equation is invariant under the action of \(\mcG\). But in this case, the complex equation is invariant under the action of the complex gauge group

\[\mcG^c = \left\{\R \to \SL(n, \C)\right\}\]

via \cref{eq:action}. In particular, the action is given by

\[g \vdot (\alpha, \beta) = \left(g\alpha g^{-1} - \frac{1}{2}\dot g g^{-1}, g\beta g^{-1}\right)\]

and so substituting into \cref{eq:complex-equation}, we get

\begin{align*}
    \dot g \beta g^{-1} + g\dot\beta g^{-1} - g\beta g^{-1}\dot g g^{-1} + 2 g\beta g^{-1} + 2 g[\alpha, \beta]g^{-1} - [\dot g g^{-1}, g\beta g^{-1}] = g\left(\dot\beta + 2\beta + 2[\alpha,\beta]\right)g^{-1}
\end{align*}

\subsection{Complex trajectories}

Let \(\rho_+, \rho_- : \su(2) \to \su(n)\) be Lie algebra homomorphisms. Extend them to Lie algebra homomorphisms \(\sl(2, \C) \to \sl(n, \C)\), and define

\[H_{\pm} = \rho_\pm\begin{pmatrix}
    1 & 0 \\
    0 & -1
\end{pmatrix} \qquad X_\pm = \rho_\pm \begin{pmatrix}
    0 & 1 \\
    0 & 0
\end{pmatrix} \qquad Y_\pm = \rho_\pm \begin{pmatrix}
    0 & 0 \\
    1 & 0
\end{pmatrix}\]

\begin{definition}
    [complex trajectory] A \emph{complex trajectory} associated to \(\rho_+, \rho_-\) is a pair of smooth functions \(\alpha, \beta : \R \to \sl(n, \C)\), which satisfy the complex equation \cref{eq:complex-equation}, and the boundary conditions

    \begin{equation}
        \label{eq:complex-trajectory-boundary-conditions}
        \begin{split}
            \lim_{t \to \infty}2\alpha(t) &= H_+ \\
            \lim_{t \to -\infty}2\alpha(t) &= gH_-g^{-1} \\
            \lim_{t \to \infty}\beta(t) &= Y_+ \\
            \lim_{t \to -\infty}\beta(t) &= gY_-g^{-1}
        \end{split}
    \end{equation}

    for some \(g \in \SU(n)\). Moreover, we require that the convergence in \cref{eq:complex-trajectory-boundary-conditions} is exponential, that is,

    \[\norm{2\alpha(t) - H_+} < Ke^{-\eta t}\]

    for some \(\eta, K > 0\) and so on. Note the choice of norm here does not matter, as all norms on \(\sl(n, \C)\) are equivalent.
\end{definition}

Now define the subgroup \(\mcG^c_0\) of \(\mcG^c\) by

\[\mcG^c_0 = \left\{g \in \mcG^c \mid g \text{ bounded}, \lim_{t \to \infty}g(t) = 1\right\}\]

Using the operator norm, which satisfies \(\norm{gh} \le \norm{g}\norm{h}\), it is clear that \(\mcG_0^c\) is closed under multiplication. Therefore, all we need to show is that it is closed under inverses. One proof is as follows:

By Cayley-Hamilton, we have coefficients \(c_1(t), \dots, c_{n-1}(t)\) such that

\[g(t)^n + c_{n-1}g(t)^{n-1} + \dots + c_1(t)g(t) + 1 = 0\]

Multiplying by \(g(t)^{-1}\), we get

\[g(t)^{-1} = -\left(g(t)^{n-1} + c_{n-1}g(t)^{n-2} + \dots + c_1(t)\right)\]

The \(c_i(t)\) are the elementary symmetric functions in the eigenvalues of \(g(t)\), and the eigenvalues of \(g(t)\) are bounded, since any eigenvalue \(\lambda\) of \(g(t)\) necessarily satisfies \(\abs{\lambda} \le \norm{g(t)}\). Therefore, the coefficients on the right hand side are bounded. Hence by the triangle inequality, we have a bound on \(\norm{g(t)^{-1}}\).

\begin{definition}
    [equivalent] We say that two complex trajectories \((\alpha, \beta)\) and \((\alpha', \beta')\) are \emph{equivalent} if there exists \(g \in \mcG^c_0\) such that

    \[(\alpha', \beta') = g \vdot (\alpha, \beta)\]

    i.e. they are in the same \(\mcG_0^c\) orbit.
\end{definition}

\subsection{Classification of complex trajectories}

First of all, note that under the \(\mcG^c\) action, we can always make \(\alpha = 0\). In particular, we need

\[\dot g = 2g\alpha\]

Assuming this, the complex equation \cref{eq:complex-equation} becomes

\[\dv{\beta}{t} + 2\beta = 0\]

which has solution

\[\beta(t) = e^{-2t}\beta_0\]

for some \(\beta_0\). Therefore, the only local invariant under the \(\mcG^c\) (and \(\mcG^c_0\)) action is the conjugacy class of \(\beta_0\). Reversing the \(\mcG^c\) action, we find that a generic local solution is

\begin{align*}
    \alpha = \frac{1}{2}g^{-1}\dot g \\
    \beta = e^{-2t}g^{-1}\beta_0g
\end{align*}

As a consequence of this, we have

\begin{lemma}
    \label{lem:complex-trajectory-equal}

    If \((\alpha, \beta)\) and \((\alpha', \beta')\) are complex trajectories which are equal outside of some compact set \(K \subseteq \R\), then \((\alpha, \beta)\) and \((\alpha', \beta')\) are equivalent.
\end{lemma}

\begin{proof}
    Without loss of generality, we may assume \(K = [-M, M]\) for some \(M > 0\). Using the \(\mcG^c\) action, we may assume that

    \[\alpha(t) = 0 \qquad \beta(t) = e^{-2t}\beta_0\]

    Now let \(g \in \mcG^c\) be such that

    \[g \vdot (\alpha', \beta') = (0, e^{-2t}\beta_0')\]

    In particular, as

    \[\dot g = 2g\alpha'\]

    \(\dot g = 0\) for \(t \notin [-M, M]\), and so \(g\) is constant outside of \([-M, M]\). Say \(g = g_-\) for \(t < -M\) and \(g = g_+\) for \(t > M\). By left multiplication by \(g_+^{-1}\), we can assume \(g_+ = 1\). This means that for \(t > M\), \(e^{-2t}\beta'(t) = e^{-2t}\beta_0'\). But in this case \(\beta = \beta'\), so \(\beta_0 = \beta_0'\). Hence \(g \cdot (\alpha', \beta') = (\alpha, \beta)\), and so they are equivalent.
\end{proof}

\begin{lemma}
    \label{lem:complex-trajectory-convergence-negative}

    Let \((\alpha, \beta)\) be a solution of the complex equation \cref{eq:complex-equation}, satisfying the boundary equations \cref{eq:complex-trajectory-boundary-conditions} at \(t \to -\infty\). That is,

    \[\lim_{t\to-\infty}2\alpha(t) = gH_-g^{-1} \qquad \lim_{t \to -\infty}\beta(t) = gY_-y^{-1}\]

    with exponential convergence. Then there exists a gauge transformation \(g_- : \R \to \SL(n, \C)\) such that \((\alpha', \beta') = g_-\vdot(\alpha, \beta)\) is the constant solution

    \[2\alpha' = H_- \qquad \beta' = Y_-\]

    and \(g_-(t)\) converges as \(t \to -\infty\).
\end{lemma}

\begin{proof}
    By conjugation, without loss of generality \(g = 1\). Considering the ODE

    \begin{equation*}
            \dot g_0 = 2g_0\alpha - H_-g_0
    \end{equation*}

    We can find \(g_0\) such that

    \[H_- = 2g_0\alpha g_0^{-1} - \dot g_0 g_0^{-1}\]

    with the boundary condition \(g_0(t) \to 1\) as \(t \to -\infty\), since \(2\alpha(t) \to H_-\) exponentially. \textbf{mhm... why?}

    Using this, we get a transformed solution \((\alpha'', \beta'') = g_0\vdot(\alpha,\beta)\), with \(2\alpha'' = H_-\). In this case, the complex equation becomes

    \[\dv{\beta''}{t} + 2\beta'' + [H_-, \beta''] = 0\]

    Trying the ansatz

    \begin{align*}
        \beta''(t) &= e^{-2t}\Ad_{f(t)}(\omega) = e^{-2t}f\omega f^{-1} \\
        f(t) &= \exp(Xt)
    \end{align*}

    We have that

    \[\dot f = Xf\]

    and so

    \begin{align*}
        \dot\beta'' &= -2e^{-2t}f\omega f^{-1} + e^{-2t}\dot f \omega f^{-1} - e^{-2t}f\omega f^{-1}\dot f f^{-1} \\
        &= -2\beta'' + X\beta'' - \beta'' X
    \end{align*}

    Therefore, the complex equation becomes

    \begin{align*}
        \dot\beta'' + 2\beta'' + H_-\beta'' - \beta''H_- = -2\beta'' + X\beta'' - \beta''X + 2\beta'' + H_-\beta'' - \beta''H_- = [X + H_-, \beta'']
    \end{align*}

    Hence setting \(X = -H_-\), we get a solution. By dimensionality arguments, this is the general solution.

    Using the composition

    % https://q.uiver.app/#q=WzAsMyxbMCwwLCJcXHNsKDIsIFxcQykiXSxbMiwwLCJcXHNsKG4sIFxcQykiXSxbNCwwLCJcXGdsKFxcc2wobiwgXFxDKSkiXSxbMCwxLCJcXHJob18tIl0sWzEsMiwiXFxhZCJdXQ==
\[\begin{tikzcd}[ampersand replacement=\&]
	{\sl(2, \C)} \&\& {\sl(n, \C)} \&\& {\gl(\sl(n, \C))}
	\arrow["{\rho_-}", from=1-1, to=1-3]
	\arrow["\ad", from=1-3, to=1-5]
\end{tikzcd}\]

    We get a representation of \(\sl(2, \C)\) on \(\sl(n, \C)\). Therefore, we have a decomposition

    \[\sl(n, \C) = \bigoplus_{k \in \Z}V_k\]

    where \(V_\lambda\) is the \(\lambda\)-eigenspace of \(\ad(H_-)\). Since we want \(\beta'' \to Y_-\) as \(t \to -\infty\), we will try the ansatz \(\omega = Y_- + \delta\). By linearity, we can first compute the case of \(\omega = Y_-\).

    First of all, notice that we also have that \(\dot f = f X = -fH_-\), and so in this case

    \begin{align*}
        \dot\beta'' &= -2e^{-2t}f Y_- f^{-1} - e^{-2t}f H_- Y_- f^{-1} + fY_-f^{-1}fH_-f^{-1} \\
        &= -2\beta'' - f[H_-, Y_-]f^{-1} \\
        &= 0
    \end{align*}

    as \([H_-, Y_-] = \rho_-([H, Y]) = \rho_-(-2Y) = -2Y_-\). Therefore, as \(\beta''(0) = Y_-\) in this case, it is constant. Now by linearity, say \(\delta = \sum_k \delta_k\), where \(\delta_k \in V_k\). Then for \(\omega = \delta_k\),

    \[\dot\beta'' = -2\beta'' - f[H_-, \delta_k]f^{-1} = -(2+k)\beta''\]

    This gives the solution

    \[\beta''(t) = e^{-(2+k)t}\beta''(0) = e^{-(2+k)t}\delta_k\]

    Since we require \(\beta''(t) \to 0\) as \(t \to -\infty\) in this case, we need \(-(2+k) > 0\), i.e. \(k < -2\). Hence the general solution in this case is

    \[\beta''(t) = Y_- + e^{-2t}\exp(-H_-t)\delta\exp(H_-t)\]

    where \(\delta \in \bigoplus\limits_{k < -2}V_k\). Now notice that \(g_0\) from earlier was not uniquely determined. We can still act on the solution by a gauge transformation \(g_1\), which preserves \(2\alpha'' = H_-\), and approaches \(1\) at \(t \to -\infty\). That is, we have the equation

    \[H_- = g_1H_-g_1^{-1} - \dot g_1 g_1^{-1}\]

    which we can rearrange to

    \[\dot g_1 = g_1H_- - H_- g_1\]

    Trying the ansatz

    \begin{align*}
        g_1(t) &= f(t)\sigma f(t)^{-1} \\
        f(t) &= \exp(-H_- t)
    \end{align*}

    for \(\sigma \in \SL(n, \C)\), we find that this gives the general solution for the equation. For the boundary condition, suppose further that \(\sigma = \exp(\gamma)\), for some \(\gamma \in \sl(n, \C)\). Define

    \[h_t(s) = f\exp(s\gamma)f^{-1}\]

    and note that \(g_1(t) = h_t(1)\). Then

    \begin{align*}
        \dv{h_t}{s}(s) &= f \exp(s\gamma)\gamma f^{-1} \\
        &= h_t(s) \cdot f \gamma f^{-1}
    \end{align*}

    Set \(\varphi(t) = f\gamma f^{-1}\), then we have that

    \[\dot\varphi = -f[H, \gamma]f^{-1}\]

    This equation is linear in \(\gamma\), and so for simplicity, we will assume \(\gamma \in V_k\). In this case, \(\dot\varphi = -k\varphi\), and so \(\varphi(t) = e^{-kt}\gamma\). Substituting this in, we get that

    \[\dv{h_t}{s} = e^{-kt}h_t \cdot \gamma\]

    and so, integrating this equation, we find that

    \[h_t(s) = \exp(se^{-kt}\gamma) \implies g_1(t) = \exp(e^{-kt}\gamma)\]

    Thus, for \(g_1 \to 1\) as \(t \to -\infty\), we must have \(k < 0\). Therefore, the general solution is

    \[g_1(t) = \exp(-H_-t)\exp(\gamma)\exp(H_-t)\]

    where \(\gamma \in \bigoplus\limits_{k < 0}V_k\). Therefore, if we consider \((\alpha', \beta') = g_1 \cdot (\alpha'', \beta'')\), we would get that \(2\alpha' = H_-\), and

    \[\beta'(t) = Y_- + e^{-2t}\exp(-H_-t)(\exp(\gamma)(Y_- + \delta)\exp(-\gamma) - Y_-)\exp(H_-t)\]

    Therefore, all that remains to show is that for all \(\delta \in \bigoplus\limits_{k < -2}V_k\), there exists \(\gamma \in \bigoplus\limits_{k < 0}V_k\) such that

    \[\exp(\gamma)(Y_- + \delta)\exp(-\gamma) - Y_- = 0\]

    We will use the implicit function theorem for this. Expand the left hand side near \(\gamma = \delta = 0\), the terms linear in \(\gamma, \delta\) are

    \[f(\gamma, \delta) = \delta + \gamma Y_- - Y_-\gamma = \delta - [Y_-, \gamma]\]

    From the representation theory of \(\sl(2, \C)\), we have a linear map

    \[[Y_-, \cdot] : \bigoplus_{k < 0}V_k \to \bigoplus_{k < -2}V_k\]
    
    and so we have a map 

    \[f : \bigoplus_{k < 0}V_k \oplus \bigoplus_{k < -2}V_k \to \bigoplus_{k < -2}V_k\]
    
    The map \(\gamma \mapsto f(\gamma, 0)\) is surjective, for example by decomposing \(\sl(n, \C)\) as a direct sum of \(\sl(2, \C)\) representations. Therefore if we have a decomposition

    \[\bigoplus_{k < 0}V_k = K \oplus W\]

    where \(K = \ker(f(\cdot, 0))\), then the map \(\hat f : W \to \bigoplus\limits_{k < -2}V_k\), given by \(\hat f(\gamma) = f(\gamma, 0)\), is an isomorphism. We can then apply the implicit function theorem to

    \begin{align*}
        F : \left(\bigoplus_{k < -2}V_k \oplus K\right) \oplus W &\to \bigoplus_{k < -2}V_k \\
        F((\delta, k), \gamma') &= \exp((\gamma', k))(Y_- + \delta)\exp(-(\gamma', k)) - Y_-
    \end{align*}

    which then gives us a neighbourhood \(U\) of \(0\) in \(\bigoplus\limits_{k < -2}V_k\), and a neighbourhood \(V\) of \(0\) in \(W\), and a map \(g : U \times V \to W\) such that

    \[F(x, g(x)) = 0\]

    for all \(x \in U \times V\). Therefore, for \(\delta \in U\), setting \(\gamma = g(\delta, 0)\) gives the required result. Finally, we will use homogeneity to extend the result to all of \(\bigoplus \limits_{k < -2}V_k\). First of all, we note that the condition is invariant under the substitution

    \begin{align*}
        \gamma &= f\hat\gamma f^{-1} \\
        \delta &= e^{-2t}f\hat\delta f^{-1}
    \end{align*}

    where \(f(t) = \exp(-H_-t)\), since we have that \(Y_- = e^{-2t}fY_-f^{-1}\), and that \(\exp(f\hat\gamma f^{-1}) = f\exp(\hat\gamma)f^{-1}\). Now suppose \([H, v] = mv\), and let \(\varphi = fvf^{-1}\). Then

    \begin{align*}
        \dot\varphi &= f\dot vf^{-1} - fvf^{-1}\dot f f^{-1} \\
        &= -fHvf^{-1} + fvHf^{-1} \\
        &= -mfvf^{-1} \\
        &= -m\varphi
    \end{align*}

    Hence \(\varphi(t) = e^{-mt}v\). Therefore in the limit \(t \to -\infty\) (as \(m < 0\)), we have that \(\gamma \to 0\), and so we can apply the result for small \(\delta\).
\end{proof}

There is a very similar result for the limit at \(t \to \infty\).

\begin{lemma}
    \label{lem:complex-trajectory-convergence-positive}

    Let \((\alpha, \beta)\) be a solution of the complex equation \cref{eq:complex-equation} satisfying the boundary equations \cref{eq:boundary-conditions} at \(t \to \infty\). That is,

    \[\lim_{t \to \infty}2\alpha(t) = H_+ \qquad \lim_{t \to \infty}\beta(t) = Y_+\]

    with exponential convergence. Then there exists a unique gauge transformation \(g_+ : \R \to \SL(n, \C)\), with \(g_+(t) \to 1\) as \(t \to \infty\), such that the transformed solution \((\alpha', \beta') = g_+ \vdot (\alpha, \beta)\) satisfies

    \[2\alpha' = H_+ \qquad \beta'(0) \in S(\rho_+)\]
\end{lemma}

\begin{proof}
    The proof is very similar to the previous lemma. We find a gauge transformation \(g_0\), approaching \(1\) as \(t \to \infty\), such that \((\alpha'', \beta'') = g\vdot (\alpha, \beta)\) satisfies

    \begin{align*}
        2\alpha'' &= H_+ \\
        \beta''(t) &= Y_+ + e^{-2t}\exp(-H_+t)\epsilon\exp(H_+t)
    \end{align*}

    with

    \[\epsilon \in \bigoplus_{k > -2}V_k\]

    where in this case, \(V_k\) is the \(k\)-eigenspace of \(\ad(H_+)\). As above, we have a further choice of gauge transformation \(g_1\) of the form

    \[g_1(t) = \exp(-H_+t)\exp(\gamma)\exp(H_+t)\]

    where \(\gamma \in \bigoplus_{i > 0}V_k\). Using this, the solution becomes

    \[\beta''(t) = Y_+ + e^{-2t}\exp(-H_+t)(\exp(\gamma)(Y_+ + \epsilon)\exp(-\gamma) - Y_+)\exp(H_+t)\]

    Recall that \(S(\rho_+) = Y_+ + Z(X_+)\). Therefore, we need to show that for each \(\epsilon \in \bigoplus\limits_{k > -2}V_k\), there exists \(\gamma \in \bigoplus\limits_{k > 0}V_k\) such that

    \[\exp(\gamma)(Y_+ + \epsilon)\exp(-\gamma) - Y_+ \in Z(X_+)\]

    Expanding the left hand side near \(\gamma = \epsilon = 0\), to first order we have

    \[f(\gamma, \epsilon) = \epsilon - [Y_+, \gamma]\]

    In this case, we have a linear map

    \[[Y_+, \cdot] : \bigoplus_{k > 0}V_k \to \bigoplus_{k > -2}V_k\]

    which is injective, and its image satisfies

    \[\bigoplus_{k > -2} V_k = \Im([Y_+, \cdot]) \oplus Z(X_+)\]

    Therefore, for each \(\epsilon\), there exists a unique \(\gamma\) such that \(f(\gamma, \epsilon) \in Z(X_+)\). Hence the linearisation has a unique solution, and so by the implicit function theorem, for \(\epsilon\) sufficiently small, there exists \(\gamma\) such that \(\exp(\gamma)(Y_+ + \epsilon)\exp(-\gamma) - Y_+ \in Z(X_+)\). Finally, we can use homogeneity to extend the result to all of \(\bigoplus\limits_{k > -2}V_k\) as above.
\end{proof}

Now let \((\alpha', \beta')\) be a solution of the complex equation \cref{eq:complex-equation} satisfying the boundary conditions \cref{eq:boundary-conditions}. Define a gauge transformation \(g : \R \to \SL(n, \C)\) via

\begin{equation}
    g(t) = \begin{cases}
        g_-(t) & t \le 0 \\
        g_+(t) & t \ge 1
    \end{cases}
\end{equation}

and smooth on \([0, 1]\). Then \(g(t)\) is bounded, since \(g_-\) and \(g_+\) are, as they converge in the limit \(t \to \pm\infty\). Therefore, \(g \in \mcG_0^c\), and \((\alpha, \beta) = g \vdot (\alpha', \beta')\) is given by

\begin{equation}
    \label{eq:complex-trajectory-form}
    \begin{split}
        \alpha(t) &= \begin{cases}
            \frac{1}{2}H_- & t \le 0 \\
            \frac{1}{2}H_+ & t \ge 1
        \end{cases} \\
        \beta(t) &= \begin{cases}
            Y_- & t \le 0 \\
            Y_+ + e^{-2t}\exp(-H_+t)\epsilon\exp(H_+t) & t \ge 1
        \end{cases}
    \end{split}
\end{equation}

and hence every complex trajectory is equivalent to one of this form. Moreover, we can choose \(\epsilon\) such that \(Y_+ + \epsilon \in S(\rho_+)\), and in this case, \(\epsilon\) is uniquely determined.

Since \((\alpha, \beta)\) is locally equivalent to the constant solution \((-\frac{1}{2}H_-, Y_-)\), the element \(Y_+ + \epsilon\) must be conjugate to \(Y_-\) in \(\sl(n, \C)\). That is, \(Y_+ + \epsilon \in \mcN(\rho_-)\). Conversely, given \(Y_+ + \epsilon \in S(\rho_+) \cap \mcN(\rho_-)\), we can always find a solution satisfying \cref{eq:complex-trajectory-form}.

\begin{proposition}
    \label{prop:complex-trajectory-classification}

    The equivalence classes of complex trajectories associated to \(\rho_+, \rho_-\) are parametrised by \(S(\rho_+) \cap \mcN(\rho_-)\).
\end{proposition}

\begin{proof}
    We have already seen that each trajectory is equivalent to one in the form \cref{eq:complex-trajectory-form}, which is parametrised by the element \(Y_+ + \epsilon \in S(\rho_+) \cap \mcN(\rho_-)\). Using \cref{lem:complex-trajectory-equal}, we see that two trajectories which are equal outside of \([0, 1]\) are equivalent. Therefore, the equivalence classes are parametrised by \(Y_+ + \epsilon \in S(\rho_+) \cap \mcN(\rho_-)\).
\end{proof}

\section{Nahm's equations}

\label{sec:nahm}

Consider the change of variables

\[T_i = e^{2t}A_i \qquad s = -\frac12e^{-2t}\]

Using this, \cref{eq:gradient-flow-system} becomes

\begin{align*}
    \dv{T_1}{s} &= -[T_2, T_3] \\
    \dv{T_2}{s} &= -[T_3, T_1] \\
    \dv{T_3}{s} &= -[T_1, T_2]
\end{align*}

which are Nahm's equations. The same change of variables also transforms \cref{eq:gradient-flow-system-extended} into 

\begin{align*}
    \dv{T_1}{s} + [T_0, T_1] + [T_2, T_3] &= 0 \\
    \dv{T_2}{s} + [T_0, T_2] + [T_3, T_1] &= 0 \\
    \dv{T_3}{s} + [T_0, T_3] + [T_1, T_2] &= 0 \\
\end{align*}

Using this, we can also consider the action of the gauge group on this system. Recall that the action is given by \cref{eq:action}, which is:

\[g \vdot A = (gA_0g^{-1} - \dot g g^{-1}, gA_1g^{-1}, gA_2g^{-1}, gA_3g^{-1})\]

Note that

\[\dot g = \dv{g}{t} = \dv{g}{s}\dv{s}{t} = e^{-2t}\dv{g}{s}\]

In this case, the gauge group action becomes

\begin{align*}
    g \vdot T &= g \vdot (e^{-2t}T_0, e^{-2t}T_1, e^{-2t}T_2, e^{-2t}T_3) \\
    &= \left(e^{-2t}gT_0g^{-1} - e^{-2t}\dv{g}{s} g^{-1}, e^{-2t}gT_1g^{-1}, e^{-2t}gT_2g^{-1}, e^{-2t}gT_3g^{-1}\right) \\
    &= \left(gT_0g^{-1} - \dv{g}{s}g^{-1}, gT_1g^{-1}, gT_2g^{-1}, gT_3g^{-1}\right)
\end{align*}

This is the same as the action as in \cite[Equation 1.6]{donaldson_nahms_1984}. Finally, we can consider the \(\SL(n, \C)\) valued paths

\[\tilde\alpha = e^{2t}\alpha = \frac{1}{2}(T_0 + iT_1) \qquad \tilde\beta = e^{2t}\beta = \frac12(T_2 + iT_3)\]

In this case the real and complex equations become

\begin{align}
    \label{eq:nahm-real-equation}
    \dv{s}(\tilde\alpha + \tilde\alpha^*) + 2([\tilde\alpha, \tilde\alpha^*] + [\tilde\beta, \tilde\beta^*]) &= 0 \\
    \dv{\tilde\beta}{s} + 2[\tilde\alpha, \tilde\beta] &= 0 \nonumber
\end{align}

With all of this in mind, this allows us to use the results from \cite{donaldson_nahms_1984}.

\section{Real equation}

Recall the real equation \cref{eq:real-equation},

\[\hat F(\alpha, \beta) = \dv{t}(\alpha + \alpha^*) + 2(\alpha + \alpha^*) + 2([\alpha, \alpha^*] + [\beta, \beta^*]) = 0\]

Write \((\alpha', \beta') = g \vdot (\alpha, \beta)\), and we will regard \(\hat F(\alpha', \beta') = 0\) as an equation for \(g\). First of all, notice that the real equation is invariant under the action of \(\mcG\), and so the action of \(g\) only depends on the corresponding path

\[\tilde g : \R \to \SL(n, \C) / \SU(n) = \mcH\]

From the polar decomposition of \(\SL(n, \C)\), we can write any \(A \in \SL(n, \C)\) uniquely as \(A = UP\), where \(U \in \SU(n)\) and \(P\) is hermitian, with positive eigenvalues and \(\det(P) = 1\). Hence we can choose

\[\mcH = \left\{A \in \SL(n, \C) \mid A \text{ hermitian, with positive eigenvalues}\right\}\]

For each \(g\), we define \(h = h(g) = g^*g\), which gives us a path \(h : \R \to \mcH\).

\subsection{Uniqueness}

\begin{lemma}
    Suppose \((\alpha, \beta)\) satisfies the complex equation on an interval \([-N, N]\). Then for any \(h_-, h_+ \in \mcH\), there exists \(g : [-N, N] \to \SL(n, \C)\) continuous and smooth on the interior, with \(h = h(g)\) satisfying

    \[h(-N) = h_- \qquad h(N) = h_+\]

    and such that \((\alpha', \beta') = g \vdot (\alpha, \beta)\) satisfies the real equation \(\hat F(\alpha', \beta') = 0\) on \([-N, N]\).
\end{lemma}

\begin{proof}
    See \cite[Proposition 2.8]{donaldson_nahms_1984}. The main idea is that the real equation (for Nahm's equations) is the Euler-Lagrange equations for a functional, and so the result follows by the direct method of the calculus of variations. To get the result, we apply \cite[Proposition 2.8]{donaldson_nahms_1984} with

    \[\text{`}\alpha\text{'} := \tilde \alpha = e^{2t}\alpha \qquad \text{`}\beta\text{'} = \tilde\beta = e^{2t}\beta\]

    and modify the interval \([\epsilon, 2-\epsilon]\) to \([-N, N]\). The work in \cref{sec:nahm} shows that \(g\) has the required properties.
\end{proof}

Now for \(h \in \mcH\), with eigenvalues \(\lambda_1, \dots, \lambda_k\), define

\[\Psi(h) = \log\max(\lambda_i)\]

Since \(\det(h) = 1\), \(\Psi(h) = 0\) if and only if \(h = 1\). Moreover, if \(h(t)\) is continuous, then \(\Psi(h(t))\) is as well.

\begin{lemma}
    If \((\alpha', \beta') = g \vdot (\alpha, \beta)\) over some interval in \(\R\), then with \(h = g^*g\),

    \[\dv[2]{t}\Psi(h) + 2\dv{t}\Psi(h) \ge -2\left(\abs{\hat F(\alpha, \beta)} + \abs{\hat F(\alpha', \beta')}\right)\]

    weakly. Note the norm on the right hand side is defined using the Killing form.
\end{lemma}

\begin{proof}
    We want to use \cite[Lemma 2.10]{donaldson_nahms_1984}. First, we will write the left hand side in terms of \(s\). In this case, we have

    \begin{align*}
        \dv{\Psi}{s} &= \dv{\Psi}{t}\dv{t}{s} \\
        \dv[2]{\Psi}{s} &= \dv[2]{\Psi}{t}\left(\dv{t}{s}\right)^2 + \dv{\Psi}{t}\dv[2]{t}{s} \\
        &= e^{4t}\left(\dv[2]{\Psi}{t} + 2\dv{\Psi}{t}\right) 
    \end{align*}

    Next, note that the real equation for Nahm's equations, \cref{eq:nahm-real-equation}, is

    \begin{align*}
        \dv{s}(\tilde\alpha + \tilde\alpha^*) + 2([\tilde\alpha, \tilde\alpha^*] + [\tilde\beta, \tilde\beta^*]) &=\dv{t}(e^{2t}(\alpha + \alpha^*))\dv{t}{s} + 2e^{4t}([\alpha, \alpha^*] + [\beta, \beta^*]) \\
        &= e^{4t}\left(\dv{t}(\alpha + \alpha^*) + 2(\alpha + \alpha^*) + 2([\alpha, \alpha^*] + [\beta, \beta^*])\right)
    \end{align*}

    Therefore, compared to \cite[Lemma 2.10]{donaldson_nahms_1984}, we have a factor of \(e^{4t}\) on both sides, which is a positive function. Therefore, the result follows.
\end{proof}

\begin{proposition}
    Suppose \((\alpha', \beta')\) and \((\alpha'', \beta'')\) are equivalent complex trajectories, satisfying the real equation \cref{eq:real-equation}, then \((\alpha'', \beta'') = g\vdot(\alpha', \beta')\) for some \(g \in \mcG\), i.e. \(g : \R \to \SU(n)\), with \(g(t) \to 1\) as \(t \to \infty\).
\end{proposition}

\begin{proof}
    Suppose \((\alpha', \beta')\) and \((\alpha'', \beta'') = g\vdot(\alpha', \beta')\) both satisfy the real equation. Setting \(h = h(g)\) and \(\Psi = \Psi(h)\), we find that

    \[\ddot \Psi + 2\dot\Psi \ge 0\]

    Using the same computation as in the previous lemma, this implies that

    \[\dv[2]{\Psi}{s} \ge 0\]

    and so \(\Psi(s)\) is convex. The other conditions transform to \(\Psi : (-\infty, 0) \to \R\) as: \(\Psi(s) \to 0\) as \(s \to 0\), \(\Psi(s)\) bounded and nonnegative. This then implies that \(\Psi\) must be identically zero. Hence \(h = 1\), and so \(g^*g = 1\). That is, \(g\) takes values in \(\SU(n)\).
\end{proof}

\subsection{Existence}

Let \((\alpha, \beta)\) be a solution to the complex equations. We can assume without loss of generality that \((\alpha, \beta)\) is in the form \cref{eq:complex-trajectory-form}.

\begin{lemma}
    If \((\alpha, \beta)\) are in the form as in \cref{eq:complex-trajectory-form}, and \(\epsilon \in Z(X_+)\), then

    \[
    \begin{cases}
        \hat F(\alpha, \beta) = 0 & \text{on }\Ioc{-\infty, 0} \\
        \abs{\hat F(\alpha, \beta)} \le Ce^{-4t} & \text{on }\Ico{0, \infty}
    \end{cases}\]
\end{lemma}

\begin{proof}
    In both cases, since \(\rho_{\pm}\) are representations of \(\su(2)\), we have that

    \begin{align*}
        H_{\pm}^* &= H_{\pm} \\
        X_{\pm}^* &= Y_{\pm} \\
        Y_{\pm}^* &= X_{\pm} \\
    \end{align*}

    Thus, in the first case, we have

    \[2H_- + 2[Y_-, X_-] = 0\]

    which is true as \(\rho_-\) is a representation of \(\sl(2, \C)\).
\end{proof}

\appendix

\section{Representation theory of \(\sl(2, \C)\)}

In this section, we will sketch the representation theory of \(\sl(2, \C)\). For more details, see \cite[Section 7]{humphreys}.

Let \(V\) be a complex vector space. Then a representation of \(\sl(2, \C)\) is a Lie algebra homomorphism \(\rho : \sl(2, \C) \to \gl(V)\). When clear, we will write \(X \vdot v := \rho(X)(v)\). Choose the basis

\[X = \begin{pmatrix}
    0 & 1 \\
    0 & 0
\end{pmatrix} \quad Y = \begin{pmatrix}
    0 & 0 \\
    1 & 0
\end{pmatrix} \quad H = \begin{pmatrix}
    1 & 0 \\
    0 & -1
\end{pmatrix}\]

for \(\sl(2, \C)\). The commutators are \([H, X] = 2X, [H, Y] = -2Y, [X,Y] = H\)

We first note that \(\rho(H)\) is diagonalisable, and so we have a Jordan decomposition

\[V = \bigoplus_{\lambda \in \C}V_\lambda\]

where

\[V_\lambda = \left\{v \in V \mid H \cdot v = \lambda v\right\}\]

is the \(\lambda\)-eigenspace of \(H\). In fact, we have:

\begin{enumerate}
    \item \[V = \bigoplus_{\lambda \in \Z}V_\lambda\]
    \item if \(v \in V_\mu\), then \(X \cdot v \in V_{\mu+2}\) and \(Y \cdot v \in V_{\mu-2}\).
\end{enumerate}

\section{Distributions}

\label{sec:distributions}

Let \(\Omega \subseteq \R^n\) be open and connected.

\begin{definition}
    [positive]
    We say that \(u \in \mcD'(\Omega)\) is \emph{positive} if for all \(\phi \in C_c^\infty(\Omega)\), with \(\phi \ge 0\), \(u[\phi] \ge 0\). We write this as \(u \ge 0\). 
\end{definition}

\begin{definition}
    [derivative]

    The derivative of a distribution \(u \in \mcD'(\Omega)\) is the distribution \(Du\) given by

    \[Du[\phi] = -u[D\phi]\]
\end{definition}

Finally, recall that we have an embedding \(T : L^1_\text{loc.}(\Omega) \to \mcD'(\omega)\), given by

\[T_f(\phi) = \int_\Omega f\phi \dd x\]

We will abuse notation and write \(Df = DT_f\). With this, let

\[L = \sum_{k=0}^d a_k D^k\]

be a linear differential operator, \(a_k : \Omega \to \R\) smooth. Suppose \(Lf \ge 0\). Then we say that the differential inequality \(Lf \ge 0\) holds \emph{weakly}.

\printbibliography

\end{document}
