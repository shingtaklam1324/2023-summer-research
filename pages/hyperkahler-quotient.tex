\documentclass{article}

\usepackage{../Style}
\usepackage{stmaryrd}

\newcommand{\surj}{\twoheadrightarrow}
\renewcommand{\tilde}{\widetilde}

\title{HyperK\"ahler quotients}
\author{Shing Tak Lam}

\begin{document}

\maketitle

Let \(M\) be a hyperK\"ahler manifold, that is, we have:

\begin{enumerate}
    \item a \(4n\)-manifold \(M\),
    \item a Riemannian metric \(g\) on \(M\),
    \item almost complex structures \(I, J, K\) on \(M\), with \((M, g, I), (M, g, J), (M, g, K)\) all K\"ahler manifolds, and satisfying the quaternionic relations \(I^2 = J^2 = K^2 = IJK = -1\).
\end{enumerate}

Let \(G\) be a Lie group, acting on \(M\) preserving the K\"ahler structures, say with moment maps \(\mu_I, \mu_J, \mu_K\) respectively. Then we can define the \emph{hyperK\"ahler moment map}

\[\mu = \mu_I i + \mu_J j + \mu_K k : M \to \mfg^* \otimes \Im (\bb H)\]

We want to show that \(\mu^{-1}(0)/G\) is a hyperK\"ahler manifold. Fix the complex structure \(I\), and we can define the \emph{real} and \emph{complex moment maps}

\begin{align*}
    \mu_r &= \mu_I &: M \to \mfg^* \\
    \mu_c &= \mu_J + \mu_K i &: M \to \mfg_\C^*
\end{align*}

With this in mind, we have the following diagram for how to obtain the hyperK\"ahler structure on \(\mu^{-1}(0)/G\):

% https://q.uiver.app/#q=WzAsNCxbMSwwLCJNIl0sWzEsNCwiXFxtdV57LTF9KDApL0ciXSxbMiwyLCJcXG11X2Neey0xfSgwKSJdLFswLDIsIlxcbXVfcl57LTF9KDApL0ciXSxbMiwxLCJrIiwyLHsic3R5bGUiOnsiaGVhZCI6eyJuYW1lIjoiZXBpIn19fV0sWzIsMCwiIiwyLHsic3R5bGUiOnsidGFpbCI6eyJuYW1lIjoiaG9vayIsInNpZGUiOiJ0b3AifX19XSxbMCwzLCJrIiwyLHsic3R5bGUiOnsiaGVhZCI6eyJuYW1lIjoiZXBpIn19fV0sWzEsMywiIiwxLHsic3R5bGUiOnsidGFpbCI6eyJuYW1lIjoiaG9vayIsInNpZGUiOiJ0b3AifX19XV0=
\begin{equation}
	\label{eq:diag-hk-quot}
	\begin{tikzcd}
		& M \\
		\\
		{\mu_r^{-1}(0)/G} && {\mu_c^{-1}(0)} \\
		\\
		& {\mu^{-1}(0)/G}
		\arrow["k"', two heads, from=3-3, to=5-2]
		\arrow[hook, from=3-3, to=1-2]
		\arrow["k"', two heads, from=1-2, to=3-1]
		\arrow[hook, from=5-2, to=3-1]
	\end{tikzcd}
\end{equation}

where

\begin{enumerate}
    \item \(A \stackrel{k}{\surj} B\) indicates that \(B\) is the K\"ahler quotient of \(A\),
    \item \(A \hookleftarrow B\) indicates that \(B\) is a submanifold of \(A\),
    \item each horizontal level has the same dimension.
\end{enumerate}

\section*{\(\mu_c^{-1}(0)\) is a complex submanifold.}

Let \(\omega_I, \omega_J, \omega_K\) be the K\"ahler forms on \(M\). Fix \(X \in \mfg\) and let \(X^\#\) be the corresponding vector field on \(M\) generated by \(X\). Then for any vector field \(Y\) on \(M\), we have that

\[\dd\mu_c^X(Y) = \dd\mu_J^X(Y) + i\dd\mu_K^X(Y) = \omega_J(X^\#, Y) + i\omega_K(X^\#, Y) = g(J \cdot X^\#, Y) + ig(K \cdot X^\#, Y)\]

Therefore,

\[\dd \mu_c^X(I \cdot Y) = g(J \cdot X^\#, I \cdot Y) + ig(K \cdot X^\#, I \cdot Y) = -g(K \cdot X^\#, Y) + ig(J \cdot X^\#, Y) = i\dd\mu_c^X(Y)\]

and so \(\mu_c^X\) is holomorphic. Thus, \(\mu_c^{-1}(0) = \mu_J^{-1}(0) \cap \mu_K^{-1}(0)\) is a complex submanifold of \(M\). In particular, \(\mu_c^{-1}(0)\) is a K\"ahler manifold.

Moreover, \(G\) acts on \(\mu_c^{-1}(0)\) by equivariance, and it preserves the K\"ahler structure, and the moment map is \(\mu_r\vert_{\mu_c^{-1}(0)}\). Therefore, if we take the K\"ahler quotient, we get \(\mu^{-1}(0)/G\).

\section*{Compatibility of the diagram}

Let \(\pi : \mu_r^{-1}(0) \to \mu_r^{-1}(0)/G\) be the quotient map, \(V_p = \ker(\dd\pi_p)\) is a vector bundle on \(\mu_r^{-1}(0)\). Let \(H_p\) be the orthogonal complement of \(V_p\) in \(\TT_p\mu^{-1}(0)\), given by (the restriction of) the Riemannian metric \(g\). Using this, one can show that the complex structure \(I_p : T_pM \to T_pM\) restricts to a map \(H_p \to H_p\). Moreover, \(\pi_p : H_p \to \TT_{\pi(p)}\left(\mu_r^{-1}(0)/G\right)\) is a linear isomorphism.

We will use the notation

\begin{equation}
    \label{eq:iso}
    \begin{split}
        H_p &\cong \TT_{\pi(p)}\left(\mu_r^{-1}(0)/G\right) \\
        X &\mapsto X_* \\
        Y^* & \mapsfrom Y
    \end{split}
\end{equation}

for this isomorphism. Using this, we have the formulae

\begin{equation}
    \label{eq:forms}
    \begin{split}
        \tilde\omega(X, Y) &= \omega(X^*, Y^*) \\
        \tilde g(X, Y) &= g(X^*, Y^*) \\
        \tilde I(X) &= I(X^*)_*
    \end{split}
\end{equation}

for the K\"ahler structure \((\tilde\omega, \tilde g, \tilde I)\) on \(\mu_r^{-1}(0)/G\).

Similarly, let \(\psi : \mu^{-1}(0) \to \mu^{-1}(0)/G\) be the quotient map. Then we have a similar isomorphism to the above, which we will denote by

\begin{equation}
    \label{eq:iso2}
    \begin{split}
        \overline H_p &\cong \TT_{\psi(p)}\left(\mu^{-1}(0)/G\right) \\
    X &\mapsto X_\bigstar \\
    Y^\bigstar & \mapsfrom Y
    \end{split}
\end{equation}

With this, we then get the formulae

\begin{equation}
    \label{eq:forms2}
    \begin{split}
        \overline\omega(X, Y) &= \omega(X^\bigstar, Y^\bigstar) \\
        \overline g(X, Y) &= g(X^\bigstar, Y^\bigstar) \\
        \overline I(X) &= I(X^\bigstar)_\bigstar
    \end{split}
\end{equation}

for the K\"ahler structure \((\overline\omega, \overline g, \overline I)\) on \(\mu^{-1}(0)/G\).

However, \(\psi\) is just the restriction of \(\pi\) to \(\mu^{-1}(0) \subseteq \mu_r^{-1}(0)\), \(\overline H_p = H_p \cap \TT_p\mu^{-1}(0)\). With this in mind, we can see that \(X_* = X_\bigstar\) and \(Y^* = Y^\bigstar\). What this also implies is that \cref{eq:forms2} is just the restriction of \cref{eq:forms} to \(\mu^{-1}(0)/G\), \(\mu^{-1}(0)/G\) is a complex submanifold of \(\mu_r^{-1}(0)/G\), and that whether we take the K\"ahler quotient first or the complex submanifold first, we get the same result.

\section*{HyperK\"ahler triple}

Repeating the above, we get three K\"ahler structures on \(\mu^{-1}(0)/G\). We want to show that they are compatible. That is, we need to show that the Riemannian metrics are the same, and that the complex structures satisfy the quaternionic relations.

We first consider the Riemannian metric. We have the following diagram of inclusions and quotients:

% https://q.uiver.app/#q=WzAsNSxbMSwwLCJNIl0sWzEsNCwiXFxtdV57LTF9KDApL0ciXSxbMiwzLCJcXG11XnstMX0oMCkiXSxbMCwxLCJcXG11X3Jeey0xfSgwKSJdLFswLDIsIlxcbXVfcl57LTF9KDApL0ciXSxbMywwLCIiLDIseyJzdHlsZSI6eyJ0YWlsIjp7Im5hbWUiOiJob29rIiwic2lkZSI6InRvcCJ9fX1dLFszLDQsIiIsMCx7InN0eWxlIjp7ImhlYWQiOnsibmFtZSI6ImVwaSJ9fX1dLFsyLDAsIiIsMCx7InN0eWxlIjp7InRhaWwiOnsibmFtZSI6Imhvb2siLCJzaWRlIjoidG9wIn19fV0sWzIsMSwiIiwyLHsic3R5bGUiOnsiaGVhZCI6eyJuYW1lIjoiZXBpIn19fV0sWzEsNCwiIiwyLHsic3R5bGUiOnsidGFpbCI6eyJuYW1lIjoiaG9vayIsInNpZGUiOiJ0b3AifX19XSxbMiwzLCIiLDEseyJzdHlsZSI6eyJ0YWlsIjp7Im5hbWUiOiJob29rIiwic2lkZSI6InRvcCJ9fX1dXQ==
\[\begin{tikzcd}
	& M \\
	{\mu_r^{-1}(0)} \\
	{\mu_r^{-1}(0)/G} \\
	&& {\mu^{-1}(0)} \\
	& {\mu^{-1}(0)/G}
	\arrow[hook, from=2-1, to=1-2]
	\arrow[two heads, from=2-1, to=3-1]
	\arrow[hook, from=4-3, to=1-2]
	\arrow[two heads, from=4-3, to=5-2]
	\arrow[hook, from=5-2, to=3-1]
	\arrow[hook, from=4-3, to=2-1]
\end{tikzcd}\]

From the fact that \cref{eq:forms2} and \cref{eq:forms} give the same result, we can see that the Riemannian metric is the one given by first restricting to \(\mu^{-1}(0)\) and then taking the Riemannian quotient. Therefore, in all three cases the Riemannian metric is the same.

Moreover, \(\psi\) is the same in all three cases, therefore, we can compute using \cref{eq:forms2} that

\[\overline I\ \overline J(X) = \overline I(J(X^*)_*) = (IJ(X^*))_* = K(X^*)_* = \overline K(X)\]

So \(\overline I\ \overline J = \overline K\).

\section*{Complex quotients}

The starting point is the following statement

\begin{theorem*}
    Let \(X\) be a K\"ahler manifold, \(G\) acting on \(X\) preserving the K\"ahler structure. Suppose \(\mu\) is a moment map for this action, and \(G_\C\) acts holomorphically on \(X\). Then the natural map

	\[\mu^{-1}(0)/G \to X/G^\C\]

	is a biholomorphism.
\end{theorem*}

\textbf{The above statement is false.} We need extra assumptions on \(X, G\) and \(\mu\), and the right hand side should be \(\tilde X/G^\C\) for some open subset \(\tilde X\) of \(X\). However, it is useful to have the above statement in mind.

We would like to apply this to the hyperK\"ahler quotient. From \cref{eq:diag-hk-quot}, we can see that there are two possible ways to do this. The difference in these two cases is that the K\"ahler manifold \(X\) which we apply the theorem to (and hence have to check the conclusions for) are different.

\subsection*{Applying to \(G\) acting on \((M, I)\)}

In this case, we have the following diagram of inclusions and quotients:

% https://q.uiver.app/#q=WzAsNyxbMiwwLCJNIl0sWzIsMiwiTS9HX1xcQyJdLFswLDIsIlxcbXVfcl57LTF9KDApL0ciXSxbMCwxLCJcXG11X3Jeey0xfSgwKSJdLFsyLDQsIlxcbXVfY157LTF9KDApL0dfXFxDIl0sWzMsMiwiXFxtdV9jXnstMX0oMCkiXSxbMCw0LCJcXG11XnstMX0oMCkvRyJdLFszLDAsIiIsMix7InN0eWxlIjp7InRhaWwiOnsibmFtZSI6Imhvb2siLCJzaWRlIjoidG9wIn19fV0sWzMsMiwiIiwwLHsic3R5bGUiOnsiaGVhZCI6eyJuYW1lIjoiZXBpIn19fV0sWzAsMSwiIiwyLHsic3R5bGUiOnsiaGVhZCI6eyJuYW1lIjoiZXBpIn19fV0sWzIsMSwiXFxjb25nIiwwLHsic3R5bGUiOnsiaGVhZCI6eyJuYW1lIjoibm9uZSJ9fX1dLFs0LDEsIiIsMix7InN0eWxlIjp7InRhaWwiOnsibmFtZSI6Imhvb2siLCJzaWRlIjoidG9wIn19fV0sWzUsNCwiIiwyLHsic3R5bGUiOnsiaGVhZCI6eyJuYW1lIjoiZXBpIn19fV0sWzUsMCwiIiwwLHsic3R5bGUiOnsidGFpbCI6eyJuYW1lIjoiaG9vayIsInNpZGUiOiJ0b3AifX19XSxbNiwyLCIiLDIseyJzdHlsZSI6eyJ0YWlsIjp7Im5hbWUiOiJob29rIiwic2lkZSI6InRvcCJ9fX1dLFs2LDQsIiIsMix7InN0eWxlIjp7ImJvZHkiOnsibmFtZSI6ImRhc2hlZCJ9LCJoZWFkIjp7Im5hbWUiOiJub25lIn19fV1d
\[\begin{tikzcd}
	&& M \\
	{\mu_r^{-1}(0)} \\
	{\mu_r^{-1}(0)/G} && {M/G_\C} & {\mu_c^{-1}(0)} \\
	\\
	{\mu^{-1}(0)/G} && {\mu_c^{-1}(0)/G_\C}
	\arrow[hook, from=2-1, to=1-3]
	\arrow[two heads, from=2-1, to=3-1]
	\arrow[two heads, from=1-3, to=3-3]
	\arrow["\cong", no head, from=3-1, to=3-3]
	\arrow[hook, from=5-3, to=3-3]
	\arrow[two heads, from=3-4, to=5-3]
	\arrow[hook, from=3-4, to=1-3]
	\arrow[hook, from=5-1, to=3-1]
	\arrow[dashed, no head, from=5-1, to=5-3]
\end{tikzcd}\]

Applying the theorem gives us that the natural map \(\phi : \mu_r^{-1}(0)/G \to M/G_\C\) is a biholomorphism. We want to show that \(\phi\) maps \(\mu^{-1}(0)/G\) to \(\mu_c^{-1}(0)/G_\C\). Denote orbits in \(\mu_r^{-1}(0)/G\) by \([x]\) and orbits in \(M/G_\C\) by \(\llbracket x \rrbracket\). Then \(\phi([x]) = \llbracket x\rrbracket\). Moreover, as \(G_\C\) acts on \(\mu_c^{-1}(0)\), we get that

\[\phi^{-1}(\mu_c^{-1}(0)/G_\C) = \left\{[x] \mid x \in \mu_r^{-1}(0) \cap \mu_c^{-1}(0) = \mu^{-1}(0)\right\} = \mu^{-1}(0)/G\]

Therefore, we have a biholomorphism \(\mu^{-1}(0)/G \cong \mu_c^{-1}(0)/G_\C\).

\subsection*{Applying to \(G\) acting on \(\mu_c^{-1}(0)\)}

In this case, we have

% https://q.uiver.app/#q=WzAsNSxbMiwyLCJcXG11X2Neey0xfSgwKSJdLFsxLDAsIk0iXSxbMCwzLCJcXG11XnstMX0oMCkiXSxbMCw0LCJcXG11XnstMX0oMCkvRyJdLFsyLDQsIlxcbXVfY157LTF9KDApL0dfXFxDIl0sWzIsMSwiIiwwLHsic3R5bGUiOnsidGFpbCI6eyJuYW1lIjoiaG9vayIsInNpZGUiOiJ0b3AifX19XSxbMiwzLCIiLDIseyJzdHlsZSI6eyJoZWFkIjp7Im5hbWUiOiJlcGkifX19XSxbMCwxLCIiLDIseyJzdHlsZSI6eyJ0YWlsIjp7Im5hbWUiOiJob29rIiwic2lkZSI6InRvcCJ9fX1dLFswLDQsIiIsMCx7InN0eWxlIjp7ImhlYWQiOnsibmFtZSI6ImVwaSJ9fX1dLFszLDQsIlxcY29uZyIsMCx7InN0eWxlIjp7ImhlYWQiOnsibmFtZSI6Im5vbmUifX19XSxbMiwwLCIiLDEseyJzdHlsZSI6eyJ0YWlsIjp7Im5hbWUiOiJob29rIiwic2lkZSI6InRvcCJ9fX1dXQ==
\[\begin{tikzcd}
	& M \\
	\\
	&& {\mu_c^{-1}(0)} \\
	{\mu^{-1}(0)} \\
	{\mu^{-1}(0)/G} && {\mu_c^{-1}(0)/G_\C}
	\arrow[hook, from=4-1, to=1-2]
	\arrow[two heads, from=4-1, to=5-1]
	\arrow[hook, from=3-3, to=1-2]
	\arrow[two heads, from=3-3, to=5-3]
	\arrow["\cong", no head, from=5-1, to=5-3]
	\arrow[hook, from=4-1, to=3-3]
\end{tikzcd}\]

and the theorem directly gives us the fact that \(\mu^{-1}(0)/G \cong \mu_c^{-1}(0)/G_\C\).

\end{document}
