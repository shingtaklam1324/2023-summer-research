\documentclass{article}

\title{K\"ahler structure on reduction via complex manifolds}
\author{Shing Tak Lam}

\usepackage{../Style}

\newcommand{\sslash}{/\!/}
\DeclareMathOperator{\gr}{grad}

\begin{document}

\maketitle

Throughout, let \(M\) be a compact K\"ahler manifold, \(G\) a compact Lie group acting on \(X\) preserving the K\"ahler structure. Let \(\mu : M \to \mfg^*\) be the moment map for the action. Suppose in addition that \(G\) acts freely on \(\mu^{-1}(0)\). Then we have that

\[M \sslash G := \mu^{-1}(0)/G\]

has a K\"ahler structure. In this note, we show another way of thinking about this, which is that

\[\mu^{-1}(0)/G \cong M^{\min} / G_\C\]

where \(M^s\) is an open submanifold of \(M\), and \(G_\C\) is the complexification of \(G\). In this case, the fact that \(M\sslash G\) has a complex structure becomes clear.

Fix a \(G\)-invariant Riemannian metric on \(M\), and define \(f : M \to \R\) by

\[f(x) = \norm{\mu(x)}^2\]

We want to consider \(f\) as a Morse function on \(M\). For \(x \in M\), define the curve \((x_t)_{t \ge 0}\) by

\begin{align*}
    x_0 &= x \\
    \dv{x_t}{t} &= -\gr f(x_t)
\end{align*}

The set of limit points of the trajectory is

\[\omega(x) = \left\{y \in M \mid \text{every neighbourhood of }y\text{ contains }x_t\text{ for }t\text{ arbitrarily large}\right\}\]

With this, we define

\[M^{\min} = \left\{x \in X \mid \omega(x) \subseteq \mu^{-1}(0)\right\}\]

Then \(M^{\min}\) is a \(G_\C\) invariant open subset of \(M\), with \(\mu^{-1}(0) \subseteq M^{\min}\). The inclusion map induces a natural continuous map

\[\mu^{-1}(0)/G \to M^{\min}/G_\C\]

which is a homeomorphism. The proof of this relies on the fact that \(G \cdot \mu^{-1}(0) = M^{\min}\), hence the map is surjective. Another argument shows the map is bijective. Therefore it is a continuous bijection from a compact space to a Hausdorff space, hence a homeomorphism.

\end{document}
