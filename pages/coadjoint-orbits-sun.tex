\documentclass{article}

\usepackage{../Style}
\usepackage{stmaryrd}

\DeclareMathOperator{\SU}{SU}
\newcommand{\su}{\mathfrak{su}}
\renewcommand{\sl}{\mathfrak{sl}}
% \newcommand{\rm}[1]{\mathrm{#1}}
\renewcommand{\tilde}{\widetilde}
\newcommand{\surj}{\twoheadrightarrow}
\DeclareMathOperator{\gr}{grad}
\DeclareMathOperator{\Gr}{Gr}

\title{Coadjoint Orbits of \(\SU(n)\)}
\author{Shing Tak Lam}

\begin{document}

\maketitle

In this note, we will consider the coadjoint orbits of \(\SU(n)\), and show that they are K\"ahler manifolds.

\tableofcontents

\section{Adjoint and Coadjoint Orbits}

Define the Lie algebra

\[\su(n) = \left\{A \in \Mat(n, \C) \mid A^* + A = 0, \tr(A) = 0\right\}\]

where \(A^*\) is the conjugate transpose of \(A\), and with the Lie bracket being the matrix commutator. We can define the adjoint representation of \(\SU(n)\) as

\begin{align*}
    \Ad : \SU(n) &\to \GL(\su(n)) \\
    \Ad_g(X) &= gXg^\dagger
\end{align*}

Taking the dual representation, we get the coadjoint representation, which is

\begin{align*}
    \Ad^* : \SU(n) &\to \GL(\su(n)^*) \\
    \Ad^*_g(\alpha)(X) &= \alpha(\Ad_{g^{-1}}(X))
\end{align*}

Now note that

\[\inner{A, B} = -\tr(AB) = \tr(AB^*)\]

defines an inner product on \(\su(n)\)\footnote{In fact, \(\inner{A, B} = \tr(AB^*)\) defines a Hermitian inner product on the space of complex matrices.}, which means that we have a natural isomorphism

\begin{align*}
    \Phi : \su(n) &\to \su(n)^* \\
    A &\mapsto \inner{A, \cdot}
\end{align*}

With this, suppose \(\alpha = \Phi(A)\), then

\[\Ad_g^*(\alpha)(X) = \inner{A, \Ad_{g^{-1}}(X)} = -\tr(Ag^{-1}Xg) = -\tr(gAg^{-1}X) = \Phi(\Ad_g(A))(X)\]

Therefore, up to identification by \(\Phi\), the adjoint and coadjoint representations are the same.

\section{Tangent space}

Let \(\mcO^*\) be a coadjoint orbit. For \(X \in \su(n)\), consider the curve \(g(t) = \exp(tX)\) in \(G\). This has \(g'(0) = X\), and we have a curve

\[\mu(t) = \Ad_{g(t)}^*(\mu)\]

through \(\mu \in \mcO^*\). In particular, we have that for \(Y \in \su(n)\),

\[\inner{\mu(t), Y} = \inner{\mu, \Ad_{g(t)^{-1}}(Y)}\]

Differentiating this at \(t = 0\), we get

\[\inner{\mu'(0), Y} = -\inner{\mu, \ad_X(Y)} = -\inner{(\ad_X)^*(\mu), Y}\]

That is, \(\mu'(0) = -(\ad_X)^*(\mu)\). Hence we have that

\[\T_\mu\left(\mcO^*\right) = \left\{(\ad_X)^*(\mu) \mid X \in \su(n)\right\}\]

If \(\alpha= \Phi(A)\), then

\begin{align*}
    \inner{(\ad_X)^*(\alpha), Y} &= \inner{\alpha, \ad_X(Y)} \\
    &= \inner{A, [X, Y]} \\
    &= -\tr(AXY - AYX) \\
    &= -\tr(AXY - XAY) \\
    &= \inner{[A, X], Y} \\
    &= \inner{-\ad_X(A), Y}
\end{align*}

Hence by identification with \(\Phi\), \((\ad_X)^* = -\ad_X\). Thus, in this case we have the tangent space to the corresponding adjoint orbit as

\[T_A\mcO = \left\{\ad_X(A) \mid X \in \su(n)\right\}\]

\section{Diagonalisation}

Since all elements in \(\mcO\) are diagonalisable by \(\SU(n)\), we can choose \(A \in \mcO\) to be diagonal. Moreover, the eigenvalues are pure imaginary, so we can assume

\[A = \begin{pmatrix}
    i\lambda_1 & \\
    & \ddots & \\
    & & i\lambda_n
\end{pmatrix}\]

with \(\lambda_1 \ge \cdots \ge \lambda_n\). Then define the projection map \(\pi : \SU(n) \to \mcO\) by

\[\pi(g) = \Ad_g(A) = gAg^\dagger\]

and it is easy to check that \(\pi\) is a surjective submersion. In particular, if \(\alpha\) is any form on \(\mcO\), then \(\alpha\) is determined by its pullback \(\pi^*\alpha\).

\section{Symplectic structure}

For details in this section, see \texttt{kirillov-kostant.pdf} or Marsden-Ratiu Chapter 14.

The Kirillov-Kostant-Souriau form on \(\mcO^*\) is given by

\[\omega_\mu(-(\ad_X)^*(\mu), -(\ad_Y)^*(\mu)) = -\inner{\mu, [X, Y]}\]

Therefore, using the isomorphism \(\Phi\), we have a corresponding symplectic form on \(\mcO\) given by

\begin{equation}
    \label{eq:kks-form}
    \omega_B(\ad_X(B), \ad_Y(B)) = -\inner{B, [X, Y]}
\end{equation}

Then \(\pi^*\omega = \dd\alpha_\ell\), where \(\alpha = \Phi(A)\), and \(\alpha_\ell\) is the left-invariant \(1\)-form on \(\SU(n)\) taking value \(\alpha\) at \(1\). That is,

\[\alpha_\ell(g) = ((\dd\ell_{g^{-1}})_g)^*(\alpha)\]

Using this, we get that \(\pi^*\dd\omega = \dd\pi^*\omega = \dd^2\alpha_\ell = 0\), and as \(\pi\) is a surjective submersion, \(\dd\omega = 0\). The facts that \cref{eq:kks-form} is well defined and non-degenerate are easy to check as well.

\section{Complex structure}

From \S 3, we assumed that \(A\) is diagonal. Suppose the eigenvalues have algebraic multiplicities \(n_1, \dots, n_k\). Then we have that the stabiliser of \(A\) is the block diagonal subgroup

\[\Stab(A) = \rm S(\rm U(n_1) \times \cdots \times \rm U(n_k))\]

and the quotient space is the flag manifold

\[\mcF(n_1, \dots, n_k) = \frac{\SU(n)}{\Stab(A)}\]

We won't discuss this any further, see the appendix for details. We will now focus on the generic case, where \(A\) has distinct eigenvalues. In this case, the stabiliser is the diagonal subgroup \(T\), which is isomorphic to the torus \(T^{n-1} = (S^1)^{n-1}\).

Let \(P\) be the subgroup of lower triangular matrices in \(\SL(n, \C)\). Consider the composition \(\varphi : \SU(n) \to \SL(n, \C)/P\) given by the composition 

% https://q.uiver.app/#q=WzAsMyxbMCwwLCJcXFNVKG4pIl0sWzIsMCwiXFxTTChuKSJdLFs0LDAsIlxcU0wobiwgXFxDKS9QIl0sWzAsMSwiIiwwLHsic3R5bGUiOnsidGFpbCI6eyJuYW1lIjoiaG9vayIsInNpZGUiOiJ0b3AifX19XSxbMSwyLCIiLDAseyJzdHlsZSI6eyJoZWFkIjp7Im5hbWUiOiJlcGkifX19XV0=
\[\begin{tikzcd}
	{\SU(n)} && {\SL(n)} && {\SL(n, \C)/P}
	\arrow[hook, from=1-1, to=1-3]
	\arrow[two heads, from=1-3, to=1-5]
\end{tikzcd}\]

Suppose \(\varphi(g) = \varphi(h)\). That is, \(gP = hP\). This is true if and only if there exists \(p \in P\), such that \(h = gp\). In this case, \(p = g^{-1}h \in \SU(n)\), therefore, \(p \in \SU(n) \cap P = T\), since \(p^\dagger = p^{-1}\) is also lower triangular. This means that \(\varphi\) induces a homeomorphism \(\SU(n)/T \cong \SL(n, \C)/P\). The right hand side is a complex manifold \((\SL(n, \C))\) quotiented by a complex Lie group \(P\), so it is a complex manifold. Using the above, we can get a complex structure on \(\SU(n)/T \cong \mcO\).

\section{Local coordinates}

This section is from \texttt{un\_kks.pdf}, which follows an exercise from Prof. Bryant's notes. The computations to verify these expressions can be found in \texttt{un\_kks.pdf}, and have been omitted.

let \(\theta\) be the Maurer-Cartan form on \(\SU(n)\). That is, it is the \(\su(n)\)-valued \(1\)-form on \(\SU(n)\) given by

\[\theta_g(u) = \dd (\ell_{g^{-1}})_g(u) \in \TT_e\SU(n) = \su(n)\]

Writing \(\theta = \sum_{j, k}\theta_{jk}\dd g^{jk}\) where \((g^{jk})\) are the matrix entries on \(\SU(n)\). Then we have that

\[\pi^*\omega = i \sum_{k > j}(\lambda_j - \lambda_k)\theta_{kj}\wedge \overline{\theta_{kj}}\]

and the Hermitian metric \(h\) on \(\mcO\) is given by

\[\pi^*h = \sum_{k > j}2(\lambda_j - \lambda_k)\theta_{kj}\overline{\theta_{kj}}\]

Using the fact that

\[\TT_g\SU(n) = g\su(n)\]

and that

\[\dd\pi_g(h) = hAg^\dagger + gAh^\dagger\]

we get that

\[\dd\pi_g(h) = [B, hg^\dagger]\]

where \(B = \pi(g) = gAg^\dagger\). With this, we can recover the hermitian metric, as

\[h_B([B, C], [B, D]) = \sum_{k > j}2(\lambda_j - \lambda_k)(g^\dagger Cg)_{kj}\overline{(g^\dagger Dg)_{kj}}\]

With this, we can see that \(\omega = -\Im(h)\), and that the real part \(g = \Re(h)\)\footnote{The usual notation for a Riemannian metric is \(g\), but I've also used \(g\) as the element of \(\SU(n)\). Hopefully it should be clear from context which one is which.} defines a Riemannian metric.

Moreover, we can recover the almost complex structure from the symplectic form and the Riemannian metric, via

\[J = \tilde g^{-1} \circ \tilde\omega\]

where \(\tilde\omega, \tilde g : \TT \SU(n) \to \TT^*\SU(n)\) are linear isomorphisms given by

\begin{align*}
    \tilde \omega(u)(v) &= \omega(u, v) \\
    \tilde g(u)(v) &= g(u, v)
\end{align*}

This follows as we have that \(g(u, v) = \omega(u, Jv)\).

\section{Root decomposition}

As the expressions we have above for \(\pi^*\omega\) and \(\pi^*h\) are left invariant, suffices to consider them at \(A\), since we can then use left translation in \(\SU(n)\) to get the expression at a different tangent space. This section should connect the results in the previous two sections.

Consider the Lie algebra \(\sl(n, \C)\). Then we have the Cartan subalgebra \(\mft\) of diagonal matrices. Let \(E_{ij}\) be the standard basis matrices for \(\Mat(n, \C)\), \(B \in \mft\). Say

\[B = \begin{pmatrix}
    b_1 \\
    & \ddots \\
    & & b_n
\end{pmatrix}\]

Then \([B, E_{ij}] = (b_i - b_j)E_{ij}\). This means that we have the eigendecomposition

\begin{equation}
    \label{eq:sln-root}
    \sl(n, \C) = \mft \oplus \bigoplus_{1 \le i, j \le n, i \ne j} \C E_{ij}
\end{equation}

In particular, if we restrict this to the subalgebra \(\su(n)\), we get the decomposition

\[\su(n) = \tilde\mft \oplus \bigoplus_{1 \le i < j \le n} \left(\R(E_{ij} - E_{ji}) \oplus i\R (E_{ij} + E_{ji})\right)\]

where \(\tilde \mft = \mft \cap \su(n)\) is the subalgebra of \(\su(n)\) of diagonal matrices. In particular, we have that (assuming the eigenvalues of \(A\) are distinct)

\[\T_A\mcO \cong \T_{[1]}\left(\frac{\SU(n)}{T^{n-1}}\right) \cong \frac{\su(n)}{\tilde t} = \bigoplus_{1 \le i < j \le n} \left(\R(E_{ij} - E_{ji}) \oplus i\R (E_{ij} + E_{ji})\right)\]

Then the almost complex structure is given by

\begin{align*}
    \R(E_{ij} - E_{ji}) &\mapsto i\R(E_{ij} + E_{ji}) \\
    i\R(E_{ij} + E_{ji}) &\mapsto i\R(E_{ij} - E_{ji})
\end{align*}

is

\begin{enumerate}
    \item The action of multiplication by \(i\) in \cref{eq:sln-root},
    \item The complex structure linking our expressions for \(\pi^*\omega\) and \(\pi^*h\).
\end{enumerate}

In particular, this means that the coadjoint orbits are K\"ahler.

\section{Another proof?}

\textbf{This section contains claims which I have not checked. But assuming the claims are true, this should give another construction for the K\"ahler structure on coadjoint orbits of \(\SU(n)\).}

The idea here is that we can make \(\T^*\SU(n)\) into a K\"ahler manifold, and if we define an appropriate moment map, then we can use K\"ahler reduction (see appendix) to get a K\"ahler structure on coadjoint orbits.

First of all, any cotangent bundle \(\TT^*M\) has a symplectic form, called the \emph{canonical symplectic form}

\[\omega = -\dd\alpha\]

where \(\alpha_p = (\dd\pi_p)^*\xi \in \TT_p^*M\), \(p = (x, \xi)\). In local cotangent coordinates,

\[\omega = \sum_i \dd x^i \wedge \dd \xi^i\]

See da Silva \S 2.3 for details.

In this case, \(\SU(n)\) is a Lie group, so we can left trivialise the cotangent bundle, via the map

\begin{align*}
    \TT^*\SU(n) &\to \SU(n) \times \su(n)^* \\
    (g, \alpha) &\mapsto (g, ((\dd\ell_g)_e)^*\alpha) 
\end{align*}

where \(\ell_g(x) = gx\) is the left multiplication map on \(\SU(n)\). Using the isomorphism \(\Phi : \su(n) \to \su(n)^*\) from above, we get an diffeomorphism

\[\T^* \SU(n) \cong \SU(n)\times \su(n)\]

Moreover, the polar decomposition

\begin{align*}
    \rm U(n) \times \mfu(n) &\to \GL(n, \C) \\
    (U, X) &\mapsto U \exp(X)
\end{align*}

Gives a diffeomorphism \(\SU(n) \times \su(n) \cong \SL(n, \C)\). Therefore, we also have a complex structure on \(\T^*\SU(n)\). 

\textbf{Unchecked claim: This is compatible with the symplectic structure, making \(\T^*\SU(n)\) into a K\"ahler manifold.}

Assuming the claim, let \(\SU(n)\) act on \(\SU(n) \times \su(n)^*\) by

\[g \cdot (h, \xi) = (h, \Ad_g^*(\xi))\]

and define a map \(\mu : \SU(n) \times \su(n)^*\) by \(\mu(h, \xi) = \xi\). Then \(\mu\) is equivariant, and

\textbf{Unchecked claim: \(\mu\) defines a moment map.}

Assuming this claim, then for \(\xi \in \su(n)^*\), \(\mu^{-1}(\xi)\) is a copy of \(\SU(n)\), and so if we perform K\"ahler reduction with momentum \(\xi\), then we get a K\"ahler structure on the coadjoint orbit

\[\frac{\SU(n)}{\Stab(\xi)} \cong \Orb(\xi)\]

\appendix

\section{Stabilisers of the adjoint action of \(\SU(n)\) and \(\SL(n, \C)\)}

This section has been copied from \texttt{coadjoint-orbits-sln.pdf}. It is included to provide some more details about the stabilisers of the adjoint action.

\subsection{Adjoint orbits of \(\SU(n)\)}

First of all, we note that

\[\su(n) = \left\{A \in \Mat(n, \C) \mid A^\dagger + A = 0, \tr(A) = 0\right\}\]

where \(A^\dagger\) is the conjugate transpose of \(A\). In particular, all elements of \(\su(n)\) are skew-hermitian, hence diagonalisable by an element of \(\SU(n)\)\footnote{From standard linear algebra arguments, we know that they are \(\rm U(n)\)-diagonalisable. But if \(PAP^{-1}\) is diagonal, then so is \((\lambda P)A(\lambda P)^{-1}\), and by choosing \(\lambda\) appropriately, \(\lambda \in \SU(n)\).}. With this, we can classify the coadjoint orbits based off a diagonal element in the orbit. Consider

\[A = \begin{pmatrix}
    i\lambda_1 I_{m_1} \\
    & \ddots \\
    & & i\lambda_k I_{m_k}
\end{pmatrix}\]

where \(\lambda_j \in \R\), with \(\lambda_1 > \lambda_2 > \dots > \lambda_k\), \(m_1 + \dots + m_k = n\) and \(m_1\lambda_1 + \dots + m_k\lambda_k = 0\). In this case, we have that the orbit is

\[\Orb(A) \cong \SU(n)/\Stab(A)\]

where \(\Stab(A)\) is the stabiliser of \(A\) under the adjoint action. In this case, we have that the stabiliser is the block diagonal subgroup

\[\Stab(A) = \rm S\left(\rm U(m_1) \times \dots \times \rm U(m_k)\right)\]

where we consider \(\rm U(m_1) \times \cdots \times \rm U(m_k) \le \rm U(n)\) as the block diagonal subgroup, and

\[\rm S\left(\rm U(m_1) \times \dots \times \rm U(m_k)\right) = \left(\rm U(m_1) \times \cdots \times \rm U(m_k)\right) \cap \SU(n)\]

the subgroup with determinant \(1\). Therefore, the coadjoint orbit is diffeomorphic to the flag manifold

\[\mcF(m_1, \dots, m_k) = \frac{\SU(n)}{\rm S\left(\rm U(m_1) \times \cdots \times \rm U(m_k)\right)}\]

In particular, note that \(\mcF(p, n-p)\) is diffeomorphic to the Grassmannian \(\Gr(p, n)\) of \(p\)-dimensional subspaces of \(\C^n\). More generally, a generic element of \(\mcF(m_1, \dots, m_k)\) can be represented as \((V_1, \dots, V_k)\), where \(V_j\) is a dimension \(m_j\) subspace of \(\C^n\), with \(V_j \perp V_k\) for all \(j \ne k\)\footnote{This may not be the usual definition of a (partial) flag manifold, which is a sequence of subspaces 

\[0 = W_0 \subset W_1 \subset \cdots \subset W_k = \C^n\]

However it is equivalent, since we can set \(W_0 = 0\), \(W_1 = V_1\), \(W_2 = V_1 \oplus V_2\) and so on. Moreover, the indexing is slightly different, since we are indexing using \(m_1, \dots, m_k\), instead of \(\dim(W_1) = m_1, \dim(W_2) = m_1 + m_2\) and so on. Again, it is easy to convert between the two definitions.}. This is because we can set \(V_j\) to be the \(i\lambda_j\) eigenspace and vice versa.

\subsection{Adjoint orbits of \(\SL(n, \C)\)}

In this case, we have that the Lie algebra is

\[\sl(n, \C) = \su(n)_\C = \left\{A \in \Mat(n, \C) \mid \tr(A) = 0\right\}\]

Define the Jordan block

\[J_n(\lambda) = \begin{pmatrix}
    \lambda & 1 \\
    & \ddots & \ddots \\
    & & \ddots & 1 \\
    & & & \lambda
\end{pmatrix} = \lambda I + J_n \in \Mat(n, \C)\]

where \(J_n = J_n(0)\). Then we know that every element of \(\sl(n, \C)\) is \(\SL(n, \C)\)-conjugate to a matrix in Jordan normal form, say

\[A = \begin{pmatrix}
    J_{m_1}(\lambda_1) \\
    & \ddots \\
    & & J_{m_k}(\lambda_k)
\end{pmatrix}\]

where \(m_1 + \dots + m_k = n\) and \(m_1\lambda_1 + \dots + m_k\lambda_k = 0\). As above, we would like to compute the stabiliser of \(A\) under the conjugation action.

\subsubsection{Diagonalisable case}

Suppose \(A\) was diagonalisable. By conjugating by a permutation matrix, we can assume that \(A\) is of the form

\[A = \begin{pmatrix}
    \lambda_1 I_{\ell_1} \\
    & \ddots \\
    & & \lambda_p I_{\ell_p}
\end{pmatrix}\]

with \(\lambda_1, \dots, \lambda_p\) distinct. In this case, we have that the stabiliser is

\[\Stab(A) = \rm S(\GL(\ell_1, \C) \times \cdots \times \GL(\ell_p, \C))\]

where we consider \(\GL(\ell_1, \C) \times \cdots \times \GL(\ell_p, \C) \le \GL(n, \C)\) as the block diagonal subgroup, and

\[\rm S(\GL(\ell_1, \C) \times \cdots \times \GL(\ell_p, \C)) = \left(\GL(\ell_1, \C) \times \cdots \times \GL(\ell_p, \C)\right) \cap \SL(n, \C)\]

In this case, we can compute the coadjoint orbits as well. In particular, a generic element \(B \in \Orb(A)\) is determined by its eigenspaces. One way to express this is as an open subset of

\[\Gr(\ell_1, n) \times \cdots \times \Gr(\ell_p, n)\]

where we require the subspaces to intersect trivially\footnote{Which is \emph{not} a flag manifold if we use the same trick as above, considering the sums of the eigenspaces, since without orthogonality we can't recover the eigenspaces from the sums. This is also not the same as the product

\[\Gr(\ell_1, n) \times \Gr(\ell_2, n - \ell_1) \times \cdots \times \Gr(\ell_p, \ell_p)\]

since we don't require the eigenspaces to be orthogonal.}. In the case where \(\ell_i = 1\), so the eigenvalues of \(A\) are distinct, then the orbit of \(A\) is precisely the space of all matrices with the same characteristic polynomial, i.e.

\[\chi_B(t) = (t - \lambda_1)\cdots(t - \lambda_n)\]

Which shows that \(\Orb(A)\) is a closed subvariety of \(\Mat(n, \C)\). Moreover, in this case, we get that the stabiliser is the diagonal subgroup

\[\Stab(A) = \left\{\left.\begin{pmatrix}
    b_1 & \\
    & \ddots \\
    & & b_n
\end{pmatrix}\ \right\vert\ b_j \in \C^*, b_1\cdots b_n = 1 \right\} \cong (\C^*)^{n-1}\]

Now notice that

\[\begin{pmatrix}
    \vert & & \vert \\
    v_1 & \cdots & v_n \\
    \vert & & \vert
\end{pmatrix}\begin{pmatrix}
    b_1 & \\
    & \ddots \\
    & & b_n
\end{pmatrix} = \begin{pmatrix}
    \vert & & \vert \\
    b_1v_1 & \cdots & b_nv_n \\
    \vert & & \vert
\end{pmatrix}\]

Which is another way of seeing that the orbit space in this case is an open subset of

\[(\C \bb P^{n-1})^n\]

since each \(v_j\) determines is a line in \(\C^n\), and we get a well defined function \(\SL(n, \C)/(\C^*)^{n-1} \to (\C \bb P^{n-1})^n\), which is injective.

\subsubsection{Jordan block}

Now suppose \(A = J_n(\lambda)\). Since \(\tr(A) = 0\), we must have \(\lambda = 0\). In this case, it is easy to show that the stabiliser is the subgroup

\[P = \left\{\left.\begin{pmatrix}
    a_1 & a_2 & \cdots & \cdots & a_n \\
    & a_1 & a_2 & &\ddots & \vdots \\
    & & \ddots & \ddots & \vdots \\
    & & & a_1 & a_2 \\
    & & & & a_1
\end{pmatrix}\ \right\vert\ a_1^n = 1, a_j \in \C \right\}\]

of upper triangular Toeplitz matrices. In fact, we prove something more general in the next subsection. In this case, the orbit of \(J_n\) is

\[\Orb(J_n) = \left\{B \in \Mat(n, \C) \mid B^n = 0, B^{n-1} \ne 0\right\}\]

which is an open subset of a closed subvariety.

\subsubsection{General case}

In general, for \(B \in \Mat(n, \C)\), the equation \(BA = AB\) becomes equations of the form

\[J_\ell(\lambda)X = XJ_m(\mu)\]

for some \(X \in \Mat(\ell \times m, \C)\). Assume without loss of generality that \(\ell \ge m\). Then note that

\[J_\ell(\lambda - \mu)X = J_\ell(\lambda)X - \mu X = XJ_m(\mu) - \mu X = XJ_m(0) = X J_m\]

\begin{lemma*}
    Let \(X\) be a \(m \times n\) matrix, with \(J_mX = XJ_n\). Then \(X\) is upper triangular and Toeplitz, and if \(m < n\), then the first \(n - m\) columns of \(X\) are zero.
\end{lemma*}

\begin{proof}
    Say the entries of \(X\) are \((x_{ij})\). Then

    \[x_{i+1, j+1} = e_i^\T J_m Xe_{j+1} = e_i^\T X J_n e_{j+1} = x_{i, j}\]

    and so \(X\) is Toeplitz. Moreover,

    \[J_m X e_1 = X J_n e_1 = 0\]

    So \(x_{i, 1} = 0\) for all \(i > 1\). Similarly, we have that

    \[J_m^\ell X e_\ell = X J_n^\ell e_\ell = 0\]

    hence \(x_{i, j} = 0\) for \(i > j\). Therefore, \(X\) is upper triangular. Now suppose \(m < n\). We will show that the all the entries on the last row except the last one is zero, since this will show the first \(n - m\) columns are zero, and that the Toeplitz part is strictly upper triangular. This follows from the fact that

    \[x_{m, n-\ell} = e_m^\T X e_{n-\ell} =  e_m^\T X J_n^\ell e_n = e_m^\T J_m^\ell X e_n = 0\]
\end{proof}

This means that if \(\lambda = \mu\), then \(X\) must satisfy the conclusions of the lemma. On the other hand, if \(\lambda \ne \mu\), then we have that

\[J_\ell(\lambda - \mu)^n X = X{J_m}^m = 0\]

and so \(X = 0\).

Therefore, we can write a generic matrix \(B\) with \(BA = AB\) blockwise as \(B_{ij}\), where \(B_{ij} \in \Mat(m_i \times m_j, \C)\), corresponding to \(J_{m_i}(\lambda_i)\) and \(J_{m_j}(\lambda_j)\). In particular, we have that

\[B_{ij} = \begin{cases}
    0 & \text{if }\lambda_j \ne \lambda_i \\
    \begin{pmatrix}
        T_{ij} \\ 0
    \end{pmatrix} & \text{if } \lambda_j = \lambda_i \text{ and } m_i > m_j \\
    \begin{pmatrix}
        0 & T_{ij}
    \end{pmatrix} & \text{if } \lambda_j = \lambda_i \text{ and } m_i < m_j \\
    T_{ij} & \text{if } \lambda_j = \lambda_i \text{ and } m_i = m_j
\end{cases}\]

where \(T_{ij}\) is a strictly upper triangular Toeplitz matrix, of size \(\min(m_i, m_j)\). Therefore, the stabiliser of \(A\) is the subgroup of \(\SL(n, \C)\) of the above form.

\subsection{Nilporent orbits of \(\SL(n, \C)\) for small \(n\)}

Throughout, we will study \emph{nonzero} nilpotent orbits. In particular, all eigenvalues are zero. Also note that

\[\dim(\SL(n, \C)) = n^2 - 1\]

For \(n = 2\), we only have one nilpotent orbit, given by the Jordan normal form

\[A = \begin{pmatrix}
    0 & 1 \\ 0 & 0
\end{pmatrix}\]

and the stabiliser is the subgroup elements of \(\SL(2, \C)\) of the form

\[\pm\begin{pmatrix}
    1 & b \\ 0 & 1
\end{pmatrix}\]

The complex dimension of the quotient is \(3 - 1 = 2\).

For \(n = 3\), the possible Jordan normal forms are

\[A_1 = \begin{pmatrix}
    0 & 1 & 0 \\
    0 & 0 & 1 \\
    0 & 0 & 0
\end{pmatrix} \quad A_2 = \begin{pmatrix}
    0 & 1 & 0 \\
    0 & 0 & 0 \\
    0 & 0 & 0
\end{pmatrix}\]

and the corresponding stabilisers are

\[\begin{pmatrix}
    a & b & c \\
    0 & a & b \\
    0 & 0 & a
\end{pmatrix} \qqtext{and} \begin{pmatrix}
    a & b & c \\
    0 & a & 0 \\
    0 & d & e
\end{pmatrix}\]

The complex dimension of the orbit spaces are \(8 - 2 = 6\) and \(8 - 4 = 4\) respectively.

For \(n=4\), we get Jordan decompositions \([4], [3, 1], [2, 2], [2, 1, 1]\). The corresponding stabilisers are

\[\begin{pmatrix}
    a & b & c & d \\
    0 & a & b & c \\
    0 & 0 & a & b \\
    0 & 0 & 0 & a
\end{pmatrix} \quad \begin{pmatrix}
    a & b & c & d \\
    0 & a & b & 0 \\
    0 & 0 & a & 0 \\
    0 & 0 & e & f
\end{pmatrix} \quad \begin{pmatrix}
    a & b & c & d \\
    0 & a & 0 & c \\
    e & f & g & h \\
    0 & e & 0 & g
\end{pmatrix} \quad \begin{pmatrix}
    a & b & c & d \\
    0 & a & 0 & 0 \\
    0 & e & f & g \\
    0 & h & i & j
\end{pmatrix}\]

The complex dimension of the orbit spaces are \(15 - 3 = 12\), \(15 - 5 = 10\), \(15 - 5 = 10\) and \(15 - 9 = 6\) respectively.

Note in each case, the complex dimension is even, which is a necessary condition for there to be a hyperk\"ahler structure on the orbit space.

\section{K\"ahler reduction}

This section is copied from \texttt{kahler-reduction.tex}. It is included for completeness as the ``proof'' from \S 8 uses the K\"ahler reduction.

Throughout,

\begin{enumerate}
    \item \((M, \omega, g, I)\) is a K\"ahler manifold,
    \item \(G\) is a compact Lie group acting on \(M\),
    \item \((M, \omega, G, \mu)\) is a Hamiltonian \(G\)-space,
    \item \(G\) acts by biholomorphisms on \(M\),
    \item \(G\) acts freely on \(\mu^{-1}(0)\).
\end{enumerate}

In particular, as \(\omega(u, v) = g(I(u), v)\), \(I\) is an isometry on \(M\), \(G\) acts by isometries on \(M\). Let \(Z = \mu^{-1}(0)/G\). Let

% https://q.uiver.app/#q=WzAsMyxbMCwwLCJcXG11XnstMX0oMCkiXSxbMCwyLCJaPVxcbXVeey0xfSgwKS9HIl0sWzIsMCwiTSJdLFswLDIsImkiLDAseyJzdHlsZSI6eyJ0YWlsIjp7Im5hbWUiOiJob29rIiwic2lkZSI6InRvcCJ9fX1dLFswLDEsIlxccGkiLDIseyJzdHlsZSI6eyJoZWFkIjp7Im5hbWUiOiJlcGkifX19XV0=
\[\begin{tikzcd}
	{\mu^{-1}(0)} && M \\
	\\
	{Z=\mu^{-1}(0)/G}
	\arrow["i", hook, from=1-1, to=1-3]
	\arrow["\pi"', two heads, from=1-1, to=3-1]
\end{tikzcd}\]

be the natural inclusion and quotient maps. In particular, note that \(\pi\) is a surjective submersion. Therefore, any tensor \(\alpha\) of type \((0, r)\) on \(Z\) is determined by its pullback \(\pi^*\alpha\).

The Marsden-Weinstein reduction theorem from symplectic geometry states that there exists a symplectic form \(\tilde\omega\) on \(Z\), such that

\[\pi^*\tilde\omega = i^*\omega\]

We will now construct the almost complex structure and Riemannian metric on \(Z\).

Since \(\pi\) is a submersion, for \(p \in \mu^{-1}(0)\), \(z = \pi(p)\), we have

\[\dd\pi_p : \TT_p\mu^{-1}(0) \surj \TT_zZ\]

Let \(V_p = \ker(\dd\pi_p)\) be the vertical bundle, and \(H_p = V_p^\perp \le \TT_p\mu^{-1}(0)\) be the horizontal bundle. Therefore, we have an isomorphism

\[\dd\pi_p\vert_{H_p} : H_p \cong \TT_zZ\]

For brevity, we write this isomorphism as

\begin{align*}
    \dd\pi_p\vert_{H_p} : H_p &\to \TT_zZ \\
    v &\mapsto v_* \\
    w^* &\mapsfrom w
\end{align*}

With this, we can see that

\[\tilde\omega(u, v) = \omega(u^*, v^*)\]

and that

\[\tilde g(u, v) = g(u^*, v^*)\]

defines a Riemannian metric on \(Z\). Therefore, the almost complex structure we want must be given by

\[\tilde I(u) = I(u^*)_*\]

Assuming this is well defined, then we have that

\begin{align*}
    \tilde\omega(u, v) &= \omega(u^*, v^*) \\
    &= g(I(u^*), v^*) \\
    &= g(\tilde I(u)^*, v^*) \\
    &= \tilde g(\tilde I(u), v)
\end{align*}

so \((\tilde\omega, \tilde g, \tilde I)\) is a compatible triple.

\begin{lemma*}
    \(I\) restricts to a map \(H_p \to H_p\).
\end{lemma*}

\begin{proof}
    Let \(N_p = \left(\TT_p\mu^{-1}(0)\right)^\perp \le \TT_pM\) be the normal bundle of \(\mu^{-1}(0) \subseteq M\). This gives us an orthogonal direct sum

    \[\TT_p M = N_p \oplus V_p \oplus H_p\]

    Fix \(X \in \mfg\). Then for \(v \in \TT_pM\),

    \[g(\gr(\mu^X), v) = \dd\mu^X(v) = \omega(X^\#, v) = g(I(X^\#), v)\]

    where \(\gr(f)\) is the \(g\)-dual of \(\dd f\). In particular, this means that \(\gr(\mu^X) = I(X^\#)\). Let \(X_1, \dots, X_k\) be a basis  of \(\mfg\), with corresponding dual basis \(\xi^1, \dots, \xi^k\). Then the moment map can be written as

    \[\mu(p) = \mu^{X_1}(p)\xi^1 + \dots + \mu^{X_k}(p)\xi^k\]

    But this means that

    \[\left\{\gr(\mu^{X_1}), \dots, \gr(\mu^{X_k})\right\} = \left\{I(X_1^\#), \dots, I(X_k^\#)\right\}\]

    is a basis of \(N_p\). As \(X_1^\#, \dots, X_k^\#\) is a basis for \(V_p\), we have that \(I\) restricts to a map \(N_p \oplus V_p \to N_p \oplus V_p\). By orthogonality, this means that \(I\) restricts to a map \(H_p \to H_p\).
\end{proof}

Therefore, the map \(\tilde I\) as above is well defined. Finally, we need to show that we have a K\"ahler structure. That is, \(I\) is integrable.

\begin{lemma*}
    Let \(M\) be a manifold, \((\omega, g, I)\) a compatible triple on \(M\). Then \((M, \omega, g, I)\) is a K\"ahler manifold if and only if \(\nabla I = 0\), where \(\nabla\) is the Levi-Civita connection induced by \(g\).

    Moreover, we have the expression

    \[\nabla I(u) = \nabla(I(u)) - I(\nabla u)\]

    and so \(\nabla I = 0\) if and only if \(\nabla(I(u)) = I(\nabla u)\) for all vector fields \(u\).
\end{lemma*}

\begin{proof}
    See Huybrechts, \S 4.A. for the first part. For the second part, see Nicolaescu page 96.
\end{proof}

\begin{lemma*}
    The Levi-Civita connection induced by \(\tilde g\) is

    \[\tilde\nabla_XY = \pr_H\left(\nabla_{X^*}Y^*\right)_*\]

    for vector fields \(X, Y\) on \(Z\), and we extend \(X^*, Y^*\) arbitrarily to a neighbourhood of \(\mu^{-1}(0) \subseteq M\). In addition, \(\pr_H : \TT_p M \to H_p\) is the orthogonal projection.
\end{lemma*}

\begin{proof}
    Omitted.
\end{proof}

Finally, we note that since \(I\) respects the orthogonal decomposition

\[\TT_pM = \left(N_p \oplus V_p\right) \oplus H_p\]

\(\pr_H\) and \(I\) commute. With this, we can now compute \(\tilde\nabla\tilde I\).


\begin{align*}
    \left(\tilde\nabla_X\tilde I(Y)\right)^* &= \pr_H\left(\nabla_{X^*}\tilde I(Y)^*)\right) \\
    &= \pr_H\left(\nabla_{X^*}I(Y^*)\right)\\
    &= \pr_H\left(I(\nabla_{X^*}Y^*)\right) \\
    &= I\left(\pr_H(\nabla_{X^*}Y^*)\right) \\
    &= \tilde I(\tilde\nabla_XY)^*
\end{align*}

Hence we have that

\[\tilde\nabla_X\tilde I(Y) = \tilde I(\tilde\nabla_XY)\]

for any vector fields \(X, Y\) on \(Z\), and so \(\tilde\nabla\tilde I = 0\), and \((Z, \tilde\omega, \tilde g, \tilde I)\) is a K\"ahler manifold.

\end{document}
