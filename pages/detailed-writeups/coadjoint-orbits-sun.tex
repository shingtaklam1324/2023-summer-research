\documentclass{article}

\usepackage{../../Style}
\usepackage{stmaryrd}

\DeclareMathOperator{\SU}{SU}
\newcommand{\su}{\mathfrak{su}}
\renewcommand{\sl}{\mathfrak{sl}}
% \newcommand{\rm}[1]{\mathrm{#1}}
\renewcommand{\tilde}{\widetilde}
\newcommand{\surj}{\twoheadrightarrow}
\DeclareMathOperator{\gr}{grad}
\DeclareMathOperator{\Gr}{Gr}

\newcommand{\iinner}[1]{\left\langle\!\left\langle #1 \right\rangle\!\right\rangle}

\usepackage{biblatex}
\addbibresource{../../bibliography.bib}

\title{Coadjoint Orbits of \(\SU(n)\)}
\author{Shing Tak Lam}

\begin{document}

\maketitle

In this note, we will consider the coadjoint orbits of \(\SU(n)\), and show that they are K\"ahler manifolds.

\tableofcontents

\section*{Notation}

Throughout this document, we will use the following notation:

\begin{itemize}
    \item \(\SU(n)\) is the group of \(n \times n\) unitary matrices with determinant \(1\).
    \item \(\su(n)\) is its Lie algebra.
    \item \(\Ad : \SU(n) \to \GL(\su(n))\) is the adjoint representation.
    \item \(\Ad^* : \SU(n) \to \GL(\su(n)^*)\) the coadjoint representaton.
    \item \(\ad : \su(n) \to \gl(\su(n))\) the adjoint representation, given by \(\ad_X(Y) = [X, Y]\).
    \item \(M\) will be a (co)adjoint orbit, \(A \in M\) a diagonal element.
    \item \(\inner{\cdot, \cdot}\) will denote both the pairing \(\su(n)^* \times \su(n) \to \R\) and the inner product on \(\su(n)\),
    \item \(\Phi : \su(n) \to \su(n)^*\) is the isomorphism induced by the inner product,
    \item \(\ell_g(h) = gh\) is the left multiplication by \(g\) map,
    \item \(\iinner{\cdot, \cdot}\) the Riemannian metric on the (co)adjoint orbit.
\end{itemize}

\section{Adjoint and Coadjoint Orbits}

Define the Lie algebra

\begin{equation}
    \label{eq:sun-def}
    \su(n) = \left\{X \in \Mat(n, \C) \mid X^* + X = 0, \tr(X) = 0\right\}
\end{equation}

where \(X^*\) is the conjugate transpose of \(X\), and with the Lie bracket being the matrix commutator. We can define the adjoint representation of \(\SU(n)\) as

\begin{align*}
    \Ad : \SU(n) &\to \GL(\su(n)) \\
    \Ad_g(X) &= gXg^*
\end{align*}

Taking the dual representation, we get the coadjoint representation, which is

\begin{align*}
    \Ad^* : \SU(n) &\to \GL(\su(n)^*) \\
    \Ad^*_g(\beta)(X) &= \inner{\beta, \Ad_{g^{-1}}(X)}
\end{align*}

where \(\inner{\cdot, \cdot}\) is used here to denote the pairing \(\su(n)^* \times \su(n) \to \R\). We will use the same notation for the inner product on \(\su(n)\), which should not be an issue as the inner product defines a natural isomorphism. Now note that \(-\kappa\), where \(\kappa\) is the Killing form, defines an inner product

\[\inner{X, Y} = -\tr(XY) = \tr(XY^*)\]

defines an inner product on \(\su(n)\)\footnote{In fact, \(\inner{A, B} = \tr(AB^*)\) defines a Hermitian inner product on the space of complex matrices.}, which means that we have a natural isomorphism

\begin{align*}
    \Phi : \su(n) &\to \su(n)^* \\
    X &\mapsto \inner{X, \cdot}
\end{align*}

With this, suppose \(\beta = \Phi(B)\), then

\[\Ad_g^*(\beta)(X) = \inner{B, \Ad_{g^{-1}}(X)} = -\tr(Bg^{-1}Xg) = -\tr(BAg^{-1}X) = \Phi(\Ad_g(B))(X)\]

Therefore, the following diagram commutes

% https://q.uiver.app/#q=WzAsNCxbMCwwLCJcXHN1KG4pIl0sWzIsMCwiXFxzdShuKSJdLFswLDIsIlxcc3UobileKiJdLFsyLDIsIlxcc3UobileKiJdLFsyLDMsIlxcQWRfZ14qIl0sWzAsMSwiXFxBZF9nIl0sWzAsMiwiXFxQaGkiLDJdLFsxLDMsIlxcUGhpIl1d
\[\begin{tikzcd}
	{\su(n)} && {\su(n)} \\
	\\
	{\su(n)^*} && {\su(n)^*}
	\arrow["{\Ad_g^*}", from=3-1, to=3-3]
	\arrow["{\Ad_g}", from=1-1, to=1-3]
	\arrow["\Phi"', from=1-1, to=3-1]
	\arrow["\Phi", from=1-3, to=3-3]
\end{tikzcd}\]

or equivalently, \(\Phi\) defines an isomorphism of representations between \(\Ad\) and \(\Ad^*\).

\section{Root decomposition}

Consider the Lie algebra \(\sl(n, \C)\) of trace free \(n \times n\) complex matrices. Then we have the Cartan subalgebra \(\mft\) of diagonal matrices. Let \(E_{ij}\) be the standard basis matrices for \(\Mat(n, \C)\), \(B \in \mft\). Say

\[B = \begin{pmatrix}
    b_1 \\
    & \ddots \\
    & & b_n
\end{pmatrix}\]

Then \([B, E_{ij}] = (b_i - b_j)E_{ij}\). This means that we have the eigendecomposition

\begin{equation}
    \label{eq:sln-root}
    \sl(n, \C) = \mft \oplus \bigoplus_{1 \le i, j \le n, i \ne j} \C E_{ij}
\end{equation}

In particular, if we restrict this to the subalgebra \(\su(n)\), we get the decomposition

\begin{equation}
    \label{eq:sun-root}
    \su(n) = \tilde\mft \oplus \bigoplus_{1 \le i < j \le n} \left(\R(E_{ij} - E_{ji}) \oplus i\R (E_{ij} + E_{ji})\right)
\end{equation}

where \(\tilde \mft = \mft \cap \su(n)\) is the subalgebra of \(\su(n)\) of diagonal matrices. 

\section{Tangent space and Diagonalisation}

\label{sec:tangent}

\subsection{Diagonalisation and Stabilisers of the coadjoint action}

First of all, we note that elements of \(\su(n)\) are skew-hermitian, hence diagonalisable by an element of \(\SU(n)\)\footnote{From standard linear algebra arguments, we know that they are \(\mathrm U(n)\)-diagonalisable. But if \(PBP^{-1}\) is diagonal, then so is \((\lambda P)B(\lambda P)^{-1}\), and by choosing \(\lambda\) appropriately, \(\lambda \in \SU(n)\).}. With this, we can classify the coadjoint orbits based off a diagonal element in the orbit. Consider

\[A = \begin{pmatrix}
    i\lambda_1 I_{m_1} \\
    & \ddots \\
    & & i\lambda_k I_{m_k}
\end{pmatrix}\]

where \(I_m\) is the \(m \times m\) identity matrix, \(\lambda_j \in \R\), with \(\lambda_1 < \lambda_2 < \dots < \lambda_k\), \(m_1 + \dots + m_k = n\) and \(m_1\lambda_1 + \dots + m_k\lambda_k = 0\). In this case, we have that the orbit is

\[\Orb(A) \cong \SU(n)/\Stab(A)\]

where \(\Stab(A)\) is the stabiliser of \(A\) under the adjoint action. In this case, we have that the stabiliser is the block diagonal subgroup

\[\Stab(A) = \rm S\left(\rm U(m_1) \times \dots \times \rm U(m_k)\right)\]

where we consider \(\rm U(m_1) \times \cdots \times \rm U(m_k) \le \SU(n)\) as the block diagonal subgroup, and

\[\rm S\left(\rm U(m_1) \times \dots \times \rm U(m_k)\right) = \left(\rm U(m_1) \times \cdots \times \rm U(m_k)\right) \cap \SU(n)\]

the subgroup with determinant \(1\).

\subsection{Tangent space}

Let \(M\) be an adjoint orbit. We will now focus on the generic case, that is, the eigenvalues of \(A\) are distinct. In this case, we have that

\[\Stab(A) = T^{n-1}\]

is the torus of diagonal matrices in \(\SU(n)\), and we have a diffeomorphism \(M \simeq \SU(n)/T\). Therefore, the tangent space at \(A\) is

\[\TT_AM = \frac{\su(n)}{\tilde \mft} = \bigoplus_{1 \le i < j \le n} \left(\R(E_{ij} - E_{ji}) \oplus i\R (E_{ij} + E_{ji})\right) = \{\ad_A(X) \mid X \in \su(n)\}\]

where the last equality follows from the fact that

\[\ad_A(E_{ij}) = i(\lambda_i - \lambda_j)E_{ij}\]

and that diagonal matrices commute. Next, for \(g \in \SU(n)\), \(\Ad_g : \su(n) \to \su(n)\) is an invertible linear map, hence a diffeomorphism. In particular, this means that if \(B = \Ad_g(A)\), then

\begin{equation}
    \label{eq:tangent-space}
    \TT_B M = \Ad_g(\TT_A M) = \left\{\Ad_g(\ad_A(X)) \mid X \in \su(n)\right\} = \left\{\ad_B(X) \mid X \in \su(n)\right\}
\end{equation}

using the fact that

\[\ad_{\Ad_g(A)} = \Ad_g \circ \ad_A \circ \Ad_{g^*}\]

which we will show in the proof of \cref{lem:ad_Ad}, and that \(\Ad_{g^*}\) is a bijection. Moreover, if \(\beta= \Phi(B)\), then

\begin{align*}
    \inner{(\ad_X)^*(\beta), Y} &= \inner{\beta, \ad_X(Y)} \\
    &= \inner{B, [X, Y]} \\
    &= -\tr(BXY - BYX) \\
    &= -\tr(BXY - XBY) \\
    &= \inner{[B, X], Y} \\
    &= \inner{-\ad_X(B), Y}
\end{align*}

Therefore, we have that the following diagram commutes.

% https://q.uiver.app/#q=WzAsNCxbMCwwLCJcXHN1KG4pIl0sWzIsMCwiXFxzdShuKSJdLFswLDIsIlxcc3UobileKiJdLFsyLDIsIlxcc3UobileKiJdLFswLDIsIlxcUGhpIiwyXSxbMSwzLCJcXFBoaSJdLFswLDEsIlxcYWRfWCJdLFsyLDMsIlxcYWRfWF4qIiwyXV0=
\[\begin{tikzcd}
	{\su(n)} && {\su(n)} \\
	\\
	{\su(n)^*} && {\su(n)^*}
	\arrow["\Phi"', from=1-1, to=3-1]
	\arrow["\Phi", from=1-3, to=3-3]
	\arrow["{-\ad_X}", from=1-1, to=1-3]
	\arrow["{\ad_X^*}"', from=3-1, to=3-3]
\end{tikzcd}\]

Thus, we have the tangent space to the corresponding coadjoint orbit is

\[T_B \tilde M = \left\{\ad_X^*(B) \mid X \in \su(n)\right\}\]

\section{Kirillov-Kostant-Souriau symplectic form}

\label{sec:kks}

This section is from \cite{marsden_ratiu} Chapter 14. The proof is slightly modified, since here we have an explicit isomorphism between the adjoint and coadjoint representation, which simplifies some of the arguments.

\begin{theorem}
    \label{thm:kks}
    [Kirillov-Kostant-Souriau, \cite[Theorem 14.4.1]{marsden_ratiu}]
    Let \(M \subseteq \su(n)^*\) be a coadjoint orbit. Define the \(2\)-form \(\omega\) on \( M\) by

    \[\omega_\mu(\ad_X^*(\mu), \ad_Y^*(\mu)) = -\inner{\mu, [X, Y]}\]

    Then \(\omega\) is a symplectic form on \( M\).
\end{theorem}

\subsection{\(\omega\) is well defined}

First of all, we show that \(\omega\) is well defined. That is, it is independent of the choice of \(X, Y \in \su(n)\).

Suppose \(Z \in \su(n)\) is such that \(\ad_Z^*(\mu) = \ad_X^*(\mu)\). Then we must have that

\[\inner{\mu, [X, Y]} = \inner{\mu, [Z, Y]}\]

for all \(Y \in \su(n)\).

\subsection{\(\omega\) is non-degenerate}

Suppose we have \(X \in \su(n)\) such that

\[\omega_\mu(\ad_X^*(\mu), \ad_Y^*(\mu)) = \inner{\mu, [X, Y]} = 0\]

for all \(Y\). But this is the same as \(\ad_X^*(\mu) = 0\). Therefore, \(\omega\) is non-degenerate.

\subsection{\(\omega\) is closed}

First of all, we will need some preliminary results.

\begin{lemma}
    \label{lem:ad_Ad}
    \[\ad_{\Ad_gX}^* = \Ad_g^* \circ \ad_X^* \circ \Ad_{g^*}^*\]
\end{lemma}

\begin{proof}
    We will prove the corresponding statement for \(\Ad\) and \(\ad\), and the result will follow by conjugation with the isomorphism \(\Phi\).

    \[\Ad_g \circ \ad_X \circ \Ad_{g^*}(Y)= gX g^* Y g g^* - g g^* Y g X g^* = [gX g^*, Y] = \ad_{\Ad_gX}(Y)\]
\end{proof}

\begin{lemma}
    \[\Ad_g([X, Y]) = [\Ad_g(X), \Ad_g(Y)]\]
\end{lemma}

\begin{proof}
    Expand using the definition of \(\Ad\).
\end{proof}

\begin{lemma}
    \(\Ad_g^* :  M \to  M\) preserves \(\omega\), that is,

    \[(\Ad_g^*)^*\omega = \omega\]
\end{lemma}

\begin{proof}
    Evaluating \(\ad^*_{\Ad_gX} = \Ad_g^* \circ \ad_X^* \circ \Ad_{g^{-1}}^*\) at \(\nu = \Ad_g^*(\mu)\), we get

    \[\ad^*_{\Ad_gX}(\nu) = \Ad_g^* \circ \ad_X^*(\mu) = \dd_\mu\Ad_g^* \circ \ad_X^*(\mu)\]

    Therefore,

    \begin{align*}
        ((\Ad_g^*)^*\omega)_\mu(\ad_X^*(\mu), \ad_Y^*(\mu)) &= \omega_\nu(\dd_\mu \Ad_g^* \cdot \ad_X^*(\mu), \dd_\mu \Ad_g^* \cdot \ad_Y^*(\mu)) \\
        &= \omega_\nu(\ad^*_{\Ad_gX}(\nu), \ad^*_{\Ad_gY}(\nu)) \\
        &= -\inner{\nu, [\Ad_gX, \Ad_gY]} \\
        &= -\inner{\nu, \Ad_g([X, Y])} \\
        &= -\inner{\Ad_{g^{-1}}^*(\nu), [X, Y]} \\
        &= -\inner{\mu, [X, Y]} \\
        &= \omega_\mu(\ad_X^*(\mu), \ad_Y^*(\mu))
    \end{align*}
\end{proof}

For \(\nu \in \su(n)^*\), define the left-invariant one-form

\[\nu_\ell(g) = (\dd_g\ell_{g^{-1}})^*(\nu)\]

for \(g \in \SU(n)\). Similarly, for \(X \in \su(n)\), let \(X_\ell\) be the corresponding left invariant vector field on \(G\). Then \(\nu_\ell(X_\ell) = \inner{\nu, X}\) at all \(g \in \SU(n)\).

Fix \(\nu \in  M\), and consider the map \(\pi : \SU(n) \to  M\), defined by

\[\pi(g) = \Ad_g^*(\nu)\]

We can use this to pullback \(\sigma = \pi^*\omega\) to a two form on \(\SU(n)\).

\begin{lemma}
    \label{lem:left-inv}
    \(\sigma\) is left invariant. That is, \(\ell_g^*\sigma = \sigma\) for all \(g \in \SU(n)\).
\end{lemma}

\begin{proof}
    First, notice that \(\pi \circ \ell_g = \Ad_g^* \circ \pi\), since

    \[\pi(\ell_g(h)) = \Ad_{gh}^*(\nu) = \Ad_g^*\circ \Ad_h^*(\nu) = \Ad_g^*(\pi(h))\]

    With this,

    \[\ell_g^*\sigma = \ell_g^*\pi^*\omega = (\pi\circ\ell_g)^*\omega = (\Ad_g^* \circ \pi)^*\omega = \pi^*(\Ad_g^*)^*\omega = \pi^*\omega = \sigma\]
\end{proof}

\begin{lemma}
    \(\sigma(X_\ell, Y_\ell) = -\inner{\nu_\ell, [X_\ell, Y_\ell]}\).
\end{lemma}

\begin{proof}
    By left invariance of both sides, suffices to show that the result holds at \(e\). First notice that

    \[\dd_I\pi(Y) = -\ad_Y^*(\nu)\]

    Therefore, \(\pi\) is a submersion at \(e\). By definition of the pullback,

    \begin{align*}
        \sigma_I(X, Y) &= (\pi^*\omega)_I(X, Y)\\ 
        &= \omega_{\pi(I)}(\dd_e\pi \cdot X, \dd_e\pi\cdot Y) \\
        &= \omega_\nu(\ad_X^*(\nu), \ad_Y^*(\nu)) \\
        &= -\inner{\nu, [X, Y]}
    \end{align*}

    Hence

    \[\sigma(X_\ell, Y_\ell)_I = \sigma_I(X, Y) = -\inner{\nu, [X, Y]} = -\inner{\nu_\ell, [X_\ell, Y_\ell]}_I\]
\end{proof}

Now for a one form \(\alpha\), we have that

\[\dd\alpha(X, Y) = X[\alpha(Y)] - Y[\alpha(X)] - \alpha([X, Y])\]

where for a smooth function \(f : M \to \R\), and a vector field \(X\) on \(M\), \(X[f] := \dd f(X)\) is a smooth function \(M \to \R\).

Since \(\nu_\ell(X_\ell)\) is constant, \(Y_\ell[\nu_\ell(X_\ell)] = 0\). Similarly, \(X_\ell[\nu_\ell(Y_\ell)] = 0\). Therefore, we have that

\[\dd \nu_\ell(X_\ell, Y_\ell) = -\nu_\ell([X_\ell, Y_\ell]) = \sigma(X_\ell, Y_\ell)\]

Now suppose \(U, V\) are vector fields on \(\SU(n)\). We want to show that \(\sigma(U, V) = \dd\nu_\ell(U, V)\). As \(\sigma\) is left invariant,

\begin{align*}
    \sigma(U, V)_g &= (\ell_{g^{-1}}^* \sigma)_g(U_g, V_g) \\
    &= \sigma_I(\underbrace{\dd \ell_{g^{-1}} \cdot U_g}_{=:X}, \underbrace{\dd \ell_{g^{-1}} \cdot V_g}_{=:Y}) \\
    &= \sigma_I(X, Y) \\
    &= \dd\nu_\ell(X_\ell, Y_\ell)_I \\
    &= (\ell_g^*\dd\nu_\ell)(X_\ell, Y_\ell)_I \\
    &= (\dd\nu_\ell)_g(\dd\ell_g(X_\ell)_I, \dd\ell_g(Y_\ell)_I) \\
    &= (\dd\nu_\ell)_g(\dd \ell_g X, \dd\ell_g Y) \\
    &= (\dd\nu_\ell)_g(U_g, V_g) \\
    &= \dd\nu_\ell(U, V)_g
\end{align*}

With this, \(\dd\sigma = \dd^2\nu_\ell = 0\). Hence \(\pi^*\dd\omega = \dd(\pi^*\omega) = \dd\sigma = 0\). Since \(\pi \circ \ell_g = \Ad_g^* \circ \ell_g\), and \(\pi\) is a submersion at \(I\), it is in fact a submersion everywhere. Moreover, \(\pi\) is surjective, by definition.

For \(\mu \in  M\), and \(X, Y \in T_\mu  M\), we have that

\[\dd\omega_\mu(X, Y, Z) = \dd\omega_{\pi(g)}(\dd\pi(U), \dd\pi(V), \dd\pi(W)) = (\pi^*\dd\omega)_g(U, V, W) = 0\]

where \(g \in \SU(n)\) is such that \(\pi(g) = \mu\), which exists by surjectivity, and \(U, V, W \in T_g\SU(n)\) such that \(\dd\pi(U) = X, \dd\pi(V) = Y\) and \(\dd\pi(W) = Z\), which exists as \(\pi\) is a submersion. Thus, as \(\mu \in  M\) is arbitrary, \(\omega\) is closed.

\subsection{\(\omega\) on adjoint orbits}

Using the isomorphism \(\Phi\), \cref{thm:kks} and the computation for \(\ad_X^*\), we get the following result.

\begin{theorem}
    Let \(M \subseteq \su(n)\) be an adjoint orbit. Define the \(2\)-form \(\omega\) on \(M\) by

    \[\omega_A([A, B], [A, C]) = -\inner{A, [B, C]} = \tr(A[B, C])\]

    Then \(\omega\) is a symplectic form on \(M\).
\end{theorem}

This will be convenient for us since we can compute the right hand side directly from the matrices.

% Then the almost complex structure from multiplication by \(i\) in \cref{eq:sln-root} is given by

% \begin{align*}
%     \R(E_{ij} - E_{ji}) &\mapsto i\R(E_{ij} + E_{ji}) \\
%     i\R(E_{ij} + E_{ji}) &\mapsto -\R(E_{ij} - E_{ji})
% \end{align*}

\section{K\"ahler structure}

In this section, we construct the K\"ahler structure on adjoint orbits of \(\SU(n)\). In principle, we only need two of \((\omega, g, J)\) as we can recover the third. See \Citeauthor{cannas_da_silva} \cite[\S 13.2]{cannas_da_silva} for a table which summarises the relations and the required conditions if we only have two of the three.

We have already constructed the Kirillov-Kostant-Souriau symplectic form \(\omega\). In the following, we will construct the Riemannian metric \(\iinner{\cdot, \cdot}\)\footnote{Quite often \(g\) will be used for an element of \(\SU(n)\), and so we will use \(\iinner{\cdot, \cdot}\) to denote the inner product.} and the almost complex structure \(J\). Once we have shown that these form a compatible triple, since \(\omega\) is closed and \(J\) comes from a diffeomorphism between the adjoint orbit and a complex manifold, we will have that \(\omega\) is in fact a K\"ahler form.

\subsection{At a diagonal element}

Let \(M\) be an adjoint orbit. Recall from \cref{sec:tangent} that the stabiliser of \(A\) is the torus \(T \cong T^{n-1}\) of diagonal matrices in \(\SU(n)\).

\[\T_A M = \bigoplus_{1 \le i < j \le n} \left(\R(E_{ij} - E_{ji}) \oplus i\R (E_{ij} + E_{ji})\right) \cong \frac{\su(n)}{\tilde t} \cong \T_{[1]}\left(\frac{\SU(n)}{T^{n-1}}\right)\]

where the isomorphism is induced by the quotient map

\begin{align*}
    \pi : \SU(n) &\to M \\
    g &\mapsto \Ad_g(A) = gAg^*
\end{align*}

Let \(e_{ij} = E_{ij} - E_{ji}\) and \(f_{ij} = i(E_{ij} + E_{ji})\).

\begin{lemma}
    \begin{enumerate}
        \item For any \(X, Y, Z \in \su(n)\), \(\inner{X, [Y, Z]} = \inner{[X, Y], Z}\),
        \item \([A, E_{ij}] = i(\lambda_i - \lambda_j)E_{ij}\)
        \item \(E_{ij}E_{kl} = \delta_{jk}E_{il}\)
        \item \(\inner{\cdot, \cdot}\) is \(\C\)-bilinear.
    \end{enumerate}
\end{lemma}

\begin{proof}
    Expand.
\end{proof}

Using these, we have that

\begin{align*}
    \inner{A, [E_{ij}, E_{kl}]} &= \inner{[A, E_{ij}], E_{kl}} \\
    &= i(\lambda_i - \lambda_j)\inner{E_{ij}, E_{kl}} \\
    &= -i(\lambda_i - \lambda_j)\delta_{jk}\tr(E_{il}) \\
    &= -i(\lambda_i - \lambda_j)\delta_{jk}\delta_{il} \\
    &= i(\lambda_j - \lambda_i)\delta_{il}\delta_{jk}
\end{align*}

and so,

\begin{align*}
    \inner{A, [e_{ij}, e_{kl}]} &= \inner{A, [E_{ij}, E_{kl}]} - \inner{A, [E_{ji}, E_{kl}]} - \inner{A, [E_{ij}, E_{lk}]} + \inner{A, [E_{ji}, E_{lk}]} \\
    &= i(\lambda_j - \lambda_i)\delta_{il}\delta_{jk} - i(\lambda_i - \lambda_j)\delta_{jl}\delta_{ik} - i(\lambda_j - \lambda_i)\delta_{ik}\delta_{jl} + i(\lambda_i - \lambda_j)\delta_{jk}\delta_{il} \\
    &= 0
\end{align*}

and

\begin{align*}
    -\inner{A, [f_{ij}, f_{kl}]} &= \inner{A, [E_{ij}, E_{kl}]} + \inner{A, [E_{ji}, E_{kl}]} + \inner{A, [E_{ij}, E_{lk}]} + \inner{A, [E_{ji}, E_{lk}]} \\
    &= i(\lambda_j - \lambda_i)\delta_{il}\delta_{jk} + i(\lambda_i - \lambda_j)\delta_{jl}\delta_{ik} + i(\lambda_j - \lambda_i)\delta_{ik}\delta_{jl} + i(\lambda_i - \lambda_j)\delta_{jk}\delta_{il} \\
    &= 0
\end{align*}

Finally, we have that

\begin{align*}
    -i\inner{A, [e_{ij}, f_{kl}]} &= \inner{A, [E_{ij}, E_{kl}]} - \inner{A, [E_{ji}, E_{kl}]} + \inner{A, [E_{ij}, E_{lk}]} - \inner{A, [E_{ji}, E_{lk}]} \\
    &= i(\lambda_j - \lambda_i)\delta_{il}\delta_{jk} - i(\lambda_i - \lambda_j)\delta_{jl}\delta_{ik} + i(\lambda_j - \lambda_i)\delta_{ik}\delta_{jl} - i(\lambda_i - \lambda_j)\delta_{jk}\delta_{il} \\
    &= 2i(\lambda_j - \lambda_i)(\delta_{il}\delta_{jk} + \delta_{ik}\delta_{jl}) \\
    &= 2i(\lambda_j - \lambda_i)\delta_{ik}\delta_{jl}
\end{align*}

where the last line is because we require \(i < j\) and \(k < l\). Hence we have that

\begin{align*}
    \omega_A(e_{ij}, e_{kl}) &= 0 \\
    \omega_A(f_{ij}, f_{kl}) &= 0 \\
    \omega_A(e_{ij}, f_{ij}) &= 2(\lambda_j - \lambda_i) \\
    \omega_A(e_{ij}, f_{kl}) &= 0 \text{ for }(i, j) \ne (k, l)
\end{align*}

We can then define an an almost complex structure on \(\TT_AM\) by

\begin{equation}
    \label{eq:cx-str-vector}
    J_A(e_{ij}) = f_{ij} \quad J_A(f_{ij}) = -e_{ij}
\end{equation}

Moreover, we can define an inner product on \(\TT_AM\) by

\[\iinner{e_{ij}, e_{ij}}_A = \iinner{f_{ij}, f_{ij}}_A = 2(\lambda_j - \lambda_i)\]

and requiring \(e_{ij}, f_{ij}\) to form an orthogonal basis. This is positive definite since we required \(\lambda_i < \lambda_j\) for \(i < j\). Using this, we find that

\[\omega_A(e_{ij}, f_{ij}) = 2(\lambda_j - \lambda_i) = \iinner{f_{ij}, f_{ij}}_A = \iinner{J_A(e_{ij}), f_{ij}}_A\]

and that \(J\) defines an isometry. Hence \((\omega_A, \iinner{\cdot, \cdot}_A, J_A)\) define a compatible triple on the vector space \(\TT_AM\).

\subsection{Complex quotient}

\label{sec:cx-quot}

Let \(P\) be the subgroup of lower triangular matrices in \(\SL(n, \C)\). Consider the composition \(\varphi : \SU(n) \to \SL(n, \C)/P\) given by the composition 

% https://q.uiver.app/#q=WzAsMyxbMCwwLCJcXFNVKG4pIl0sWzIsMCwiXFxTTChuKSJdLFs0LDAsIlxcU0wobiwgXFxDKS9QIl0sWzAsMSwiIiwwLHsic3R5bGUiOnsidGFpbCI6eyJuYW1lIjoiaG9vayIsInNpZGUiOiJ0b3AifX19XSxbMSwyLCIiLDAseyJzdHlsZSI6eyJoZWFkIjp7Im5hbWUiOiJlcGkifX19XV0=
\[\begin{tikzcd}
	{\SU(n)} && {\SL(n)} && {\SL(n, \C)/P}
	\arrow[hook, from=1-1, to=1-3]
	\arrow[two heads, from=1-3, to=1-5]
\end{tikzcd}\]

Suppose \(\varphi(g) = \varphi(h)\). That is, \(gP = hP\). This is true if and only if there exists \(p \in P\), such that \(h = gp\). In this case, \(p = g^{-1}h \in \SU(n)\), therefore, \(p \in \SU(n) \cap P = T\), since \(p^* = p^{-1}\) is also lower triangular. This means that \(\varphi\) induces a homeomorphism \(\SU(n)/T \cong \SL(n, \C)/P\). The right hand side is a complex manifold \(\SL(n, \C)\) quotiented by a complex Lie group \(P\), so it is a complex manifold. Using the above, we can get a complex structure on \(\SU(n)/T \cong M\).

Using the root decompositions \cref{eq:sln-root} for \(\sl(n, \C)\) and \cref{eq:sun-root} for \(\su(n)\), we can see that the almost complex structure we defined in \cref{eq:cx-str-vector} is the same as the action of multiplication by \(i\).

\subsection{At a general \(B \in M\)}

Fix \(g \in \SU(n)\), and let \(B = \Ad_g(A)\). Let \(\varphi : \SU(n)/T \to \SL(n, \C)/P\) be the diffeomorphism from above, which is given by \(\varphi([g]) = \llbracket g\rrbracket\), where \([g] = gT \in \SU(n)/T\) and \(\llbracket g \rrbracket = gP \in \SL(n, \C)/P\).

First of all, \(\SL(n, \C)\) is a complex Lie group, so left multiplication is holomorphic. That is, \(\dd \ell_g(iv) = i\dd\ell_g(v)\). If \(\tilde J\) denotes the complex structure on \(\SL(n, \C)\), then we have that

\[\tilde J_g = \dd \ell_g \circ \tilde J_I \circ \dd\ell_{g^{-1}}\]

\(\ell_g\) descends to a biholomorphism on \(\SL(n, \C)/P \cong \SU(n)/T\), and so the corresponding almost complex structure \(\overline J\) on \(\SU(n)/T\) is given by

\begin{equation}
    \label{eq:cx-str-quot}
    \overline J_{[g]} = \dd\ell_g \circ \overline J_{[I]} \circ \dd\ell_{g^*}
\end{equation}

Since the diffeomorphism \(\SU(n)/T\) is induced by the map \(\pi(g) = \Ad_g(A)\), we have that

\begin{align*}
    J_B &= \dd\pi \circ \overline J_{[g]} \circ \dd\pi^{-1} \\
    &= \dd\pi \circ \dd \ell_g \circ \overline J_{[I]} \circ \dd \ell_{g^*} \circ \dd \pi^{-1} \\
    &= \dd(\pi \circ \ell_g \circ \pi^{-1}) \circ J_A \circ \dd(\pi \circ \ell_{g^*} \circ \pi^{-1}) \\
\end{align*}

From the proof of \cref{lem:left-inv}, we have that \(\pi \circ \ell_g = \Ad_g \circ \pi\). Moreover, since \(\Ad_g\) is a linear map, \(\dd \Ad_g = \Ad_g\). Therefore, we have that the almost complex structure is given by

\[J_B = \Ad_g \circ J_A \circ \Ad_{g^*}\]

We want to show that this is compatible with the Kirillov-Kostant-Souriau symplectic form.

Recall from \cref{eq:tangent-space} that

\[\TT_BM = \Ad_g(\TT_A M)\]

Then we have that

\begin{align*}
    \omega_B([B, X], [B, Y]) &= -\inner{B, [X, Y]} \\
    &= -\inner{\Ad_g(A), \Ad_g([\Ad_{g^*}(X), \Ad_{g^*}(Y)])} \\
    &= -\inner{A, [\Ad_{g^*}(X), \Ad_{g^*}(Y)]} \\
    &= \omega_A([A, \Ad_{g^*}(X)], [A, \Ad_{g^*}(X)]) \\
    &= \omega_A(\Ad_{g^*}([B, X]), \Ad_{g^*}([B, Y]))
\end{align*}

Note this also follows from \cref{lem:left-inv} where we showed \(\pi^*\omega\) is left invariant. Therefore, the Riemannian metric is given by

\[\iinner{X, Y}_B = \iinner{\Ad_{g^*}(X), \Ad_{g^*}(Y)}_A\]

% and the \(\tilde e_{ij}, \tilde f_{ij}\) forming an orthonormal basis. Equivalently, we have

% \begin{align}
%     \label{eq:cx-str-direct}
%     J_B(X) &= \Ad_g(J_A(\Ad_{g^*}(X))) \\
%     \iinner{X, Y}_B &= \iinner{\Ad_{g^*}(X), \Ad_{g^*}(Y)}_A \nonumber
% \end{align}

% Finally, we want to show that this is the almost complex structure that we get from the complex quotient. 

% Then we have that

% \[\TT_g\SU(n) = g\su(n)\]

% using the fact that left translation is a diffeomorphism. Let

% \begin{align*}
%     \pi : \SU(n) &\to M \\
%     g &\mapsto \Ad_g(A) = gAg^*
% \end{align*}

% be the quotient map. Then

% \[\dd\pi_g(gX) = gXAg^* + gAX^*g^* = gXAg^* - gAXg^*\]

% Let \(B = \pi(g)\), then \(A = g^*Bg\), and

% \[\dd\pi_g(gX) = gXg^*Bgg^* - gg^*BgXg^* = gXg^*B - BgXg^* = -[B, gXg^*] = -[B, \Ad_g(X)]\]

% and so the pullback of \(\omega\) is

% \begin{align*}
%     (\pi^*\omega)_g(gX, gY) &= \omega_B(\dd\pi_g(gX), \dd\pi_g(gY)) \\
%     &= \omega_B([B, \Ad_g(X)], [B, \Ad_g(Y)]) \\
%     &= -\inner{B, [\Ad_g(X), \Ad_g(Y)]} \\
%     &= -\inner{\Ad_g(A), \Ad_g([X, Y])} \\
%     &= -\inner{A, [X, Y]}
% \end{align*}

% where we use the fact that \(\Ad_g\) is an isometry with respect to \(\inner{\cdot, \cdot}\). Equivalently, this is just the left invariance of the pullback we proved in \cref{lem:left-inv}.

\printbibliography

\end{document}
