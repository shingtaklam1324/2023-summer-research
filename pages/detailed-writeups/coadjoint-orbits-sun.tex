\documentclass{article}

\usepackage{../../Style}
\usepackage{stmaryrd}

\DeclareMathOperator{\SU}{SU}
\newcommand{\su}{\mathfrak{su}}
\renewcommand{\sl}{\mathfrak{sl}}
% \newcommand{\rm}[1]{\mathrm{#1}}
\renewcommand{\tilde}{\widetilde}
\newcommand{\surj}{\twoheadrightarrow}
\DeclareMathOperator{\gr}{grad}
\DeclareMathOperator{\Gr}{Gr}

\usepackage{biblatex}
\addbibresource{../../bibliography.bib}

\title{Coadjoint Orbits of \(\SU(n)\)}
\author{Shing Tak Lam}

\begin{document}

\maketitle

In this note, we will consider the coadjoint orbits of \(\SU(n)\), and show that they are K\"ahler manifolds.

Sections 1-7 is an outline of the proof, appendices A, B, C contains the details of the proof. Section 8 contains an alternative proof, which have a few holes.

\tableofcontents

\section{Adjoint and Coadjoint Orbits}

Define the Lie algebra

\[\su(n) = \left\{A \in \Mat(n, \C) \mid A^* + A = 0, \tr(A) = 0\right\}\]

where \(A^*\) is the conjugate transpose of \(A\), and with the Lie bracket being the matrix commutator. We can define the adjoint representation of \(\SU(n)\) as

\begin{align*}
    \Ad : \SU(n) &\to \GL(\su(n)) \\
    \Ad_g(X) &= gXg^*
\end{align*}

Taking the dual representation, we get the coadjoint representation, which is

\begin{align*}
    \Ad^* : \SU(n) &\to \GL(\su(n)^*) \\
    \Ad^*_g(\alpha)(X) &= \inner{\alpha, \Ad_{g^{-1}}(X)}
\end{align*}

where \(\inner{\cdot, \cdot}\) is used here to denote the pairing \(\su(n)^* \times \su(n) \to \R\). We will use the same notation for the inner product on \(\su(n)\), which should not be an issue as the inner product defines a natural isomorphism. Now note that \(-\kappa\), where \(\kappa\) is the Killing form, defines an inner product

\[\inner{A, B} = -\tr(AB) = \tr(AB^*)\]

defines an inner product on \(\su(n)\)\footnote{In fact, \(\inner{A, B} = \tr(AB^*)\) defines a Hermitian inner product on the space of complex matrices.}, which means that we have a natural isomorphism

\begin{align*}
    \Phi : \su(n) &\to \su(n)^* \\
    A &\mapsto \inner{A, \cdot}
\end{align*}

With this, suppose \(\alpha = \Phi(A)\), then

\[\Ad_g^*(\alpha)(X) = \inner{A, \Ad_{g^{-1}}(X)} = -\tr(Ag^{-1}Xg) = -\tr(gAg^{-1}X) = \Phi(\Ad_g(A))(X)\]

Therefore, the following diagram commutes

% https://q.uiver.app/#q=WzAsNCxbMCwwLCJcXHN1KG4pIl0sWzIsMCwiXFxzdShuKSJdLFswLDIsIlxcc3UobileKiJdLFsyLDIsIlxcc3UobileKiJdLFsyLDMsIlxcQWRfZ14qIl0sWzAsMSwiXFxBZF9nIl0sWzAsMiwiXFxQaGkiLDJdLFsxLDMsIlxcUGhpIl1d
\[\begin{tikzcd}
	{\su(n)} && {\su(n)} \\
	\\
	{\su(n)^*} && {\su(n)^*}
	\arrow["{\Ad_g^*}", from=3-1, to=3-3]
	\arrow["{\Ad_g}", from=1-1, to=1-3]
	\arrow["\Phi"', from=1-1, to=3-1]
	\arrow["\Phi", from=1-3, to=3-3]
\end{tikzcd}\]

or equivalently, \(\Phi\) defines an isomorphism of representations between \(\Ad\) and \(\Ad^*\).

\section{Tangent space and Diagonalisation}

\subsection{Tangent space}

Let \( M\) be a coadjoint orbit. For \(X \in \su(n)\), consider the curve \(g(t) = \exp(tX)\) in \(\SU(n)\). This has \(g'(0) = X\), and we have a curve

\[\mu(t) = \Ad_{g(t)}^*(\mu)\]

through \(\mu \in  M\). In particular, we have that for \(Y \in \su(n)\),

\[\inner{\mu(t), Y} = \inner{\mu, \Ad_{g(t)^{-1}}(Y)}\]

Differentiating this at \(t = 0\), we get

\[\inner{\mu'(0), Y} = -\inner{\mu, \ad_X(Y)} = -\inner{(\ad_X)^*(\mu), Y}\]

That is, \(\mu'(0) = -(\ad_X)^*(\mu)\). Hence we have that

\[\T_\mu\left( M\right) = \left\{(\ad_X)^*(\mu) \mid X \in \su(n)\right\}\]

If \(\alpha= \Phi(A)\), then

\begin{align*}
    \inner{(\ad_X)^*(\alpha), Y} &= \inner{\alpha, \ad_X(Y)} \\
    &= \inner{A, [X, Y]} \\
    &= -\tr(AXY - AYX) \\
    &= -\tr(AXY - XAY) \\
    &= \inner{[A, X], Y} \\
    &= \inner{-\ad_X(A), Y}
\end{align*}

Therefore, we have that

% https://q.uiver.app/#q=WzAsNCxbMCwwLCJcXHN1KG4pIl0sWzIsMCwiXFxzdShuKSJdLFswLDIsIlxcc3UobileKiJdLFsyLDIsIlxcc3UobileKiJdLFswLDIsIlxcUGhpIiwyXSxbMSwzLCJcXFBoaSJdLFswLDEsIlxcYWRfWCJdLFsyLDMsIlxcYWRfWF4qIiwyXV0=
\[\begin{tikzcd}
	{\su(n)} && {\su(n)} \\
	\\
	{\su(n)^*} && {\su(n)^*}
	\arrow["\Phi"', from=1-1, to=3-1]
	\arrow["\Phi", from=1-3, to=3-3]
	\arrow["{\ad_X}", from=1-1, to=1-3]
	\arrow["{\ad_X^*}"', from=3-1, to=3-3]
\end{tikzcd}\]

Thus, in this case we have the tangent space to the corresponding adjoint orbit as

\[T_A M = \left\{\ad_X(A) \mid X \in \su(n)\right\}\]

\subsection{Diagonalisation and Stabilisers of the coadjoint action}

First of all, we note that

\[\su(n) = \left\{A \in \Mat(n, \C) \mid A^* + A = 0, \tr(A) = 0\right\}\]

where \(A^*\) is the conjugate transpose of \(A\). In particular, all elements of \(\su(n)\) are skew-hermitian, hence diagonalisable by an element of \(\SU(n)\)\footnote{From standard linear algebra arguments, we know that they are \(\mathrm U(n)\)-diagonalisable. But if \(PAP^{-1}\) is diagonal, then so is \((\lambda P)A(\lambda P)^{-1}\), and by choosing \(\lambda\) appropriately, \(\lambda \in \SU(n)\).}. With this, we can classify the coadjoint orbits based off a diagonal element in the orbit. Consider

\[A = \begin{pmatrix}
    i\lambda_1 I_{m_1} \\
    & \ddots \\
    & & i\lambda_k I_{m_k}
\end{pmatrix}\]

where \(I_m\) is the \(m \times m\) identity matrix, \(\lambda_j \in \R\), with \(\lambda_1 > \lambda_2 > \dots > \lambda_k\), \(m_1 + \dots + m_k = n\) and \(m_1\lambda_1 + \dots + m_k\lambda_k = 0\). In this case, we have that the orbit is

\[\Orb(A) \cong \SU(n)/\Stab(A)\]

where \(\Stab(A)\) is the stabiliser of \(A\) under the adjoint action. In this case, we have that the stabiliser is the block diagonal subgroup

\[\Stab(A) = \rm S\left(\rm U(m_1) \times \dots \times \rm U(m_k)\right)\]

where we consider \(\rm U(m_1) \times \cdots \times \rm U(m_k) \le \SU(n)\) as the block diagonal subgroup, and

\[\rm S\left(\rm U(m_1) \times \dots \times \rm U(m_k)\right) = \left(\rm U(m_1) \times \cdots \times \rm U(m_k)\right) \cap \SU(n)\]

the subgroup with determinant \(1\). Therefore, the coadjoint orbit is diffeomorphic to the flag manifold

\[\mcF(m_1, \dots, m_k) = \frac{\SU(n)}{\rm S\left(\rm U(m_1) \times \cdots \times \rm U(m_k)\right)}\]

In particular, note that \(\mcF(p, n-p)\) is diffeomorphic to the Grassmannian \(\Gr(p, n)\) of \(p\)-dimensional subspaces of \(\C^n\). More generally, a generic element of \(\mcF(m_1, \dots, m_k)\) can be represented as \((V_1, \dots, V_k)\), where \(V_j\) is a dimension \(m_j\) subspace of \(\C^n\), with \(V_j \perp V_k\) for all \(j \ne k\)\footnote{This may not be the usual definition of a (partial) flag manifold, which is a sequence of subspaces 

\[0 = W_0 \subset W_1 \subset \cdots \subset W_k = \C^n\]

However it is equivalent, since we can set \(W_0 = 0\), \(W_1 = V_1\), \(W_2 = V_1 \oplus V_2\) and so on. Moreover, the indexing is slightly different, since we are indexing using \(m_1, \dots, m_k\), instead of \(\dim(W_1) = m_1, \dim(W_2) = m_1 + m_2\) and so on. Again, it is easy to convert between the two definitions.}. This is because we can set \(V_j\) to be the \(i\lambda_j\) eigenspace and vice versa.

\section{Kirillov-Kostant-Souriau symplectic form}

This section is from \cite{marsden_ratiu} Chapter 14. The proof is slightly modified, since here we have an explicit isomorphism between the adjoint and coadjoint representation, which simplifies some of the arguments.

\begin{theorem}
    \label{thm:kks}
    Let \(M \subseteq \su(n)^*\) be a coadjoint orbit. Define the \(2\)-form \(\omega\) on \( M\) by

    \[\omega_\mu(\ad_\xi^*(\mu), \ad_\eta^*(\mu)) = -\inner{\mu, [\xi, \eta]}\]

    Then \(\omega\) is a symplectic form on \( M\).
\end{theorem}

\subsection{\(\omega\) is well defined}

First of all, we show that \(\omega\) is well defined. That is, it is independent of the choice of \(\xi, \eta \in \su(n)\).

Suppose \(\zeta \in \su(n)\) is such that \(\ad_\zeta^*(\mu) = \ad_\xi^*(\mu)\). Then we must have that

\[\inner{\mu, [\xi, \eta]} = \inner{\mu, [\zeta, \eta]}\]

for all \(\eta \in \su(n)\).

\subsection{\(\omega\) is non-degenerate}

Since the pairing \(\inner{,}\) is non-degenerate, \(\omega(\mu)(\ad_\xi^*(\mu), \ad_\eta^*(\mu))\) for all \(\ad_\eta^*(\mu)\) implies that \(\inner{\mu, [\xi, \eta]} = 0\), for all \(\eta\). But this then means that \(\ad_\xi^*(\mu) = 0\), so \(\omega\) is non-degenerate.

\subsection{\(\omega\) is closed}

First of all, we will need some preliminary results.

\begin{lemma}
    \[\ad_{\Ad_g\xi}^* = \Ad_g^* \circ \ad_\xi^* \circ \Ad_{g^*}^*\]
\end{lemma}

\begin{proof}
    We will prove the corresponding statement for \(\Ad\) and \(\ad\), and the result will follow by conjuation with the isomorphism \(\Phi\).

    \[\Ad_g \circ \ad_\xi \circ \Ad_g^*(X)= g\xi g^* X g g^* - g g^* X g \xi g^* = [g\xi g^*, X] = \ad_{\Ad_g\xi}(X)\]
\end{proof}

\begin{lemma}
    \[\Ad_g([\xi, \eta]) = [\Ad_g(\xi), \Ad_g(\eta)]\]
\end{lemma}

\begin{proof}
    First, notice that

    \[C_g(C_h(k)) = ghkh^{-1}g^{-1}= C_g(h)C_g(k)C_g(h^{-1})\]

    Differentiating this at \(h = e\) and \(k = e\) gives the result.
\end{proof}

\begin{lemma}
    \(\Ad_g^* :  M \to  M\) preserves \(\omega\), that is,

    \[(\Ad_g^*)^*\omega = \omega\]
\end{lemma}

\begin{proof}
    Evaluating \((\Ad_\xi)_{\su(n)^*} = \Ad_g^* \circ \ad_\xi^* \circ \Ad_{g^{-1}}^*\) at \(\nu = \Ad_g^*(\mu)\), we get

    \[(\Ad_g\xi)_{\su(n)^*}(\nu) = \Ad_g^* \circ \ad_\xi^*(\mu) = \dd_\mu\Ad_g^* \circ \ad_\xi^*(\mu)\]

    Therefore,

    \begin{align*}
        ((\Ad_g^*)^*\omega)(\mu)(\ad_\xi^*(\mu), \ad_\eta^*(\mu)) &= \omega(\nu)(\dd_\mu \Ad_g^* \cdot \ad_\xi^*(\mu), \dd_\mu \Ad_g^* \cdot \ad_\eta^*(\mu)) \\
        &= \omega(\nu)((\Ad_g\xi)_{\su(n)^*}(\nu), (\Ad_g\eta)_{\su(n)^*}(\nu)) \\
        &= -\inner{\nu, [\Ad_g\xi, \Ad_g\eta]} \\
        &= -\inner{\nu, \Ad_g([\xi, \eta])} \\
        &= -\inner{\Ad_{g^{-1}}^*(\nu), [\xi, \eta]} \\
        &= -\inner{\mu, [\xi, \eta]} \\
        &= \omega(\mu)(\ad_\xi^*(\mu), \ad_\eta^*(\mu))
    \end{align*}
\end{proof}

For \(\nu \in \su(n)^*\), define the left-invariant one-form

\[\nu_\ell(g) = (\dd_g\ell_{g^{-1}})^*(\nu)\]

for \(g \in G\). Similarly, for \(\xi \in \su(n)\), let \(\xi_\ell\) be the corresponding left invariant vector field on \(G\). Then \(\nu_\ell(\xi_\ell) = \inner{\nu, \xi}\) at all \(g \in G\).

Fix \(\nu \in  M\), and consider the map \(\varphi_\nu : G \to  M\), defined by

\[\varphi_\nu(g) = \Ad_g^*(\nu)\]

We can use this to pullback \(\sigma = (\varphi_\nu)^*\omega\) to a two form on \(G\).

\begin{lemma}
    \(\sigma\) is left invariant. That is, \(\ell_g^*\sigma = \sigma\) for all \(g \in G\).
\end{lemma}

\begin{proof}
    First, notice that \(\varphi_\nu \circ \ell_g = \Ad_g^* \circ \varphi_\nu\), since

    \[\varphi_\nu(\ell_g(h)) = \Ad_{gh}^*(\nu) = \Ad_g^*\circ \Ad_h^*(\nu) = \Ad_g^*(\varphi_\nu(h))\]

    With this,

    \[\ell_g^*\sigma = \ell_g^*\varphi^*\omega = (\varphi\circ\ell_g)^*\omega = (\Ad_g^* \circ \varphi_\nu)^*\omega = (\varphi_\nu)^*(\Ad_g^*)^*\omega = (\varphi_\nu)^*\omega = \sigma\]
\end{proof}

\begin{lemma}
    \(\sigma(\xi_\ell, \eta_\ell) = -\inner{\nu_\ell, [\xi_\ell, \eta_\ell]}\).
\end{lemma}

\begin{proof}
    By left invariance of both sides, suffices to show that the result holds at \(e\). First notice that

    \[\dd_e\varphi_\nu(\eta) = \eta_{\su(n)_*}(\nu)\]

    Therefore, \(\varphi_\nu\) is a submersion at \(e\). By definition of the pullback,

    \begin{align*}
        \sigma(e)(\xi, \eta) &= (\varphi_\nu)^*\omega(e)(\xi, \eta)\\ 
        &= \omega(\varphi_\nu(e))(\dd_e\varphi_\nu \cdot \xi, \dd_e\varphi_\nu\cdot\eta) \\
        &= \omega(\nu)(\ad_\xi^*(\nu), \ad_\eta^*(\nu)) \\
        &= -\inner{\nu, [\xi, \eta]}
    \end{align*}

    Hence

    \[\sigma(\xi_\ell, \eta_\ell)(e) = \sigma(e)(\xi, \eta) = -\inner{\nu, [\xi, \eta]} = -\inner{\nu_\ell, [\xi_\ell, \eta_\ell]}(e)\]
\end{proof}

Now for a one form \(\alpha\), we have that

\[\dd\alpha(X, Y) = X[\alpha(Y)] - Y[\alpha(X)] - \alpha([X, Y])\]

where for a smooth function \(f : M \to \R\), and a vector field \(X\) on \(M\), \(X[f] := \dd f(X)\) is a smooth function \(M \to \R\).

Since \(\nu_\ell(\xi_\ell)\) is constant, \(\eta_\ell[\nu_\ell(\xi_\ell)] = 0\). Similarly, \(\xi_\ell[\nu_\ell(\eta_\ell)] = 0\). Therefore, we have that

\[\dd \nu_\ell(\xi_\ell, \eta_\ell) = -\nu_\ell([\xi_\ell, \eta_\ell]) = \sigma(\xi_\ell, \eta_\ell)\]

Now suppose \(X, Y\) are vector fields on \(G\). We want to show that \(\sigma(X, Y) = \dd\nu_\ell(X, Y)\). As \(\sigma\) is left invariant,

\begin{align*}
    \sigma(X, Y)(g) &= (\ell_{g^{-1}}^* \sigma)(g)(X(g), Y(g)) \\
    &= \sigma(e)(\underbrace{\dd \ell_{g^{-1}} \cdot X(g)}_{=\xi}, \underbrace{\dd \ell_{g^{-1}} \cdot Y(g)}_{=\eta}) \\
    &= \sigma(e)(\xi, \eta) \\
    &= \dd\nu_\ell(\xi_\ell, \eta_\ell)(e) \\
    &= (\ell_g^*\dd\nu_\ell)(\xi_\ell, \eta_\ell)(e) \\
    &= (\dd\nu_\ell)(g)(\dd\ell_g\cdot \xi_\ell(e), \dd\ell_g\cdot\eta_\ell(e)) \\
    &= (\dd\nu_\ell)(g)(\dd \ell_g\cdot \xi, \dd\ell_g\cdot \eta) \\
    &= (\dd\nu_\ell)(g)(X(g), Y(g)) \\
    &= \dd\nu_\ell(X, Y)(g)
\end{align*}

With this, \(\dd\sigma = \dd^2\nu_\ell = 0\). Hence \((\varphi_\nu)^*\dd\omega = \dd((\varphi_\nu)^*\omega) = \dd\sigma = 0\). Since \(\varphi_\nu \circ \ell_g = \Ad_g^* \circ \ell_g\), and \(\varphi_\nu\) is a submersion at \(e\), it is infact a submersion everywhere. Moreover, \(\varphi_\nu\) is surjective, by definition.

For \(\mu \in  M\), and \(X, Y \in T_\mu  M\), we have that

\[\dd\omega(\mu)(X, Y) = \dd\omega_{\varphi_\nu(g)}(\dd\varphi_\nu(\xi), \dd\varphi_\nu(\eta)) = ((\varphi_\nu)^*\dd\omega)(g)(\xi, \eta) = 0\]

where \(g \in G\) is such that \(\varphi_\nu(g) = \mu\), which exists by surjectivity, and \(\xi, \eta \in T_gG\) such that \(\dd\varphi_\nu(\xi) = X\) and \(\dd\varphi_\nu(\eta) = Y\), which exists as \(\varphi_\nu\) is a submersion. Thus, as \(\mu \in  M\) is arbitrary, \(\omega\) is closed.

\subsection{\(\omega\) on adjoint orbits}

Using the isomorphism \(\Phi\), \cref{thm:kks} and the computation for \(\ad_\xi^*\), we get the following result.

\begin{theorem}
    Let \(M \subseteq \su(n)\) be an adjoint orbit. Define the \(2\)-form \(\omega\) on \(M\) by

    \[\omega_A([A, B], [A, C]) = -\inner{A, [B, C]} = \tr(A[B, C])\]

    Then \(\omega\) is a symplectic form on \(M\).
\end{theorem}

\section{Root decomposition}

Consider the Lie algebra \(\sl(n, \C)\) of trace free \(n \times n\) complex matrices. Then we have the Cartan subalgebra \(\mft\) of diagonal matrices. Let \(E_{ij}\) be the standard basis matrices for \(\Mat(n, \C)\), \(B \in \mft\). Say

\[B = \begin{pmatrix}
    b_1 \\
    & \ddots \\
    & & b_n
\end{pmatrix}\]

Then \([B, E_{ij}] = (b_i - b_j)E_{ij}\). This means that we have the eigendecomposition

\begin{equation}
    \label{eq:sln-root}
    \sl(n, \C) = \mft \oplus \bigoplus_{1 \le i, j \le n, i \ne j} \C E_{ij}
\end{equation}

In particular, if we restrict this to the subalgebra \(\su(n)\), we get the decomposition

\[\su(n) = \tilde\mft \oplus \bigoplus_{1 \le i < j \le n} \left(\R(E_{ij} - E_{ji}) \oplus i\R (E_{ij} + E_{ji})\right)\]

where \(\tilde \mft = \mft \cap \su(n)\) is the subalgebra of \(\su(n)\) of diagonal matrices. In particular, we have that (assuming the eigenvalues of \(A\) are distinct)

\[\T_A M \cong \T_{[1]}\left(\frac{\SU(n)}{T^{n-1}}\right) \cong \frac{\su(n)}{\tilde t} = \bigoplus_{1 \le i < j \le n} \left(\R(E_{ij} - E_{ji}) \oplus i\R (E_{ij} + E_{ji})\right)\]

Then the almost complex structure from multiplication by \(i\) in \cref{eq:sln-root} is given by

\begin{align*}
    \R(E_{ij} - E_{ji}) &\mapsto i\R(E_{ij} + E_{ji}) \\
    i\R(E_{ij} + E_{ji}) &\mapsto -\R(E_{ij} - E_{ji})
\end{align*}

\section{Complex structure}

We will now focus on the generic case, where \(A\) has distinct eigenvalues. In this case, the stabiliser is the diagonal subgroup \(T\), which is isomorphic to the torus \(T^{n-1} = (S^1)^{n-1}\).

Let \(P\) be the subgroup of lower triangular matrices in \(\SL(n, \C)\). Consider the composition \(\varphi : \SU(n) \to \SL(n, \C)/P\) given by the composition 

% https://q.uiver.app/#q=WzAsMyxbMCwwLCJcXFNVKG4pIl0sWzIsMCwiXFxTTChuKSJdLFs0LDAsIlxcU0wobiwgXFxDKS9QIl0sWzAsMSwiIiwwLHsic3R5bGUiOnsidGFpbCI6eyJuYW1lIjoiaG9vayIsInNpZGUiOiJ0b3AifX19XSxbMSwyLCIiLDAseyJzdHlsZSI6eyJoZWFkIjp7Im5hbWUiOiJlcGkifX19XV0=
\[\begin{tikzcd}
	{\SU(n)} && {\SL(n)} && {\SL(n, \C)/P}
	\arrow[hook, from=1-1, to=1-3]
	\arrow[two heads, from=1-3, to=1-5]
\end{tikzcd}\]

Suppose \(\varphi(g) = \varphi(h)\). That is, \(gP = hP\). This is true if and only if there exists \(p \in P\), such that \(h = gp\). In this case, \(p = g^{-1}h \in \SU(n)\), therefore, \(p \in \SU(n) \cap P = T\), since \(p^* = p^{-1}\) is also lower triangular. This means that \(\varphi\) induces a homeomorphism \(\SU(n)/T \cong \SL(n, \C)/P\). The right hand side is a complex manifold \((\SL(n, \C))\) quotiented by a complex Lie group \(P\), so it is a complex manifold. Using the above, we can get a complex structure on \(\SU(n)/T \cong  M\).

\section{Local coordinates}

This section

Let \(\theta\) be the Maurer-Cartan form on \(\SU(n)\). That is, it is the \(\su(n)\)-valued \(1\)-form on \(\SU(n)\) given by

\[\theta_g(u) = \dd (\ell_{g^{-1}})_g(u) \in \TT_e\SU(n) = \su(n)\]

Writing \(\theta = \sum_{j, k}\theta_{jk}\dd g^{jk}\) where \((g^{jk})\) are the matrix entries on \(\SU(n)\). Then we have that

\[\pi^*\omega = i \sum_{k > j}(\lambda_j - \lambda_k)\theta_{kj}\wedge \overline{\theta_{kj}}\]

and the Hermitian metric \(h\) on \( M\) is given by

\[\pi^*h = \sum_{k > j}2(\lambda_j - \lambda_k)\theta_{kj}\overline{\theta_{kj}}\]

Using the fact that

\[\TT_g\SU(n) = g\su(n)\]

and that

\[\dd\pi_g(h) = hAg^* + gAh^*\]

we get that

\[\dd\pi_g(h) = [B, hg^*]\]

where \(B = \pi(g) = gAg^*\). With this, we can recover the hermitian metric, as

\[h_B([B, C], [B, D]) = \sum_{k > j}2(\lambda_j - \lambda_k)(g^* Cg)_{kj}\overline{(g^* Dg)_{kj}}\]

With this, we can see that \(\omega = -\Im(h)\), and that the real part \(g = \Re(h)\)\footnote{The usual notation for a Riemannian metric is \(g\), but I've also used \(g\) as the element of \(\SU(n)\). Hopefully it should be clear from context which one is which.} defines a Riemannian metric.

Moreover, we can recover the almost complex structure from the symplectic form and the Riemannian metric, via

\[J = \tilde g^{-1} \circ \tilde\omega\]

where \(\tilde\omega, \tilde g : \TT \SU(n) \to \TT^*\SU(n)\) are linear isomorphisms given by

\begin{align*}
    \tilde \omega(u)(v) &= \omega(u, v) \\
    \tilde g(u)(v) &= g(u, v)
\end{align*}

This follows as we have that \(g(u, v) = \omega(u, Jv)\).

\appendix

\section{Local coordinate computations}

\subsection{Kirillov-Konstant-Souriau form}

Let \( M\) be an adjoint orbit in \(\su(n)\), and fix \(\xi \in  M\) as above. Recall that for \(\mu \in  M\),

\[\TT_\mu M = \left\{[\mu, A] \mid A \in \su(n)\right\}\]

and the Kirillov-Konstant-Souriau form \(\omega\) is defined by

\[\omega_\mu([\mu, A], [\mu, B]) = -\inner{\mu, [A, B]}\]

\subsubsection{Computation at \(I\)}

Computing the pullback, we find that \(\pi^*\omega\) at the identity is given by

\[(\pi^*\omega)_I(A, B) = \omega_\xi([\mu, A], [\mu, B]) = -\inner{\xi, [A, B]}\]

Computing this in terms of the coordinates, we find that it is

\begin{align*}
    -\inner{\xi, [A, B]} &= \tr(\xi[A, B]) \\
    &= i\sum_{j, k}\xi_j(A_{jk}B_{kj} - B_{jk}A_{kj}) \\
    &= i\sum_{j, k}\xi_j(A_{kj}\overline{B_{kj}} - \overline{A_{kj}}B_{kj}) \\
    &= \frac{i}{2} \sum_{j,k}\xi_j(A_{kj}\overline{B_{kj}} - \overline{A_{kj}}B_{kj}) + \frac{i}{2}\sum_{j, k}\xi_k(A_{jk} \overline{B_{jk}} - \overline{A_{jk}}B_{jk}) \\
    &= \frac{i}{2} \sum_{j,k}\xi_j(A_{kj}\overline{B_{kj}} - \overline{A_{kj}}B_{kj}) - \frac{i}{2}\sum_{j, k}\xi_k(A_{kj} \overline{B_{kj}} - \overline{A_{kj}}B_{kj}) \\
    &= \frac{i}{2}\sum_{j, k}(\xi_j - \xi_k)(A_{kj}\overline{B_{kj}} - \overline{A_{kj}}B_{kj}) \\
    &= i \sum_{k > j}(\xi_j - \xi_k)(A_{kj}\overline{B_{kj}} - \overline{A_{kj}}B_{kj})
\end{align*}

Let \((\theta_{jk})\) be the standard coordinate functions on \(\su(n)\), then

\[(\pi^*\omega)_I = i \sum_{k > j}(\xi_j - \xi_k)\theta_{kj}\wedge \overline{\theta_{kj}}\]

where \(\alpha \wedge \beta(v, w) = \alpha(v)\beta(w) - \alpha(w)\beta(v)\).

\subsubsection{Computation at any \(g \in \SU(n)\)}

Now let \(\theta\) be the Maurer-Cartan form on \(\SU(n)\). That is, it is the \(\su(n)\)-valued \(1\)-form on \(\SU(n)\) given by

\[\theta_g(u) = \dd (\ell_{g^{-1}})_g(u) \in \TT_e\SU(n) = \su(n)\]

Writing \(\theta = \sum_{j, k}\theta_{jk}\dd g^{jk}\) where \((g^{jk})\) are the matrix entries on \(\SU(n)\), in fact we have that

\[\pi^*\omega = i \sum_{k > j}(\xi_j - \xi_k)\theta_{kj}\wedge \overline{\theta_{kj}}\]

Since\footnote{\S A.3.3} \(\pi^*\omega\) and \(\theta\) are both left invariant, and the above expressions agree at the identity, they must agree everywhere. Alternatively, we can compute this directly, as in the next section.

\subsection{Hermitian structure}

We will define a Hermitian metric \(h\) on \( M\), using the fact that \(\pi\) is a surjective submersion. That is, we have

\[\pi^*h = \sum_{k > j}2(\xi_j - \xi_k)\theta_{kj}\overline{\theta_{kj}}\]

where \(\alpha\beta(v, w) = \alpha(v)\beta(w)\).

\subsubsection{Computation at \(I\)}

At \(\xi\), the above formula gives us that

\[h_\xi([\xi, A], [\xi, B]) = \sum_{k > j}2(\xi_j - \xi_k)A_{kj}\overline{B_{kj}}\]

which one can check defines a Hermitian metric (at least in the case when the \(\xi_j\) are distinct).

\subsection{Computation at \(g \in \SU(n)\)}

In this case, if \(A, B \in \TT_g\SU(n)\), then we can see that \(\theta(A) = g^* A\) and \(\theta(B) = g^* B\). Therefore, we have that

\[\pi^*h(A, B) = \sum_{k > j}2(\xi_j - \xi_k)(g^* A)_{kj}\overline{(g^* B)_{kj}}\]

However, by the definition of the pullback, we also have that

\[\pi^*h(A, B) = h_{\pi(g)}(\dd \pi_g(A), \dd\pi_g(B))\]

Set \(\mu = \pi(g)\), then we have that

\[\pi^*h(A, B) = h_\mu([\mu, Ag^*], [\mu, Bg^*])\]

Say \(A = Cg, B = Dg\), then we get that

\[h_\mu([\mu, C], [\mu, D]) = \sum_{k > j}2(\xi_j - \xi_k)(g^* Cg)_{kj}\overline{(g^* Dg)_{kj}}\]

Again, in the case the \(\xi_j\) are distinct, we can see that this defines a Hermitian metric on \( M\).

\printbibliography

\end{document}
