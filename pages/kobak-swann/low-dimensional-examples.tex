\documentclass{article}

\usepackage{../../Style}
\usepackage[english]{babel}
\usepackage{biblatex}

\newcommand{\rla}{\rightleftarrows}

\addbibresource{../../bibliography.bib}

\title{Low dimensional examples}

\author{Shing Tak Lam}

\begin{document}

\maketitle

In this note, we study low dimensional and short examples of the construction in \cite{kobak_classical_1996}. We will use the notation as in the paper. In addition, we will write

\begin{itemize}
	\item \([a_1^{n_1}, \dots, a_{\ell}^{n_\ell}]\) for the Jordan type with \(n_i\) Jordan blocks of size \(a_i\).
	\item \(\mcO\left(a_1^{n_1}, \dots, a_\ell^{n_\ell}\right)\) for the corresponding orbit.
	\item \(\mcM\left(n_1, \dots, n_{k-1}\right)\) for the space of diagrams of the form
	% https://q.uiver.app/#q=WzAsNixbMiwwLCJcXENee25fMX0iXSxbNCwwLCJcXENee25fMn0iXSxbNiwwLCJcXGNkb3RzIl0sWzgsMCwiXFxDXntuX3trLTF9fSJdLFsxMCwwLCJcXENebiJdLFswLDAsIjAiXSxbMCwxLCIiLDAseyJvZmZzZXQiOi0xfV0sWzEsMiwiIiwwLHsib2Zmc2V0IjotMX1dLFsyLDMsIiIsMCx7Im9mZnNldCI6LTF9XSxbMyw0LCIiLDAseyJvZmZzZXQiOi0xfV0sWzQsMywiIiwwLHsib2Zmc2V0IjotMX1dLFszLDIsIiIsMCx7Im9mZnNldCI6LTF9XSxbMiwxLCIiLDAseyJvZmZzZXQiOi0xfV0sWzEsMCwiIiwwLHsib2Zmc2V0IjotMX1dLFswLDUsIiIsMCx7Im9mZnNldCI6LTF9XSxbNSwwLCIiLDAseyJvZmZzZXQiOi0xfV1d
	\[\begin{tikzcd}
		0 && {\C^{n_1}} && {\C^{n_2}} && \cdots && {\C^{n_{k-1}}} && {\C^n}
		\arrow[shift left=1, from=1-3, to=1-5]
		\arrow[shift left=1, from=1-5, to=1-7]
		\arrow[shift left=1, from=1-7, to=1-9]
		\arrow[shift left=1, from=1-9, to=1-11]
		\arrow[shift left=1, from=1-11, to=1-9]
		\arrow[shift left=1, from=1-9, to=1-7]
		\arrow[shift left=1, from=1-7, to=1-5]
		\arrow[shift left=1, from=1-5, to=1-3]
		\arrow[shift left=1, from=1-3, to=1-1]
		\arrow[shift left=1, from=1-1, to=1-3]
	\end{tikzcd}\]
	and we will write \(\mathbf n = (n_1, \dots, n_{k-1})\).
	\item \(\mcN(\mathbf n) = \mu_c^{-1}(0)/G_\C\) is the quotient space.
\end{itemize}

\section*{Case \(k=1\)}

In this case, the diagram is

% https://q.uiver.app/#q=WzAsMixbMCwwLCIwIl0sWzIsMCwiVl8xIl0sWzAsMSwiXFxhbHBoYV8wIiwwLHsib2Zmc2V0IjotMX1dLFsxLDAsIlxcYmV0YV8wIiwwLHsib2Zmc2V0IjotMX1dXQ==
\[\begin{tikzcd}
	0 && {V_1}
	\arrow["{\alpha_0}", shift left=1, from=1-1, to=1-3]
	\arrow["{\beta_0}", shift left=1, from=1-3, to=1-1]
\end{tikzcd}\]

So there is only one point in \(M\), which is \((0, 0)\). In this case, the image of \(\Phi^c\) is the zero orbit.

\section*{Case \(k=2\)}

In this case, we have the diagram

% https://q.uiver.app/#q=WzAsMixbMCwwLCJcXENeayJdLFsyLDAsIlxcQ15uIl0sWzAsMSwiIiwwLHsib2Zmc2V0IjotMX1dLFsxLDAsIiIsMCx7Im9mZnNldCI6LTF9XV0=
\[\begin{tikzcd}
	{\C^m} && {\C^n}
	\arrow[shift left=1, from=1-1, to=1-3]
	\arrow[shift left=1, from=1-3, to=1-1]
\end{tikzcd}\]

which means that \(X\) has \(X^2 = 0\), and \(\rank(X) \le m\). Equality can be achieved if \(n \ge 2k\), and \(X\) has Jordan type \([2^m, 1^{n-2m}]\). In this case, the quotient space is

\[\bigcup_{\ell = 0}^m \mcO\left(2^\ell, 1^{n-2\ell}\right)\]

\section*{Orderings}

For the general case, we have the following statement:

\begin{theorem*}
	Let \(X, Y\) be nilpotent \(n \times n\) matrices. Then \(Y\) is in the closure of the orbit of \(X\) if and only if \(\rank(Y^i) \le \rank(X^i)\) for all \(i\).
\end{theorem*}

We can see that the \(k=2\) case above is a special case of this, and the theorem by \Citeauthor{kobak_classical_1996}.

Moreover, we can see that if \(n_i \le m_i\) for all \(i\), then we have a natural embedding \(\mcM(\mathbf n) \le \mcM(\mathbf m)\). In addition, since

\begin{theorem*}
	If \(\rank(X^i) = n_{k-i}\) for all \(i\), then the quotient of \(\mcM(\mathbf n)\) is the closure of the orbit of \(X\).
\end{theorem*}

and the result is sharp, in the sense that if \(n_{k-i} < \rank(X^i)\), then \(X\) can't be in \(\mcN(\mathbf n)\), as \(X^i\) factors through \(\C^{n_{k-i}}\). Therefore, we can use this to explicitly write down \(\mcN(\mathbf n)\) as a union of orbits.

A Mathematica notebook to compute the Hasse diagram for a fixed \(n\), and the ranks of the Jordan blocks is at \texttt{code/Hasse.nb}.

\section*{Examples}

We can use the above notebook to compute examples for small \(n\). For \(n \le 5\), we get a linear order. Since the maximal element is always the Jordan block, which comes from the diagram

\[0 \rla \C \rla \C^2 \rla \cdots \rla \C^{n-1} \rla \C^n\]

Therefore, all the other orbits can be constructed from ``sub-diagrams'' of the above.

For \(n = 6\), we no longer get a linear order. \(\rank([4, 1^2]) = [3,2,1]\) and \(\rank([3,3]) = [4, 2]\).

\printbibliography

\end{document}
