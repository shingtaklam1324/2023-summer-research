\documentclass{article}

\usepackage{../../Style}
\usepackage[english]{babel}
\usepackage{biblatex}

\addbibresource{../../bibliography.bib}

\title{Low dimensional examples}

\author{Shing Tak Lam}

\begin{document}

\maketitle

In this note, we study low dimensional examples of the construction in \cite{kobak_classical_1996}. We will use the notation as in \S 2 of the paper.

\section{Case \(k=1\)}

In this case, the diagram is

% https://q.uiver.app/#q=WzAsMixbMCwwLCIwIl0sWzIsMCwiVl8xIl0sWzAsMSwiXFxhbHBoYV8wIiwwLHsib2Zmc2V0IjotMX1dLFsxLDAsIlxcYmV0YV8wIiwwLHsib2Zmc2V0IjotMX1dXQ==
\[\begin{tikzcd}
	0 && {V_1}
	\arrow["{\alpha_0}", shift left=1, from=1-1, to=1-3]
	\arrow["{\beta_0}", shift left=1, from=1-3, to=1-1]
\end{tikzcd}\]

So there is only one point in \(M\), which is \((0, 0)\). In this case, the image of \(\Phi^c\) is the zero orbit.

\section{Case \(k=2\)}

In this case, we have the diagram

% https://q.uiver.app/#q=WzAsMyxbMCwwLCIwIl0sWzIsMCwiVl8xIl0sWzQsMCwiVl8yIl0sWzAsMSwiXFxhbHBoYV8wIiwwLHsib2Zmc2V0IjotMX1dLFsxLDAsIlxcYmV0YV8wIiwwLHsib2Zmc2V0IjotMX1dLFsxLDIsIlxcYWxwaGFfMSIsMCx7Im9mZnNldCI6LTF9XSxbMiwxLCJcXGJldGFfMSIsMCx7Im9mZnNldCI6LTF9XV0=
\[\begin{tikzcd}
	0 && {V_1} && {V_2}
	\arrow["{\alpha_0}", shift left=1, from=1-1, to=1-3]
	\arrow["{\beta_0}", shift left=1, from=1-3, to=1-1]
	\arrow["{\alpha_1}", shift left=1, from=1-3, to=1-5]
	\arrow["{\beta_1}", shift left=1, from=1-5, to=1-3]
\end{tikzcd}\]

In this case, for a point \(p = (\alpha_1, \beta_1) \in \mu_c^{-1}(0)\)\footnote{We omit \(\alpha_0, \beta_0\) as they are zero.}, we have \(X = \alpha_1\beta_1\), and \(\beta_1\alpha_1 = 0\), so \(X^2 = 0\). Therefore, all of the Jordan blocks for \(X\) have size at most \(2\).

Since \(\rank(X) = \rank(\alpha_1\beta_1) \le \min\left\{\rank(\alpha_1, \beta_1)\right\} \le \dim(V_1)\) , and \(\rank(X)\) is the number of nonzero Jordan blocks, this gives us a relation between the number of Jordan blocks and the dimension of \(V_1\).

\printbibliography

\end{document}
