\documentclass{article}

\usepackage{../../Style}

\usepackage{biblatex}
\addbibresource{../../bibliography.bib}

\DeclareMathOperator{\SU}{SU}
\newcommand{\su}{\mathfrak{su}}

\renewcommand{\sl}{\mathfrak{sl}}

\DeclareMathOperator{\gr}{grad}

\newcommand{\iinner}[1]{\left\langle\!\left\langle #1 \right\rangle\!\right\rangle}

\title{Riemannian metric}
\author{Shing Tak Lam}

\begin{document}

\maketitle

In this note, we will compute the Riemannian metric on the adjoint orbit, following \cite{kobak_classical_1996}.

Choose a sequence \(n_0, \dots, n_k\), with \(n_0 = 0\) and \(n_k = n\). Then define the space

\[M = \bigoplus_{j=0}^{k-1}\left(\Hom(\C^{n_j}, \C^{n_{j+1}}) \oplus \Hom(\C^{n_{j+1}}, \C^{n_j})\right)\]

We will identify \(M\) with the matrix space

\[M = \bigoplus_{j=1}^{k-1}\left(\Mat_{n_{j+1} \times n_j}(\C) \oplus \Mat_{n_j \times n_{j+1}}(\C)\right)\]

and write a general point as \((\alpha_j, \beta_j)\), where \(\alpha_j : \C^{n_j} \to \C^{n_{j+1}}\) and \(\beta_j : \C^{n_{j+1}} \to \C^{n_j}\). Define the moment maps

\begin{align*}
    \mu_r(\alpha_j, \beta_j) &= (\alpha_{j-1}\alpha_{j-1}^* - \beta_{j-1}^*\beta_{j-1}+\beta_j\beta_j^* - \alpha_j^*\alpha_j)_{j=1}^{k-1} \\
    \mu_c(\alpha_j, \beta_j) &= (\alpha_{j-1}\beta_{j-1} - \beta_j\alpha_j)_{j=1}^{k-1}
\end{align*}

Computing the derivatives at the fixed point \(p = (\alpha_j, \beta_j)\),

\begin{align*}
    \dd\mu_r(\delta_j, \epsilon_j) &= \left(\alpha_{j-1}\delta_{j-1}^* + \delta_{j-1}\alpha_{j-1}^* - \epsilon_{j-1}^*\beta_{j-1} - \beta_{j-1}^*\epsilon_{j-1} + \epsilon_j\beta_j^* + \beta_j\epsilon_j^* - \alpha_j^*\delta_j - \delta_j^*\alpha_j\right)_{j=1}^{k-1} \\
    \dd\mu_c(\delta_j, \epsilon_j) &= \left(\delta_{j-1}\beta_{j-1} + \alpha_{j-1}\epsilon_{j-1} - \beta_j\delta_j - \epsilon_j\alpha_j\right)_{j=1}^{k-1}
\end{align*}

and the tangent space of \(\mu^{-1}(0) = \mu_c^{-1}(0) \cap \mu_r^{-1}(0)\) is

\[\T_p\mu^{-1}(0) = \ker(\dd\mu_r) \cap \ker(\dd\mu_c)\]

Next, note that if we differentiate the \(G = \rm U(n_1) \times \cdots \times \rm U(n_{k-1})\) action, we will get the kernel of the quotient map, which is

\[V = \left\{(X_{j+1}\alpha_j - \alpha_jX_j, X_j\beta_j - \beta_jX_{j+1}) \mid X_j \in \mfu(n_j)\right\}\]

The Riemannian metric on the quotient space \(\mu^{-1}(0)/G\) is induced by the linear isomorphism

\[\T_{[p]}(\mu^{-1}(0)/G) \cong H := V^\perp\]

given by the quotient map. Next, note that the map \(\tilde\Phi^c(\alpha_j, \beta_j) = \alpha_{k-1}\beta_{k-1}\) defines a diffeomorphism \(\Phi^c : \mu^{-1}(0)/G \to N\), where \(N\) is a nilpotent orbit (when restricted to an open subset of \(\mu^{-1}(0)\)).

% https://q.uiver.app/#q=WzAsMyxbMCwwLCJcXG11XnstMX0oMCkiXSxbMCwyLCJcXG11XnstMX0oMCkvRyJdLFsyLDIsIk4iXSxbMCwxLCIiLDAseyJzdHlsZSI6eyJoZWFkIjp7Im5hbWUiOiJlcGkifX19XSxbMSwyLCJcXFBoaV5jIl0sWzAsMiwiXFx0aWxkZVxcUGhpXmMiXV0=
\[\begin{tikzcd}
	{\mu^{-1}(0)} \\
	\\
	{\mu^{-1}(0)/G} && N
	\arrow["\pi"', two heads, from=1-1, to=3-1]
	\arrow["{\Phi^c}", from=3-1, to=3-3]
	\arrow["{\tilde\Phi^c}", from=1-1, to=3-3]
\end{tikzcd}\]

Therefore, we can compute the Riemannian metric on \(\mu^{-1}(0)/G\) by considering its pullback to \(\mu^{-1}(0)\). Moreover, since \(N\) is a complex submanifold of \(\sl(n, \C)\), it also has a natural Riemannian metric \(g\) from its K\"ahler structure. We can then consider the pullback \((\tilde\Phi^c)^*g\) as an inner product on \(H\).

Suppose \((\delta_j, \epsilon_j) \in H\). Then for all \((u_j, v_j) \in V\), say \(u_j = X_{j+1}\alpha_j - \alpha_jX_j, v_j = X_j\beta_j - \beta_jX_{j+1}\),

\begin{align*}
    \sum_{j=1}^{k-1}\Re\left(\tr(\delta_j u_j^*) + \tr(\epsilon_jv_j^*)\right) &= \sum_{j=1}^{k-1}\Re\left(\tr((\alpha_j^*\delta_j - \epsilon_j\beta_j^*)X_j) + \tr((\beta_j^*\epsilon_j - \delta_j\alpha_j^*)X_{j+1})\right) \\
    &= \Re\sum_{j=1}^{k-1}\tr((\alpha_j^*\delta_j - \epsilon_j\beta_j^* + \beta_{j-1}^*\epsilon_{j-1} - \delta_{j-1}\alpha_{j-1}^*)X_j)
\end{align*}

Therefore, a sufficient condition is

\[\alpha_j^*\delta_j - \epsilon_j\beta_j^* + \beta_{j-1}^*\epsilon_{j-1} - \delta_{j-1}\alpha_{j-1}^* = 0\]

Finally, we compute

\[\dd\Phi^c(\delta, \epsilon) = \delta_{k-1}\beta_{k-1} + \alpha_{k-1}\epsilon_{k-1}\]

Combining all of the above, we have the following conditions

\begin{enumerate}
    \item \(\ker(\dd\mu_r)\) \[\alpha_{j-1}\delta_{j-1}^* + \delta_{j-1}\alpha_{j-1}^* - \epsilon_{j-1}^*\beta_{j-1} - \beta_{j-1}^*\epsilon_{j-1} + \epsilon_j\beta_j^* + \beta_j\epsilon_j^* - \alpha_j^*\delta_j - \delta_j^*\alpha_j = 0\]
    \item \(\ker(\dd\mu_c)\) \[\delta_{j-1}\beta_{j-1} + \alpha_{j-1}\epsilon_{j-1} - \beta_j\delta_j - \epsilon_j\alpha_j = 0\]
    \item \(\mu_c = 0\) \[\alpha_{j-1}\beta_{j-1} - \beta_j\alpha_j = 0\]
    \item \(\mu_r = 0\) \[\alpha_{j-1}\alpha_{j-1}^* - \beta_{j-1}^*\beta_{j-1}+\beta_j\beta_j^* - \alpha_j^*\alpha_j = 0\]
    \item Orthogonality \[\alpha_j^*\delta_j - \epsilon_j\beta_j^* + \beta_{j-1}^*\epsilon_{j-1} - \delta_{j-1}\alpha_{j-1}^* = 0\]
\end{enumerate}

Note also that 5. implies 1.

With all of these, it is then clear that

\[J(\delta_j, \epsilon_j) = (-\epsilon_j^*, \delta_j^*)\]

defines a linear map \(H \to H\). Hence we have the complex structure \(J\) on \(N\), given by

\[J(\delta_{k-1}\beta_{k-1} + \alpha_{k-1}\epsilon_{k-1}) = \alpha_{k-1}\delta_{k-1}^* - \epsilon_{k-1}^*\beta_{k-1}\]

\printbibliography

\end{document}