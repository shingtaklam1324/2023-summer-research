\documentclass{article}

\usepackage{../../Style}
\usepackage{biblatex}
\addbibresource{../../bibliography.bib}

\newcommand{\iinner}[1]{\left\langle\!\left\langle #1 \right\rangle\!\right\rangle}

\DeclareMathOperator{\Sp}{Sp}

\title{Computation of moment map}

\author{Shing Tak Lam}

\begin{document}

\maketitle

\section{HyperK\"ahler moment maps}

On \(\bb H^N\), we can construct a hyperK\"ahler structure, using the standard metric given by

\[\inner{u, v} = \overline u^\T v\]

where \(\overline u\) denotes the (elementwise) quaternionic conjugate of \(u\). The complex structures are given by right multiplication by \(-i, -j, -k\) respectively. Let \(\omega_I, \omega_J, \omega_K\) be the corresponding K\"ahler forms, and \(\eta = \omega_I i + \omega_J j + \omega_K k\).

Let \(H\) be a subgroup of \(\Sp(N)\). Then \(H\) acts on \(\bb H^N\) preserving the hyperK\"ahler structure. In this case, a hyperK\"ahler moment map is a map \(\mu : \bb H^N \to \mfh^* \otimes \Im(\bb H)\), which is equivariant with respect to the \(H\), and with \(\dd(\mu^X) = X \lrcorner \eta\)

In particular, in \cite{kobak_classical_1996}, we make the choice

\[\mu^X(q) = -\overline q^\T X q = -\inner{q, Xq}\]

where we define \(\inner{u, v} = \overline u^\T v\) for elements of \(\bb H^N\).

\section{\(\rm U(n)\) action}

Choose a sequence \((V_0, \dots, V_k)\) of Hermitian vector spaces, with \(\dim_\C(V_i) = n_i\), \(n_0 = 0, n_k = n\). Let

\[M = \bigoplus_{i=0}^{k-1}\left(\Hom(V_i, V_{i+1}) \oplus \Hom(V_{i+1}, V_i)\right)\]

and we write a point of \(M\) as \((\alpha_i, \beta_i)\), where

% https://q.uiver.app/#q=WzAsNCxbMCwwLCJWXzAiXSxbMSwwLCJWXzEiXSxbMywwLCJWX2siXSxbMiwwLCJcXGNkb3RzIl0sWzAsMSwiXFxhbHBoYV8wIiwwLHsib2Zmc2V0IjotMX1dLFsxLDMsIlxcYWxwaGFfMSIsMCx7Im9mZnNldCI6LTF9XSxbMywyLCJcXGFscGhhX3trLTF9IiwwLHsib2Zmc2V0IjotMX1dLFsyLDMsIlxcYmV0YV97ay0xfSIsMCx7Im9mZnNldCI6LTF9XSxbMywxLCJcXGJldGFfMSIsMCx7Im9mZnNldCI6LTF9XSxbMSwwLCJcXGJldGFfMCIsMCx7Im9mZnNldCI6LTF9XV0=
\[\begin{tikzcd}[ampersand replacement=\&]
	{V_0} \& {V_1} \& \cdots \& {V_k}
	\arrow["{\alpha_0}", shift left=1, from=1-1, to=1-2]
	\arrow["{\alpha_1}", shift left=1, from=1-2, to=1-3]
	\arrow["{\alpha_{k-1}}", shift left=1, from=1-3, to=1-4]
	\arrow["{\beta_{k-1}}", shift left=1, from=1-4, to=1-3]
	\arrow["{\beta_1}", shift left=1, from=1-3, to=1-2]
	\arrow["{\beta_0}", shift left=1, from=1-2, to=1-1]
\end{tikzcd}\]

Note that \(\inner{\alpha, \beta} = \tr(\alpha\beta^*)\) defines a Hermitian metric on \(\Hom(V, W)\), and hence on \(M\), we have the metric

\[\iinner{(\alpha_i, \beta_i), (\tilde\alpha_i, \tilde\beta_i)} = \sum_{i=0}^{k-1}\left(\inner{\alpha_i, \tilde\alpha_i} + \inner{\beta_i, \tilde\beta_i}\right)\]

The complex structures are

\[I(\alpha_i, \beta_i) = (i\alpha_i, i\beta_i) \quad J(\alpha_i, \beta_i) = (-\beta_i^*, \alpha_i^*)\]

The Lie group \(G = \rm U(n_1) \times \rm U(n_{k-1})\) acts on \(M\) via

\begin{equation*}
    \begin{split}
        \alpha_i &\mapsto g_{i+1}\alpha_i g_i^{-1} = g_{i+1}\alpha_i g_i^* \\
        \beta_i &\mapsto g_i\beta_i g_{i+1}^{-1} = g_i\beta_i g_{i+1}^*
    \end{split}
\end{equation*}

Now notice that \(\inner{\alpha, \beta}\) as above induces an isomorphism \(\mfu(m) \cong \mfu(m)^*\), via \(X \mapsto \inner{X, \cdot}\).

\section{Moment map}

Now let \(X_i \in \mfu(n_i)\), and let \(X = (0, \dots, X_i, \dots, 0) \in \mfu(n_1) \oplus \cdots \oplus \mfu(n_{k-1})\). Let \(q = (\alpha_i, \beta_i) \in M\). Then the action of \(X\) is

\[Xq = (0, \dots, X_i \alpha_{i-1}, -\alpha_iX_i, \dots, 0, 0, \dots, -\beta_{i-1}X_i, X_i\beta_i, \dots, 0)\]

In particular, we have that

\begin{align*}
    \iinner{q, Xq} &= \inner{\alpha_{i-1}, X_i\alpha_{i-1}} - \inner{\alpha_i, \alpha_iX_i} - \inner{\beta_{i-1}, \beta_{i-1}X_i} + \inner{\beta_i, X_i\beta_i} \\
    &= \tr(\alpha_{i-1}\alpha_{i-1}^*X^* - \alpha_iX_i^*\alpha_i^* - \beta_{i-1}X_i^*\beta_{i-1}^* + \beta_i\beta_i^*X^*) \\
    &= \tr((\alpha_i^*\alpha_i - \beta_i\beta_i^* + \beta_{i-1}^*\beta_{i-1} - \alpha_{i-1}\alpha_{i-1}^*)X_i) \\
\end{align*}

which gives us

\[\mu_r = (\alpha_{i-1}\alpha_{i-1}^* - \beta_{i-1}^*\beta_{i-1} + \beta_i\beta_i^* -\alpha_i^*\alpha_i)\]

Next, we can take

\begin{align*}
    \iinner{q, X\cdot (-J)(q)} &= -\inner{\alpha_{i-1}, -X_i\beta_{i-1}^*} + \inner{\alpha_i, -\beta_iX_i^*} + \inner{\beta_{i-1}, \alpha_{i-1}^*X_i} - \inner{\beta_i, \alpha_i^*X_i} \\
    &= \tr(\alpha_{i-1}\beta_{i-1}X_i^* - \alpha_iX_i^*\beta_i + \beta_{i-1}X_i^*\alpha_{i-1} - \beta_iX_i^*\alpha_i) \\
    &= -2\tr((\alpha_{i-1}\beta_{i-1} - \beta_i\alpha_i)X_i)
\end{align*}

which gives us

\[\mu_c = (\alpha_{i-1}\beta_{i-1} - \beta_i\alpha_i)\]

Note also that \(IX = XI\) and \(JX = XJ\), and that \(I, J, K=IJ\) define isometries with respect to \(\iinner{\cdot, \cdot}\), so we also have that

\[\iinner{q, X \cdot (-J)q} = \iinner{Jq, Xq}\]

\printbibliography

\end{document}
