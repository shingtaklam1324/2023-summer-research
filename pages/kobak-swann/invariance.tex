\documentclass{article}

\usepackage{../../Style}
\usepackage{biblatex}
\addbibresource{../../bibliography.bib}

\title{Invariance of the hyperK\"ahler metric}
\author{Shing Tak Lam}

\begin{document}

\maketitle

In this note, we will investigate the invariance of the hyperK\"ahler metric from \cite{kobak_classical_1996}. We will use the same notation as in section 2.

\textbf{Nope, this doesn't work...}

Let \(g\) denote the Riemannian metric on \(M\). That is,

\[g((\alpha_j, \beta_j), (\tilde\alpha_j, \tilde\beta_j)) = \sum_{j=0}^{k-1}\Re\tr(\alpha_j^*\tilde\alpha_j + \beta_j\tilde\beta_j^*)\]

We have a \(\SL(n, \C)\) acton on \(M\), by

\[\psi_\gamma(\alpha_j, \beta_j) = (\alpha_0, \dots, \alpha_{k-2}, \gamma\alpha_{k-1}, \beta_0, \dots, \beta_{k-2}, \beta_{k-1}\gamma^{-1})\]

Since \(\psi_\gamma : M \to M\) is linear, \(\dd\psi_\gamma = \psi_\gamma\). In particular, for

\begin{align*}
    \omega_J((\alpha_j, \beta_j), (\tilde\alpha_j, \tilde\beta_j)) &= g(J(\alpha_j, \beta_j), (\tilde\alpha_j, \tilde\beta_j)) \\
    &= \sum_{j=0}^{k-1}\Re\tr(-(\beta_j^*)^*\tilde\alpha_j + \alpha_j^*\tilde\beta_j^*) \\
    &= \sum_{j=0}^{k-1}\Re\tr(-\beta_j\tilde\alpha_j + \alpha_j^*\tilde\beta_j^*) \\
\end{align*}

We will need to show that



But trace is conjugation invariant. Hence \(\psi_g^*\omega_J = \omega_J\). Similarly, we have that \(\psi_g^*\omega_K = \omega_K\). Therefore, the complex symplectic form \(\omega_c = \omega_J + i\omega_K\) is \(\psi_g\)-invariant. Now notice that if \(\tilde\Phi = \Phi^c \circ \pi\), then \(\tilde\Phi\) is a surjective submersion, and

% https://q.uiver.app/#q=WzAsNCxbMCwwLCJcXG11X2Neey0xfSgwKSJdLFsyLDAsIlxcbXVfY157LTF9KDApIl0sWzAsMiwiXFxPcmIoQSkiXSxbMiwyLCJcXE9yYihBKSJdLFswLDIsIlxcUGhpIiwyXSxbMSwzLCJcXFBoaSJdLFswLDEsIlxccHNpX1xcZ2FtbWEiXSxbMiwzLCJcXEFkX1xcZ2FtbWEiLDJdXQ==
\[\begin{tikzcd}[ampersand replacement=\&]
	{\mu_c^{-1}(0)} \&\& {\mu_c^{-1}(0)} \\
	\\
	{\Orb(A)} \&\& {\Orb(A)}
	\arrow["\Phi"', from=1-1, to=3-1]
	\arrow["\Phi", from=1-3, to=3-3]
	\arrow["{\psi_\gamma}", from=1-1, to=1-3]
	\arrow["{\Ad_\gamma}"', from=3-1, to=3-3]
\end{tikzcd}\]

commutes. This means that the complex-symplectic form \(\tilde\omega_c\) is \(\Ad_\gamma\)-invariant. To see this, first note that \(\Phi^*\tilde\omega_c\) is the restriction of \(\omega_c\) to \(\mu_c^{-1}(0)\). In this case,

\begin{align*}
    \Phi^*\Ad_\gamma^*\tilde\omega_c &= \psi_\gamma^*\Phi^*\tilde\omega_c \\
    &= \psi_\gamma^*i^*\omega_c \\
    &= i^*\omega_c \\
    &= \Phi^*\tilde\omega_c
\end{align*}

As \(\Phi^*\) is injective, \(\Ad_\gamma^*\tilde\omega_c = \tilde\omega_c\).

Note however that

\begin{align*}
    \psi_\gamma^*g((\alpha_j, \beta_j), (\tilde\alpha_j, \tilde\beta_j)) &= g(\psi_\gamma(\alpha_j, \beta_j), \psi_\gamma(\tilde\alpha_j, \tilde\beta_j)) \\
\end{align*}

\end{document}
