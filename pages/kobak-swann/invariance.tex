\documentclass{article}

\usepackage{../../Style}
\usepackage{biblatex}
\addbibresource{../../bibliography.bib}

\renewcommand{\tilde}{\widetilde}
\DeclareMathOperator{\SU}{SU}

\title{Invariance of the hyperK\"ahler metric}
\author{Shing Tak Lam}

\begin{document}

\maketitle

In this note, we will investigate the invariance of the hyperK\"ahler metric from \cite{kobak_classical_1996}. We will use the same notation as in section 2.

Let \(g\) denote the Riemannian metric, that is,

\[g((\alpha_j, \beta_j), (\tilde\alpha_j, \tilde\beta_j)) = \sum_{j=0}^{k-1}\left(\Re\tr(\alpha_j^*\tilde\alpha_j) + \Re\tr(\beta_j^*\tilde\beta_j)\right)\]

Therefore, we have that

\begin{align*}
    \omega_J((\alpha_j, \beta_j), (\tilde\alpha_j, \tilde\beta_j)) &= g(J(\alpha_j, \beta_j), (\tilde\alpha_j, \tilde\beta_j))\\
    &= g((-\beta_j^*, \alpha_j^*), (\tilde\alpha_j, \tilde\beta_j)) \\
    &= \sum_{j=0}^{k-1}\left(\Re\tr(-\beta_j\tilde\alpha_j) + \Re\tr(\alpha_j\tilde\beta_j)\right)
\end{align*}

We also have a \(\SL(n, \C)\) action on \(M\), given by

\[\psi_\gamma (\alpha_j, \beta_j) = (\alpha_0, \dots, \alpha_{k-2}, \gamma\alpha_{k-1}, \beta_0, \dots, \beta_{k-2}, \beta_{k-1}\gamma^{-1})\]

\(\psi_\gamma : M \to M\) is linear, therefore the derivative is itself. In particular,

\begin{align*}
    \psi_\gamma^*\omega_J((\alpha_j, \beta_j), (\tilde\alpha_j, \tilde\beta_j)) &= \sum_{j=0}^{k-2}\left(\Re\tr(-\beta_j\tilde\alpha_j) + \Re\tr(\alpha_j\tilde\beta_j)\right) + \Re\tr(-\beta_{k-1}\gamma^{-1}\gamma\tilde\alpha_{k-1}) + \Re\tr(\gamma\alpha_{k-1}\tilde\beta_{k-1}\gamma^{-1}) \\
    &= \omega_J((\alpha_j, \beta_j), (\tilde\alpha_j, \tilde\beta_j))
\end{align*}

using the fact that trace is conjugation invariant. Similarly, \(\psi_\gamma^*\omega_K = \omega_K\) as trace is \(\C\)-linear, and so the \(\Re\) becomes \(-\Im\).

Define \(\omega_c = \omega_J + i\omega_K\) for the complex symplectic form on \(M\). Let \(\Phi = \Phi^c \circ \pi : \mu_c^{-1}(0) \to \mcN\) be defined by

\[\Phi(\alpha_j, \beta_j) = \alpha_{k-1}\beta_{k-1}\]

Let \(N\) be the image of \(\Phi\). Using the results from \cite{kobak_classical_1996}, \(N\) has a hyperK\"ahler metric \(\tilde\omega_c\), satisfying

\[\Phi^*\tilde\omega_c = i^*\omega_c\]

where \(i : \mu_c^{-1}(0) \to M\) is the inclusion. Since \(\Phi : \mu_c^{-1}(0) \to N\) is a surjective submersion, this completely determines \(\tilde\omega_c\). We would like to show \((\Ad_\gamma)^*\tilde\omega_c = \tilde\omega_c\) for all \(\gamma \in \SL(n, \C)\). As

% https://q.uiver.app/#q=WzAsNCxbMCwwLCJcXG11X2Neey0xfSgwKSJdLFsyLDAsIlxcbXVfY157LTF9KDApIl0sWzAsMiwiXFxPcmIoQSkiXSxbMiwyLCJcXE9yYihBKSJdLFsyLDMsIlxcQWRfXFxnYW1tYSIsMl0sWzAsMiwiXFxQaGkiLDJdLFsxLDMsIlxcUGhpIl0sWzAsMSwiXFxwc2lfXFxnYW1tYSJdXQ==
\[\begin{tikzcd}[ampersand replacement=\&]
	{\mu_c^{-1}(0)} \&\& {\mu_c^{-1}(0)} \\
	\\
	{\Orb(A)} \&\& {\Orb(A)}
	\arrow["{\Ad_\gamma}"', from=3-1, to=3-3]
	\arrow["\Phi"', from=1-1, to=3-1]
	\arrow["\Phi", from=1-3, to=3-3]
	\arrow["{\psi_\gamma}", from=1-1, to=1-3]
\end{tikzcd}\]

commutes,

\[\Phi^*(\Ad_\gamma)^*\tilde\omega_c = \psi_\gamma^*\Phi^*\tilde\omega_c = \psi_\gamma^*i^*\omega_c = i^*\omega_c = \Phi^*\tilde\omega_c\]

Therefore, we have that \(\tilde\omega_c\) is invariant under the \(\SL(n, \C)\) action.

Moreover, the Riemannian metric is invariant under the action of the compact subgroup \(\SU(n)\). To see this,

\begin{align*}
    \psi_\gamma^*g((\alpha_j, \beta_j), (\tilde\alpha_j, \tilde\beta_j)) &= \sum_{j=0}^{k-2}\left(\Re\tr(\alpha_j^*\tilde\alpha_j) + \Re\tr(\beta_j^*\tilde\beta_j)\right) + \Re\tr(\alpha_j\gamma^*\gamma\tilde\alpha_j) + \Re\tr(\gamma\beta_j^*\tilde\beta_j\gamma^*) \\
    &= g((\alpha_j, \beta_j), (\tilde\alpha_j, \tilde\beta_j))
\end{align*}

\printbibliography

\end{document}
