\documentclass{article}

\usepackage{../../Style}
\usepackage{biblatex}
\addbibresource{../../bibliography.bib}


\title{Length reduction in diagrams in \citetitle{kobak_classical_1996}}

\author{Shing Tak Lam}

\begin{document}

\maketitle
On page 23 of \cite{kobak_classical_1996}, there is an example of constructing a two step diagram from a three step one, and showing that the first is a quotient of the second. In this note, we will go through the details of this construction.

For simplicity, let us first consider the case of \(\SL(n, \C)\) orbits. We start off with a diagram of the form

% https://q.uiver.app/#q=WzAsMyxbMCwwLCJcXENeYSJdLFsxLDAsIlxcQ157YStifSJdLFsyLDAsIlxcQ157YStiK2N9Il0sWzAsMSwiXFxhbHBoYV8xIiwwLHsib2Zmc2V0IjotMX1dLFsxLDIsIlxcYWxwaGFfMiIsMCx7Im9mZnNldCI6LTF9XSxbMiwxLCJcXGJldGFfMiIsMCx7Im9mZnNldCI6LTF9XSxbMSwwLCJcXGJldGFfMSIsMCx7Im9mZnNldCI6LTF9XV0=
\begin{equation}
    \label{eq:three-step}
    \begin{tikzcd}
        {\C^a} & {\C^{a+b}} & {\C^{a+b+c}}
        \arrow["{\alpha_1}", shift left=1, from=1-1, to=1-2]
        \arrow["{\alpha_2}", shift left=1, from=1-2, to=1-3]
        \arrow["{\beta_2}", shift left=1, from=1-3, to=1-2]
        \arrow["{\beta_1}", shift left=1, from=1-2, to=1-1]
    \end{tikzcd}
\end{equation}

We can then use this diagram to construct a two step diagram

% https://q.uiver.app/#q=WzAsMixbMCwwLCJcXENee2ErYn0iXSxbMiwwLCJcXENeYSBcXG9wbHVzIFxcQ157YStiK2N9Il0sWzAsMSwiXFxiZWdpbntwbWF0cml4fVxcYmV0YV8xIFxcXFwgXFxhbHBoYV8yXFxlbmR7cG1hdHJpeH0iLDAseyJvZmZzZXQiOi0xfV0sWzEsMCwiXFxiZWdpbntwbWF0cml4fVxcYWxwaGFfMSAmIC1cXGJldGFfMlxcZW5ke3BtYXRyaXh9IiwwLHsib2Zmc2V0IjotMX1dXQ==
\begin{equation}
    \label{eq:two-step}
    \begin{tikzcd}[ampersand replacement=\&]
        {\C^{a+b}} \&\& {\C^a \oplus \C^{a+b+c}}
        \arrow["{\begin{pmatrix}\beta_1 \\ \alpha_2\end{pmatrix}}", shift left=1, from=1-1, to=1-3]
        \arrow["{\begin{pmatrix}-\alpha_1 & \beta_2\end{pmatrix}}", shift left=1, from=1-3, to=1-1]
    \end{tikzcd}
\end{equation}

\section*{Dimensions}

The dimension of the flat space (as a complex vector space) in the first case is

\[2a(a+b) + 2(a+b)(a+b+c)\]

and in the second case, it is

\[2(a + (a + b + c))(a + b)\]

and these are equal.

\section*{Group actions}

In \cref{eq:three-step}, we have a \(\rm U(a) \times \rm U(a + b)\) action, by

\begin{equation}
    \label{eq:three-step-action}
    (g_1, g_2) \cdot (\alpha_1, \beta_1, \alpha_2, \beta_2) = (g_2\alpha_1g_1^{-1}, g_1\beta_1g_2^{-1}, \alpha_2g_2^{-1}, g_1\beta_2)
\end{equation}

and the corresponding moment map is given by

\begin{align*}
    \mu_c &= (-\beta_1\alpha_1, \alpha_1 \beta_1 - \beta_2\alpha_2) \\
    \mu_r &= (\beta_1\beta_1^* - \alpha_1^*\alpha_1, \alpha_1\alpha_1^* - \beta_1^*\beta_1 + \beta_2\beta_2^* - \alpha_2^*\alpha_2)
\end{align*}

In \cref{eq:two-step}, we have a \(\rm U(a + b)\) action, by

\begin{equation}
    \label{eq:two-step-action}
    g \cdot \left(\begin{pmatrix}
        \beta_1 \\ \alpha_2
    \end{pmatrix}, \begin{pmatrix}
        -\alpha_1 & \beta_2
    \end{pmatrix}\right) = \left(\begin{pmatrix}
        \beta_1 \\ \alpha_2
    \end{pmatrix} \cdot g^{-1}, g \cdot \begin{pmatrix}
        -\alpha_1 & \beta_2
    \end{pmatrix}\right) = \left(\begin{pmatrix}
        \beta_1g^{-1} \\ \alpha_2g^{-1}
    \end{pmatrix}, \begin{pmatrix}
        -g\alpha_1 & g\beta_2
    \end{pmatrix}\right)
\end{equation}

Which we can also see is \cref{eq:three-step-action} with \(g_1 = 1, g_2 = g\). In this case, the moment map is

\begin{align*}
    \tilde \mu_c &= -\begin{pmatrix}
        -\alpha_1 & \beta_2
    \end{pmatrix}\begin{pmatrix}
        \beta_1 \\ \alpha_2
    \end{pmatrix} = \alpha_1\beta_2 - \beta_2\alpha_2 \\
    \tilde \mu_r &= \begin{pmatrix}
        -\alpha_1 & \beta_2
    \end{pmatrix}\begin{pmatrix}
        -\alpha_1 & \beta_2
    \end{pmatrix}^* - \begin{pmatrix}
        \beta_1 \\ \alpha_2
    \end{pmatrix}^*\begin{pmatrix}
        \beta_1 \\ \alpha_2
    \end{pmatrix} = \alpha_1\alpha_1^* - \beta_1^*\beta_1 + \beta_2\beta_2^* - \alpha_2^*\alpha_2
\end{align*}

Moreover, we can consider the \(\rm U(a)\) action coming from \cref{eq:three-step-action}, i.e. 

\[g \cdot \left(\begin{pmatrix}
    \beta_1 \\ \alpha_2
\end{pmatrix}, \begin{pmatrix}
    -\alpha_1 & \beta_2
\end{pmatrix}\right) = \left(\begin{pmatrix}
    g\beta_1 \\ \alpha_2
\end{pmatrix}, \begin{pmatrix}
    -\alpha_1 g^{-1} & \beta_2
\end{pmatrix}\right)\]

and the moment map is

\begin{align*}
    \overline \mu_c &= -\beta_1\alpha_1 \\
    \overline \mu_r &= \beta_1\beta_1^* - \alpha_1^*\alpha_1
\end{align*}

With all of these, we can see that

\[\frac{\mu^{-1}(0)}{\rm U(a) \times \rm U(a + b)} \cong \frac{\tilde \mu^{-1}(0)/\rm U(a + b)}{\rm U(a)}\]

i.e. the hyperK\"ahler quotient of \cref{eq:three-step} is the hyperK\"ahler quotient by \(\rm U(a)\) of the hyperK\"ahler quotient of \cref{eq:two-step}.

\printbibliography

\end{document}
