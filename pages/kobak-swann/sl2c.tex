\documentclass{article}

\usepackage{../../Style}
\usepackage{biblatex}
\addbibresource{../../bibliography.bib}
\newcommand{\rla}{\rightleftarrows}

\renewcommand{\sl}{\mfs\mfl}

\title{Computation of hyperK\"ahler metric on nilpotent \(\SL(2, \C)\) orbits}
\author{Shing Tak Lam}

\begin{document}

\maketitle

In this note, we will compute explicitly the hyperK\"ahler metric on the nilpotent \(\SL(2, \C)\) orbit using \cite{kobak_classical_1996}.

In this case, the only nonzero nilpotent orbit is the orbit of

\[A = \begin{pmatrix}
    0 & 1 \\
    0 & 0
\end{pmatrix}\]

which corresponds to the diagram

\[0 \rla \C \rla \C^2\]

Choosing a basis for \(\C\) and \(\C^2\), we can identify \(M = \C^4\), with

\[\alpha = \begin{pmatrix}
    \alpha_1 \\ \alpha_2
\end{pmatrix} \qquad \beta = \begin{pmatrix}
    \beta_1 & \beta_2
\end{pmatrix}\]

The complex moment map in this case is

\[\mu_c(\alpha, \beta) = \beta\alpha = \alpha_1 \beta_1 + \alpha_2\beta_2\]

The action of the group \(\rm U(1) \cong S^1\) on \(M\) is given by

\[g \vdot (\alpha, \beta) = (\alpha g^{-1}, g\beta)\]

In terms of the components, this is given by

\[\lambda \cdot (\alpha_1, \alpha_2, \beta_1, \beta_2) = (\lambda^{-1}\alpha_1, \lambda^{-1}\alpha_2, \lambda\beta_1, \lambda\beta_2)\]

Define the map \(\Phi : M \to \mcN\) by

\[\Phi(\alpha, \beta) = \alpha\beta = \begin{pmatrix}
    \alpha_1\beta_1 & \alpha_1\beta_2 \\
    \alpha_2\beta_1 & \alpha_2\beta_2
\end{pmatrix}\]

where \(\mcN\) is the nilpotent variety of \(\sl(2, \C)\). In particular, as we are interested in the nonzero orbit, we will want to consider the restriction of \(\Phi\) to \(M\setminus \{0\}\). In particular, \(\Phi\) factors as

% https://q.uiver.app/#q=WzAsMyxbMCwwLCJcXG11X2Neey0xfSgwKSJdLFswLDIsIlxcbXVfY157LTF9KDApL0deXFxDIl0sWzIsMiwiXFxtY04iXSxbMCwxLCJcXHBpIiwyLHsic3R5bGUiOnsiaGVhZCI6eyJuYW1lIjoiZXBpIn19fV0sWzEsMiwiXFxvdmVybGluZVxcUGhpIiwyXSxbMCwyLCJcXFBoaSJdXQ==
\[\begin{tikzcd}[ampersand replacement=\&]
	{\mu_c^{-1}(0)} \\
	\\
	{\mu_c^{-1}(0)/G^\C} \&\& \mcN
	\arrow["\pi"', two heads, from=1-1, to=3-1]
	\arrow["\overline\Phi"', from=3-1, to=3-3]
	\arrow["\Phi", from=1-1, to=3-3]
\end{tikzcd}\]

From \cite[Theorem 2.7]{kobak_classical_1996}, \(\overline\Phi\) defines a diffeomorphism onto its image. In particular, this means that we can compute the hyperK\"ahler metric on the adjoint orbit of \(A\).

Let \(\Orb(A)\) denote the adjoint orbit of \(A\). Then we have that

\[\T_A\Orb(A) = \Span_\C\left\{e, f\right\}\]

where

\[e = \begin{pmatrix}
    0 & -2 \\
    0 & 0
\end{pmatrix} \qquad f = \begin{pmatrix}
    -1 & 0 \\
    0 & 1
\end{pmatrix}\]

If \(\omega_c = \omega_J + i\omega_K\) is the complex-symplectic form on \(M\), and \(\hat\omega_c\) is the complex-symplectic form on \(\mu_c^{-1}(0)\), then we have that

\[\pi^*\hat\omega_c = i^*\omega_c\]

where \(i : \mu_c^{-1}(0) \to M\) is the inclusion map. In particular, if we transport across the diffeomorphism \(\overline\Phi\), and let \(\tilde \omega_c\) be the complex-symplectic form on \(\Orb(A)\), then we have that

\[\Phi^*\tilde\omega_c = i^*\omega_c\]

We would like to compute \(\tilde\omega_c\) at \(A\). One preimage is \((1, 0, 0, 1) \in M\). Computing \(\dd\Phi\) at \((1, 0, 0, 1)\), we find that

\[\dd\Phi_{(1, 0, 0, 1)}(h_1, h_2, k_1, k_2) = \begin{pmatrix}
    k_1 & h_1 + k_2 \\
    0 & h_2
\end{pmatrix}\]

Next, using standard arguments, we find that

\[\T_{(1, 0, 0, 1)}\mu_c^{-1}(0) = \left\{k_1 + h_2 = 0\right\}\]

In particular, setting

\[\hat e = (-2, 0, 0, 0) \qquad \hat f = (0, 1, -1, 0)\]

we have that \(\dd\Phi(\hat e) = e\) and \(\dd\Phi(\hat f) = f\). Therefore, to compute \(\tilde\omega_c\), all we need to do is compute \(\omega_c(\hat e, \hat f)\).

In terms of the coordinates, the complex structure \(J\) is given by

\[J(\alpha_1, \alpha_2, \beta_1, \beta_2) = (-\overline \beta_1, -\overline\beta_2, \overline \alpha_1, \overline \alpha_2)\]

and so

\[\omega_J(\hat e, \hat f) = g(J(-2, 0, 0, 0), (0, 1, -1, 0)) = g((0, 0, -2, 0), (0, 1, -1, 0)) = 2\]

where \(g\) is the standard Riemannian metric on \(\C^4 \cong \R^8\). Next, we have that

\[\omega_K(\hat e, \hat f) = g(IJ(-2, 0, 0, 0), (0, 1, -1, 0)) = g((0, 0, -2i, 0), (0, 1, -1, 0)) = 0\]

Therefore, \(\tilde \omega_c(e, f) = 2\). \(\SL(2, \C)\) acts on \(\mu_c^{-1}(0)\) via

\[\gamma \cdot (\alpha, \beta) = (\gamma\alpha, \beta \gamma^{-1})\]

which means that

\[\Phi(\gamma \cdot (\alpha, \beta)) = \gamma\Phi(\alpha, \beta)\gamma^{-1}\]

i.e. if \(\psi_\gamma(\alpha, \beta) = (\gamma\alpha, \beta \gamma^{-1})\), then

% https://q.uiver.app/#q=WzAsNCxbMCwwLCJcXG11X2Neey0xfSgwKSJdLFsyLDAsIlxcbXVfY157LTF9KDApIl0sWzAsMiwiXFxPcmIoQSkiXSxbMiwyLCJcXE9yYihBKSJdLFswLDIsIlxcUGhpIiwyXSxbMSwzLCJcXFBoaSJdLFswLDEsIlxccHNpX1xcZ2FtbWEiXSxbMiwzLCJcXEFkX1xcZ2FtbWEiLDJdXQ==
\[\begin{tikzcd}[ampersand replacement=\&]
	{\mu_c^{-1}(0)} \&\& {\mu_c^{-1}(0)} \\
	\\
	{\Orb(A)} \&\& {\Orb(A)}
	\arrow["\Phi"', from=1-1, to=3-1]
	\arrow["\Phi", from=1-3, to=3-3]
	\arrow["{\psi_\gamma}", from=1-1, to=1-3]
	\arrow["{\Ad_\gamma}"', from=3-1, to=3-3]
\end{tikzcd}\]

commutes. We would like to show that \(\Ad_\gamma^*\tilde\omega_c = \tilde\omega_c\). Since \(\Phi\) is a surjective submersion, it suffices to show that

\[\Phi^*\tilde\omega_c = \Phi^*\Ad_g^*\tilde\omega_c\]

i.e.

\[\psi_\gamma^*i^*\omega_c = i^*\omega_c\]

That is,

\[\omega_c(\dd\psi_\gamma(u), \dd\psi_\gamma(v)) = \omega_c(u, v)\]

for all \(u, v \in \T_{(1,0,0,1)}\mu_c^{-1}(0)\). In particular, we need to check this on the subspace spanned by \(\hat e, \hat f\). This means that all we need to compute is

\[\omega_c(\dd\psi_\gamma(\hat e), \dd\psi_\gamma(\hat f))\]

Since \(\psi_\gamma\) gives a linear map \(M \to M\), \(\dd\psi_\gamma = \psi_\gamma\). If \(\gamma = \begin{pmatrix}
    a & b \\ c & d
\end{pmatrix}\), then it is easy to show that

\[\dd\psi_\gamma(\hat e) = (-2a, -2c, 0, 0) \qquad \dd\psi_\gamma(\hat f) = (b, d, -d, b)\]

This then shows that

\begin{align*}
    \omega_J(\dd\psi_\gamma(\hat e), \dd\psi_\gamma(\hat f)) &= g(J(-2a, -2c, 0, 0), (b, d, -d, b)) \\
    &= g((0, 0, -2a, -2c), (b, d, -d, b)) \\
    &= \Re(2ad - 2bc) \\
    &= 2
\end{align*}

and

\begin{align*}
    \omega_K(\dd\psi_\gamma(\hat e), \dd\psi_\gamma(\hat f)) &= g(IJ(-2a, -2c, 0, 0), (b, d, -d, b)) \\
    &= g((0, 0, -2ai, -2ci), (b, d, -d, b)) \\
    &= \Re((2ad - 2bc)i) \\
    &= 0
\end{align*}

Therefore, we have that \(\tilde\omega_c\) is invariant under the adjoint action of \(\SL(2, \C)\).

\printbibliography

\end{document}
