\documentclass{article}

\usepackage{../../Style}
\usepackage[english]{babel}
\usepackage{biblatex}

\DeclareMathOperator{\Sp}{Sp}
\newcommand{\sslash}{/\!/}
\newcommand{\ssslash}{/\!/\!/}

\newcommand{\rU}{\mathrm{U}}
\renewcommand{\sl}{\mfs\mfl}
\renewcommand{\sp}{\mfs\mfp}

\DeclareMathOperator{\gr}{grad}

\addbibresource{../../bibliography.bib}

\title{Ideas from \citetitle{kobak_classical_1996}}

\author{Shing Tak Lam}

\begin{document}

\maketitle

In this note, we provide a high level overview for the paper \cite{kobak_classical_1996} by \citeauthor{kobak_classical_1996}. We will use the same notation and numbering as in the paper, but we will omit technical details from the proofs.

\section{Introduction}

Here, we define the hyperK\"ahler structure on \(\bb H^n\), where the complex structures are given by right multiplication by \(-i, -j, -k\) respectively, and the Riemannian metric is the Euclidean metric from the isomorphism \(\bb H^n \cong \R^{4n}\) of vector spaces. For a subgroup \(H\) of \(\Sp(N)\), \(H\) acts (on the left) on \(\bb H^n\), and the \emph{hyperK\"ahler moment map} for this action is an equivariant map

\[\mu : \bb H^n \to \mfh^* \otimes \Im(\bb H) \cong \mfh \otimes \Im(\bb H)\]

where \(\dd(\mu^X) = X\lrcorner \eta\), \(\eta = \omega_I i + \omega_J j + \omega_K k\) the quaternion values form given by the symplectic forms. What this means is that if we write

\[\mu = \mu_I i + \mu_J j + \mu_K k\]

then \(\mu_I\) is a moment map for the action of \(H\) on \(\bb H^n\) with respect to the complex structure \(I\), and so on.

Throughout, we will have the moment map

\[\mu^X(q) = -\overline q^\T X q\]

for \(X \in \mfh\), which we consider to be an \(N \times N\) quaternionic matrix.

\section{The Constructions}

First of all, we reduce to the case where \(\mcO\) is an adjoint orbit of a classical simple Lie algebra over \(\C\), since in the general case we can take product/sums.

In each case, we first specify a hyperK\"ahler vector space \(M\) and the group \(G\), then we prove that the complex symplectic quotient by \(G^\C\) is what we want, then we show that the complex quotient is the same as the hyperK\"ahler quotient.

\subsection{The Special Linear Group}

In this case, let \(V_0, \dots, V_k\) be Hermitian vector spaces, \(\dim(V_i) = n_i\), \(n_0 = 0\) and \(n_k = n\). Then we can define the vector space

\[M = \bigoplus_{i=0}^{k-1} \left(\Hom(V_i, V_{i+1}) \oplus \Hom(V_{i+1}, V_i)\right)\]

and we write each point \(p = (\alpha_i, \beta_i)\) as the diagram

% https://q.uiver.app/#q=WzAsNSxbMCwwLCIwPVZfMCJdLFsyLDAsIlZfMSJdLFs0LDAsIlZfMiJdLFs2LDAsIlxcY2RvdHMiXSxbOCwwLCJWX2s9XFxDXm4iXSxbMCwxLCJcXGFscGhhXzAiLDAseyJvZmZzZXQiOi0xfV0sWzEsMiwiXFxhbHBoYV8xIiwwLHsib2Zmc2V0IjotMX1dLFsyLDMsIlxcYWxwaGFfMiIsMCx7Im9mZnNldCI6LTF9XSxbMyw0LCJcXGFscGhhX3trLTF9IiwwLHsib2Zmc2V0IjotMX1dLFs0LDMsIlxcYmV0YV97ay0xfSIsMCx7Im9mZnNldCI6LTF9XSxbMywyLCJcXGJldGFfMiIsMCx7Im9mZnNldCI6LTF9XSxbMiwxLCJcXGJldGFfMSIsMCx7Im9mZnNldCI6LTF9XSxbMSwwLCJcXGJldGFfMCIsMCx7Im9mZnNldCI6LTF9XV0=
\[\begin{tikzcd}
	{0=V_0} && {V_1} && {V_2} && \cdots && {V_k=\C^n}
	\arrow["{\alpha_0}", shift left=1, from=1-1, to=1-3]
	\arrow["{\alpha_1}", shift left=1, from=1-3, to=1-5]
	\arrow["{\alpha_2}", shift left=1, from=1-5, to=1-7]
	\arrow["{\alpha_{k-1}}", shift left=1, from=1-7, to=1-9]
	\arrow["{\beta_{k-1}}", shift left=1, from=1-9, to=1-7]
	\arrow["{\beta_2}", shift left=1, from=1-7, to=1-5]
	\arrow["{\beta_1}", shift left=1, from=1-5, to=1-3]
	\arrow["{\beta_0}", shift left=1, from=1-3, to=1-1]
\end{tikzcd}\]

We have a left \(\bb H\) action on \(M\), given by

\[i(\alpha_i, \beta_i) = (i\alpha_i, i\beta_i) \quad j(\alpha_i, \beta_i) = (-\beta_i^*, \alpha_i^*)\]

which makes \(M\) into a quaternionic vector space. In this case, the Lie group action of \(G = \rU(n_1) \times \cdots \times \rU(n_{k-1})\) on \(M\) is

\begin{equation}
    \label{eq:group-action}
    \begin{split}
        \alpha_i &\mapsto g_{i+1}\alpha_ig_i^{-1} \\
        \beta_i &\mapsto g_i\beta_ig_{i+1}^{-1}
    \end{split}
\end{equation}

where \(g_i \in U_{n_i}\), \(g_0 = g_k = 1\). The moment map in this case is \(\mu = i\mu_r + 2k\mu_c : M \to \mfg^* \otimes \Im(\bb H)\), where (up to identifying \(\mfg^*\) with \(\mfg\) via the Killing form),

\begin{align*}
    \mu_r &= (\alpha_{i-1}\alpha_{i-1}^* - \beta_{i-1}^*\beta_{i-1} + \beta_i\beta_i^* - \alpha_i^*\alpha_i)_{i=1}^{k-1} &\in \mfg \otimes i\R = i\mfg = i\mfu(n_1, \C) \oplus \cdots \oplus i\mfu(n_{k-1}, \C) \\
    \mu_c &= (\alpha_{i-1}\beta_{i-1} - \beta_i\alpha_i)_{i=1}^{k-1} &\in \mfg \otimes \C = \gl(n_1, \C) \oplus \cdots \oplus \gl(n_{k-1}, \C)
\end{align*}

\subsubsection{The Complex Quotient}

Fix a point \(p = (\alpha_i, \beta_i) \in \mu_c^{-1}(0)\). In this case, we can define \(X = \alpha_{k-1}\beta_{k-1} \in \End(\C^n)\). Then \(X^k = 0\), as \(p \in \mu_c^{-1}(0)\). Moreover, the action of \(G^\C = \GL(n_1, \C) \times \cdots \times \GL(n_{k-1}, \C)\) preserves \(X\), so we have a well defined map

\begin{align*}
    \Phi^c : \mu_c^{-1}(0) / G^\C &\to \mcN \\
    (\alpha_i, \beta_i) &\mapsto \alpha_{k-1}\beta_{k-1}
\end{align*}

where \(\mcN\) is the nilpotent variety of \(\sl(n, \C)\)\footnote{Any nilpotent endomorphism necessarily has all eigenvalues being zero, and so it is trace free.}.

\begin{theorem}
    \label{thm:cx-quot-sln}
    The map \(\Phi^c\), restricted to the set of closed \(G^\C\) orbits, is injective. Furthermore, its image consists of a union of closures of nilpotent orbits in \(\sl(n, \C)\). If there exists \(X \in \sl(n, \C)\) such that \(\rank(X^i) = n_{k-i}\) for all \(i\), then the image is precisely the closure of the nilpotent orbit containing \(X\).
\end{theorem}

\begin{proof}
    [Proof sketch]

    First of all, notice that \cref{eq:group-action} defines a \(\GL(n, \C)\) action on \(M\), by taking \(g_k = g\), and \(g_i = 1\) for \(i < k\). This action preserves \(\mu_c^{-1}(0)\), and \(\Phi^c\) is equivariant with respect to this action\footnote{In fact, \(\Phi^c\) is a complex symplectic moment map for a \(\GL(n, \C)\) action on the quotient.}. Therefore, the image of \(\Phi^c\) is a union of nilpotent orbits.

    To show the injectivity statement, we use the \(\GL(n, \C)\) action to assume without loss of generality that \(X\) is in Jordan normal form. In fact, we can assume that \(X\) is a Jordan block, each \(\beta_i\) is surjective and each \(\alpha_i\) is injective using the \(G^\C\) action and the fact that the orbits are closed. In this case, by an appropriate choice of basis, using the \(G^\C \times \GL(n, \C)\)-action, we can assume \(\beta\) has matrix

    \[\begin{pmatrix}
        0 & 1 & \cdots & 0 \\
        \vdots & \ddots & \ddots & \vdots \\
        0 & \cdots & \cdots & 1 \\
    \end{pmatrix}\]

    With this, \(\alpha_1\) is upper triangular, and by induction all \(\alpha_i\) are upper triangular, and we get uniqueness.

    The rest of the statement follows from the facts that

    \begin{enumerate}
        \item If \(X\) is nilpotent, the numbers \(\rank(X^i)\) determines the Jordan normal form of \(X\), and so the nilpotent orbit of \(X\),
        \item If \(X, Y\) are nilpotent, then \(Y\) is in the closure of the nilpotent orbit containing \(X\) if and only if \(\rank(Y^i) \le \rank(X^i)\) for all \(i\).
    \end{enumerate}
\end{proof}

\subsubsection{Equivalence of K\"ahler and Complex Quotients}

We have the following result by Kirwan (paraphrased):

\begin{theorem}
    \label{thm:kirwan}
    Let \(X\) be a K\"ahler manifold, \(G\) a compact Lie group acting on \(X\) preserving the K\"ahler structure, such that \(G^\C\) also acts holomorphically on \(X\). Let \(\mu\) be the K\"ahler moment map for the action of \(G\), satisfying condition \((\star)\). Let

    \[X^{\min} = \left\{y\ \big\vert\ \text{ limit under steepest decent of }\norm{\mu}^2\text{ lies in }\mu^{-1}(0)\right\}\]

    Then \(x \in G^\C\mu^{-1}(0)\) if and only if \(x \in X^{\min}\) and the orbit \(G^\C x\) is closed in \(X^{\min}\). In this case, the map

    \[\mu^{-1}(0)/G \to G^\C\mu^{-1}(0)\sslash G^\C\]

    is a homeomorphism, where \(G^\C\mu^{-1}(0)\sslash G^\C\) is the set of closed \(G^\C\) orbits in \(G^\C \mu^{-1}(0)\).
\end{theorem}

We will return to the condition \((\star)\) later, but for now, we will first assume that \((\star)\) holds in the cases which we want, and then prove that it holds later on.

First, since \((M, \omega_I, \omega_J, \omega_K)\) is hyperK\"ahler, \((M, \omega_I)\) is K\"ahler. Moreover, the group \(G\) acts on \((M, \omega_I)\) preserving the K\"ahler structure, with moment map \(\mu_r\). Therefore, applying \cref{thm:kirwan}, we get that

\[\mu_r^{-1}(0)/G \cong G^\C \mu_r^{-1}(0)\sslash G^\C\]

Next, we assume \(M^{\min} = M\), so \(G^\C\mu_r^{-1}(0)\) is just the set of points for which the orbit \(G^\C x\) is closed. In this case, we have a natural inclusion

\[X = \mu_c^{-1}(0) \cap G^\C \mu_r^{-1}(0) \subseteq G^\C\mu_r^{-1}(0)\]

Since \(X\) is \(G^\C\) invariant, we have an induced map

\[X/G^\C \hookrightarrow G^\C\mu_r^{-1}(0)\sslash G^\C \cong \mu_r^{-1}(0)/G\]

Finally, we want to find the image of this map. But this is just \(\mu^{-1}(0)/G\), which is the hyperK\"ahler quotient. Therefore, all that remains is to show that \(M^{\min} = M\), and that \((\star)\) holds.

\underline{\(M^{\min} = M\):} For this, it suffices to show that the critical points of \(\norm{\mu_r}^2\) are global minima. Since \(\mu_r^* = \mu_r\), we have that \(\gr\left(\norm{\mu_r}^2\right) = 2(\dd\mu_r)\mu_r\), which vanishes if and only if \(\mu_r = 0\) and so \(\norm{\mu_r}^2 = 0\).

The condition \((\star)\) is that \underline{the trajectories of the gradient flow of \(\norm{\mu_r}^2\) are bounded}. In this case, we have that

\[\norm{\mu_r(x)}^2 \le \norm{x}^4\]

for all \(x \in M\). \textbf{The paper then claims that this implies each trajectory is bounded, but I don't see why this is true.}

\addtocounter{theorem}{4}

\begin{theorem}
    The hyperK\"ahler quotient of \(M\) by \(G\) is a union of nilpotent orbits in \(\sl(n, \C)\). If there is a nilpotent element \(X \in \sl(n, \C)\), with \(\rank(X^i) = n_{k-i}\) for all \(i\), then the quotient is isomorphic to the closure of the nilpotent orbit containing \(X\).
\end{theorem}

\subsection{Orthogonal and Symplectic Lie Algebras}

For \(\mfo(n, \C)\) and \(\sp(n, \C)\), we write

\begin{align*}
    \mfo(n, \C) &= \mfc^\C_0\\
    \sp(n, \C) &= \mfc^\C_1
\end{align*}

and the corresponding Lie groups \(C^\C_\delta\) are Lie groups acting on \(V = \C^{(1 + \delta) n}\) preserving a non-degenerate bilinear form \(B\) such that \(B(u, v) = (-1)^\delta B(v, u)\).

If \(X \in \mfc_\delta^\C\) is nilpotent, with \(X^k = 0\), then \(X^{k-i}V\) has a bilinear form \(B_i\), with

\begin{equation}
    \label{eq:bilinear-form}
    B_i(u, v) = (-1)^{k-i+\delta}B_i(v, u)
\end{equation}

As a matrix, \(B_{i-1} = XB_i\) restricted to \(X^{k-i}V\). Therefore, we consider the same construction as in the previous subsection, except we require each \(V_i\) to have a bilinear form \(B_i\) satisfying \cref{eq:bilinear-form}. Moreover, let \(A^\dagger\) be the adjoint of \(A\) with respect to \(B_i\), then we require that

\[(A^*)^\dagger = (A^\dagger)^*\]

Let \(M_\delta\) be the vector subspace of \(M\), given by

\[\beta_i = \alpha_i^\dagger\]

Note \((\alpha_i^\dagger)^\dagger = -\alpha_i\). Then \(M_\delta\) is also a flat hyperK\"ahler manifold. Let \(H = C_1 \times \cdots \times C_n\) be the subgroup of \(G\), where \(C_i\) preserves \(B_i\). In particular, \(C_i\) is \(\mathrm{O}(n_i)\) or \(\Sp(n_i/2)\). \(H\) acts on \(M_\delta\), and the moment map is the same as above, just with \(\beta_i = \alpha_i^\dagger\).

\begin{theorem}
    The hyperK\"ahler quotient of \(M_\delta\) by \(H\) is a union of closures of nilpotent orbits of \(C_\delta^\C\), where \(C_0^\C = \mathrm{O}(n, \C)\) and \(C_1^\C = \Sp(n/2, \C)\). Moreover, this quotient agrees with the algebraic quotient \(\mu_c^{-1}(0)\sslash H^\C\). If there is an \(X \in \mfc_\delta^\C\) with \(\rank(X^i) = n_{k-i}\) for all \(i\), then the hyperK\"ahler quotient is the closure of the nilpotent orbit containing \(X\).
\end{theorem}

\begin{proof}
    [Proof sketch]

    First, we note that nilpotent orbits in \(\mfo(n, \C)\) and \(\sp(n/2, \C)\) are the intersections of the \(\sl(n, \C)\) orbits with \(\mfc_\delta^\C\). Therefore, suffices to show that for any \(X \in \mfc_\delta^\C\), there exists \((\alpha_1, \dots, \alpha_{k-1})\) in a closed \(H^\C\)-orbit of \(\mu_c^{-1}(0) \cap M_\delta\), with \(\alpha_{k-1}\alpha_{k-1}^\dagger = X\). This is because we have natural maps

    \[\frac{\mu_c^{-1}(0) \cap M_\delta}{H^\C} \to \frac{\mu_c^{-1}(0)}{G^\C} \stackrel{\Phi^c}{\to} \mcN\]

    for which the composition is \([(\alpha_i)] \mapsto \alpha_{k-1}\alpha_{k-1}^\dagger\).

    To prove the claim, we again consider Jordan blocks, and define the \(B_i\)s and \(\alpha_i\)s appropriately.
\end{proof}

\section{Consequences and Examples}

\begin{lemma}
    Let \(H\) be a Lie group acting on \(\bb H^N\) preserving the complex structures. Let a non-zero quaternion \(a \in \bb H^*\) act on \(q \in \bb H^N\) on the right, that is, \(q \mapsto qa^{-1}\), and on \(p \in \Im(\bb H)\) by conjugation. That is, \(p \mapsto apa^{-1}\). Then the map \(\mu : \bb H^N \to \mfh^* \otimes \Im(\bb H)\) defined as above is the unique moment map for the action of \(H\) on \(\bb H^N\) which is equivariant with respect to the action of \(\bb H^*\).
\end{lemma}

Using this, we get that the set \(\mu^{-1}(0)\) is \(\bb H^*\)-invariant, and if \(K\) is another Lie group so that \(H \times K\) also acts on \(\bb H^N\) preserving the complex structures, then the moment map for the action of \(H \times K\) is the direct sum of the moment maps for \(H\) and for \(K\), and the hyperK\"ahler quotient by \(H \times K\) is the hyperK\"ahler quotient by \(H\) followed by the hyperK\"ahler quotient by \(K\).

\subsection{Quaternionic K\"ahler metrics}

Omitted.

\subsection{Finite Quotients}

\subsection{HyperK\"ahler Quotients}

\printbibliography

\end{document}
