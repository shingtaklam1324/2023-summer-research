\documentclass{article}

\usepackage{../Style}

\DeclareMathOperator{\SU}{SU}
\newcommand{\su}{\mfs\mfu}

\title{\(\SU(n)/T \cong \SL(n, \C)/P\)}
\author{Shing Tak Lam}

\begin{document}

\maketitle

Consider the following Lie groups:

\[\SU(n) \qqtext{with} T = \left\{\begin{pmatrix}
    \lambda_1 \\ & \ddots \\ && \lambda_n
\end{pmatrix}\right\} \le \SU(n)\]

and

\[\SL(n, \C) \qqtext{with} P = \left\{\begin{pmatrix}
    \lambda_1 \\ & \ddots \\ * && \lambda_n
\end{pmatrix}\right\} \le \SL(n, \C)\]

where \(P\) is the parabolic subgroup of lower triangular matrices. Consider the composition \(\varphi : \SU(n) \to \SL(n, \C)/P\) given by the composition 

% https://q.uiver.app/#q=WzAsMyxbMCwwLCJcXFNVKG4pIl0sWzIsMCwiXFxTTChuKSJdLFs0LDAsIlxcU0wobiwgXFxDKS9QIl0sWzAsMSwiIiwwLHsic3R5bGUiOnsidGFpbCI6eyJuYW1lIjoiaG9vayIsInNpZGUiOiJ0b3AifX19XSxbMSwyLCIiLDAseyJzdHlsZSI6eyJoZWFkIjp7Im5hbWUiOiJlcGkifX19XV0=
\[\begin{tikzcd}
	{\SU(n)} && {\SL(n)} && {\SL(n, \C)/P}
	\arrow[hook, from=1-1, to=1-3]
	\arrow[two heads, from=1-3, to=1-5]
\end{tikzcd}\]

Suppose \(\varphi(g) = \varphi(h)\). That is, \(gP = hP\). This is true if and only if there exists \(p \in P\), such that \(h = gp\). In this case, \(p = g^{-1}h \in \SU(n)\), therefore, \(p \in \SU(n) \cap P = T\), since \(p^\dagger = p^{-1}\) is also lower triangular. This means that \(\varphi\) induces a homeomorphism \(\SU(n)/T \cong \SL(n, \C)/P\).

\end{document}
